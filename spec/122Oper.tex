\section{Описание операций}\label{Oper.Descr}

\subsection{Активация личного ключа}\label{Oper.Descr.ActivateKey}

Для активации личного ключа используется команда <Activate>,
которая определяется согласно таблице~\ref{Table.Oper.ActivateCmd}.

\begin{table}[hbt]
\caption{}\label{Table.Oper.ActivateCmd}
\begin{tabular}{|c|p{14cm}|}
\hline
Компонент & Описание\\
\hline
\hline
INS & $\hex{44}$: активировать\\
\hline
$\text{P1} \parallel \text{P2}$ & $\hex{2100}$: 
активировать ключ, определяемый CDF\\
\hline
CDF &  $\der(\hex{84}, X)$,   
где $X$ определяет идентификатор личного ключа 
(см. таблицу~\ref{Table.Oper.KeyRef}) \\ 
\hline
RDF &  --- \\
\hline
$\text{SW1} \parallel \text{SW2}$ & $\hex{9000}$: ключ активирован успешно \\
%  & $ \hex{6A88}$: cсылочные данные (ключ) не найдены \\
  & другое: см. таблицу~\ref{Table.Errors.General} \\
\hline
\end{tabular}
\end{table}


При генерации личного ключа (см.~\ref{Oper.Descr.GenKeys})
он автоматически активируется, 
и в вызове команды <Activate> нет явной необходимости. 
Вызов команды может потребоваться после 
принудительной деактивации ключа (см.~\ref{Oper.Descr.DeactivateKey}).  

Если ключ уже активирован, то должен быть 
возвращен статус $\text{SW1} \parallel \text{SW2} = \hex{9000}$.

Команда может быть вызвана для приложения eSign в состояниях PS, AS:AT.
В состоянии AS:AT команда может быть вызвана только
авторизованным терминалом, в сертификате которого задано 
право активировать ключи (см.~\ref{DATA.Access}).

Команда требует предварительной аутентификации по PIN.


%%%%%%%%%%%%%%%%%%%%%%%%%%%%%%%%%%%%%%%%%%%%%%%%%%%%%%%%%
\subsection{Активация PIN}\label{Oper.Descr.ActivatePIN}

Для активации PIN (см.~\ref{OBJ.PWD})  используется
команда <Activate>, которая определяется согласно 
таблице~\ref{Table.Oper.ActivatePINCmd}.

\begin{table}[hbt]
\caption{}\label{Table.Oper.ActivatePINCmd}
\begin{tabular}{|c|p{14cm}|}
\hline
Компонент & Описание\\
\hline
\hline
INS & $\hex{44}$: активировать\\
\hline
P1 & $\hex{10}$: активировать пароль, определяемый P2\\
\hline
P2 & $\hex{03}$: PIN \\
\hline
CDF &  ---  \\
\hline 
RDF &  --- \\
\hline
$\text{SW1} \parallel \text{SW2}$ & 
  $\hex{9000}$:  PIN активирован успешно\\
  & другое: см. таблицу~\ref{Table.Errors.General}\\
\hline
\end{tabular}
\end{table}

Первоначально PIN является активированным, и в вызове команды <Activate> 
нет явной необходимости. Вызов команды может потребоваться после 
принудительной деактивации PIN (см.~\ref{Oper.Descr.DeactivatePIN}). 

Если PIN уже активирован, то должен быть возвращен 
статус $\text{SW1} \parallel \text{SW2} = \hex{9000}$.

Команда может быть вызвана для приложения eSign 
в состояниях PS, AS:AT и для приложения eID в состоянии AS:AT. 
В состоянии AS:AT команда может быть вызвана 
только авторизованным терминалом, в сертификате которого задано 
право активации пароля PIN (см.~\ref{DATA.Access}).

Команда требует предварительной аутентификации по PUK.


%%%%%%%%%%%%%%%%%%%%%%%%%%%%%%%%%%%%%%%%%%%%%%%%%%%%
\subsection{Выбор мастер-файла}
\label{Oper.Descr.SelectMF}

Для выбора мастер-файла (см. приложение~\ref{FILES}) 
используется команда <Select File>, 
которая определяется согласно 
таблице~\ref{Table.Oper.SelectMFCmd}.

\begin{table}[hbt]
\caption{}\label{Table.Oper.SelectMFCmd}
\begin{tabular}{|c|p{14cm}|}
\hline
Компонент & Описание \\
\hline
\hline
INS & $\hex{A4}$: выбрать файл\\ 
\hline
$\text{P1} \parallel \text{P2}$ & $\hex{0000}$: выбор мастер-файла\\
\hline
CDF & --- \\
\hline 
RDF &  --- \\
\hline
$\text{SW1}\parallel\text{SW2}$ & 
$\hex{9000}$: мастер-файл выбран успешно \\
  & другое: см. таблицу~\ref{Table.Errors.General} \\
\hline
\end{tabular}
\end{table}

Первоначально выделенный файл (см. приложение~\ref{FILES}), 
соответствующий мастер-файлу, является текущим,
и в вызове команды <Select File> нет явной необходимости. 
Вызов команды может потребоваться после 
выбора приложения (см.~\ref{Oper.Descr.SelectApp})
или выбора элементарного файла (см.~\ref{Oper.Descr.SelectEF}). 

Если мастер-файл уже выбран, то должен быть возвращен 
статус $\text{SW1} \parallel \text{SW2} = \hex{9000}$.

Команда может быть вызвана в любом из состояний на 
любом уровне.

После успешного выполнения команды выделенный файл, 
соответствующий мастер-файлу, становится текущим.


%%%%%%%%%%%%%%%%%%%%%%%%%%%%%%%%%%%%%%%%%%%%%%%%%%%%%%%%%
\subsection{Выбор прикладной программы}
\label{Oper.Descr.SelectApp}

Для выбора прикладной программы используется 
команда <Select File>, 
которая определяется согласно 
таблице~\ref{Table.Oper.SelectAppCmd}.

\begin{table}[hbt]
\caption{}\label{Table.Oper.SelectAppCmd}
\begin{tabular}{|c|p{14cm}|}
\hline
Компонент & Описание \\
\hline
\hline
INS & $\hex{A4}$: выбрать файл\\ 
\hline
P1 & $\hex{04}$: выбор прикладной программы\\
\hline
P2 & $\hex{0C}$: возврат информации о файле не требуется \\
\hline
CDF & AID прикладной программы (см. приложение~\ref{FILES})\\
\hline 
RDF &  --- \\
\hline
$\text{SW1}\parallel\text{SW2}$ & 
$\hex{9000}$: прикладная программа выбрана успешно \\
  & $\hex{6A82}$: файл не найден \\
  & другое: см. таблицу~\ref{Table.Errors.General}\\
\hline
\end{tabular}
\end{table}

Команда может быть вызвана в состояниях PS и AS
на любом уровне.

Для выбора приложения eSign команда требует 
предварительной аутентификации по PIN или PUK.

Для выбора приложения eID команда требует 
предварительной аутентификации по CAN, PIN или PUK.

Если прикладная программа уже выбрана, то должен быть возвращен
статус $\text{SW1} \parallel \text{SW2} = \hex{9000}$.

После успешного выполнения команды выделенный файл 
(см. приложение~\ref{FILES}), 
соответствующий выбранной прикладной программе, становится текущим.


%%%%%%%%%%%%%%%%%%%%%%%%%%%%%%%%%%%%%%%%%%%%%%%%%%%%%%%%%%%%%%%%%%%%%%%
\subsection{Выбор элементарного файла}
\label{Oper.Descr.SelectEF}

Для выбора элементарного файла используется 
команда <Select File>, 
которая определяется согласно 
таблице~\ref{Table.Oper.SelectEFCmd}.

\begin{table}[hbt]
\caption{}\label{Table.Oper.SelectEFCmd}
\begin{tabular}{|c|p{14cm}|}
\hline
Компонент & Описание \\
\hline
\hline
INS & $\hex{A4}$: выбрать файл\\ 
\hline
P1 & $\hex{02}$: выбор элементарного файла для текущей прикладной программы\\
\hline
P2 & $\hex{0C}$: возврат информации о файле не требуется \\
\hline
CDF & FID файла для текущей прикладной программы (см. приложение~\ref{FILES})\\
\hline 
RDF &  --- \\
\hline
$\text{SW1}\parallel\text{SW2}$ & 
$\hex{9000}$: элементарный файл выбран успешно \\
  & $\hex{6A82}$: файл не найден \\
  & другое: см. таблицу~\ref{Table.Errors.General}\\
\hline
\end{tabular}
\end{table}

Команда может вызываться в состояниях PS, AS:AT 
для прикладной программы eSign и в состоянии 
AS:AT для прикладной программы eID.
В состоянии AS:AT команда может вызываться только 
авторизованным терминалом, в сертификате которого
задано право доступа к соответствующему элементарному файлу (см.~\ref{DATA.Access}).

Для выбора элементарных файлов приложения eSign команда требует 
предварительной аутентификации по PIN.

Для выбора элементарных файлов приложения eID команда требует 
предварительной аутентификации по CAN или PIN.

Для выбора элементарных файлов в состоянии AS:AT
команда требует взаимной аутентификации КТ и терминала
по протоколу BAUTH (см.~\ref{Oper.Descr.SetBAUTH}).

Элементарные файлы, которые могут быть выбраны 
для приложений eSign и eID, приводятся в приложении~\ref{FILES}. 

После успешного выполнения команды выбранный элементарный файл
становится текущим.
Выбор элементарного файла может потребоваться для
чтения (см.~\ref{Oper.Descr.Read}) или 
обновления (см.~\ref{Oper.Descr.Update}) данных.

%%%%%%%%%%%%%%%%%%%%%%%%%%%%%%%%%%%%%%%%%%%%%%%%%%%%%%%%%%%%%
\subsection{Выполнение основных шагов протокола BAUTH}
\label{Oper.Descr.GABAUTH} 

Для выполнения основных шагов протокола BAUTH 
используется команда <General Authenticate>, 
которая определяется согласно 
таблице~\ref{Table.Oper.GABAUTHCmd}.

\begin{table}[hbt]
\caption{}\label{Table.Oper.GABAUTHCmd}
\begin{tabular}{|c|p{14cm}|}
\hline
Компонент & 	Описание \\
\hline
\hline
INS & $\hex{86}$: общая аутентификация \\
\hline
$\text{P1} \parallel \text{P2}$ & $\hex{0000}$: протокол уже задан\\ 
\hline
CDF & отсутствует или $\der(\hex{7C}, X)$, 
где~$X$~--- сообщение протокола, сформированное терминалом
(см. таблицу~\ref{Table.Oper.BAUTH})\\
\hline 
RDF & отсутствует или $\der(\hex{7C}, Y)$, где~$Y$~--- 
сообщение протокола, сформированное КТ 
(см. таблицу~\ref{Table.Oper.BAUTH})\\
\hline
$\text{SW1} \parallel \text{SW2}$ & $\hex{9000}$: протокол (шаг) выполнен успешно \\
& $\hex{6300}$: протокол (шаг) выполнен с ошибкой\\
  & другое: см. таблицу~\ref{Table.Errors.General} \\
\hline
\end{tabular}
\end{table}

Для выполнения основных шагов протокола BAUTH
должна вызываться цепочка команд <General Authenticate> 
с входными и выходными данными из таблицы~\ref{Table.Oper.BAUTH}, 
которые определяются в соответствии с~\ref{CRYPTO.BAUTH}. 
При односторонней аутентификации компонент RDF в ответе на вторую команду
в цепочке не возвращается.

\begin{table}[hbt]
\caption{Данные цепочки команд, реализующих протокол BAUTH}
\label{Table.Oper.BAUTH}
\begin{tabular}{|c|c|c|}
\hline
№ вызова & Данные в команде & Данные в ответе\\
\hline
\hline
1 & --- & $\der(\hex{80}, \text{M1})$\\
\hline
2 & $\der(\hex{81}, \text{M2})$ & 
$\der(\hex{82}, \text{M3})$  \\
\hline
\end{tabular}
\end{table}

Команда может вызываться на уровне мастер-файла в состоянии PS 
непосредственно после проверки сертификата терминала
(см.~\ref{Oper.Descr.VerifyCert}).

После успешного выполнения протокола BAUTH между КТ и терминалом 
устанавливается защищенное соединение (см.~\ref{CMDS.SM}).

%%%%%%%%%%%%%%%%%%%%%%%%%%%%%%%%%%%%%%%%%%%%%%%%%%%%%%%%%%%%%%%%
\subsection{Выполнение шагов протокола BPACE}
\label{Oper.Descr.GABPACE} 

Для выполнения шагов протокола BPACE 
используется команда <General Authenticate>,
которая определяется согласно 
таблице~\ref{Table.Oper.GABPACECmd}.

\begin{table}[hbt]
\caption{}\label{Table.Oper.GABPACECmd}
\begin{tabular}{|c|p{14cm}|}
\hline
Компонент & 	Описание \\
\hline
\hline
INS & $\hex{86}$: общая аутентификация \\
\hline
$\text{P1} \parallel \text{P2}$ & $\hex{0000}$: протокол уже задан\\ 
\hline
CDF & $\der(\hex{7C}, X)$, 
где~$X$~--- сообщение протокола, сформированное КП (см. 
таблицу~\ref{Table.Oper.BPACE})\\ 
\hline 
RDF & $\der(\hex{7C}, Y)$, где~$Y$~--- 
сообщение протокола, сформированное КТ (см. таблицу~\ref{Table.Oper.BPACE})\\
\hline
$\text{SW1} \parallel \text{SW2}$ & 
  $\hex{9000}$: протокол (шаг) выполнен успешно \\
  & $\hex{63CX}$: протокол (шаг) выполнен с ошибкой, 
значение \texttt{X} содержит количество оставшихся попыток аутентификации 
(при $\texttt{X} = 1$ пароль приостановлен, а при $\texttt{X} = 0$~--- 
заблокирован) \\
  & $\hex{6984}$: пароль деактивирован \\
  & другое: см. таблицу~\ref{Table.Errors.General} \\
\hline
\end{tabular}
\end{table}

Для выполнения шагов протокола BPACE должна вызываться цепочка 
команд <General Authenticate> с входными и выходными данными 
из таблицы~\ref{Table.Oper.BPACE}, 
которые определяются в соответствии с~\ref{CRYPTO.BPACE}. 

\begin{table}[hbt]
\caption{Данные цепочки команд, реализующих протокол BPACE}
\label{Table.Oper.BPACE}
\begin{tabular}{|c|c|c|}
\hline
№ вызова & Данные в команде & Данные в ответе\\
\hline
\hline
 & <General Authenticate> &  \\
\hline
\hline
1 & $\der(\hex{80}, \text{M1})$ & 
$\der(\hex{81}, \text{M2})$\\
\hline
2 & $\der(\hex{82}, \text{M3})$ & 
$\der(\hex{83}, \text{M4})$\\
\hline
\end{tabular}
\end{table}

Команда может вызываться на уровне мастер-файла в любом состоянии
непосредственно после инициализации протокола 
BPACE (см.~\ref{Oper.Descr.SetBPACE}).

Успешное выполнение протокола BPACE с использованием пароля PIN 
устанавливает статус подтверждения пароля PIN.
В свою очередь, неуспешное выполнение протокола BPACE с использованием 
пароля PIN сбрасывает статус подтверждения пароля PIN, 
если он был установлен ранее.

После успешного выполнения протокола BPACE между КТ и КП 
устанавливается защищенное соединение (см.~\ref{CMDS.SM}).

%%%%%%%%%%%%%%%%%%%%%%%%%%%%%%%%%%%%%%%%%%%%%%%%%%%%%%%%%%%%%%%%%%%%%
\subsection{Выработка подписи}
\label{Oper.Descr.Signature}

Для выработки подписи используется 
команда <PSO: Compute Digital Signature>
(аббр. от <<Perform Security Operation: Compute Digital Signature>>),
которая определяется согласно 
таблице~\ref{Table.Oper.SignatureCmd}.

\begin{table}[hbt]
\caption{}\label{Table.Oper.SignatureCmd}
\begin{tabular}{|c|p{14cm}|}
\hline
Компонент & Описание\\ 
\hline
\hline
INS & $\hex{2A}$: управление средой безопасности \\
\hline
$\text{P1} \parallel \text{P2}$ & $\hex{9E9A}$: выработать
электронную подпись \\ 
\hline
CDF & Хэш-значение, вычисленное от данных\\
\hline 
RDF &  --- \\
\hline
$\text{SW1} \parallel \text{SW2}$ & 
  $\hex{9000}$: подпись выработана успешно \\
  & другое: см. таблицу~\ref{Table.Errors.General} \\
\hline
\end{tabular}
\end{table}

В компоненте CDF команды должно передаваться хэш-значение, которое вычислено
от подписываемых данных алгоритмом хэширования, указанным 
при инициализации алгоритма подписи, и длина которого соответствует
уровню стойкости личного ключа, выбранного при инициализации алгоритма подписи.

При выработке подписи используется алгоритм \texttt{bign-sign}
(см.~\ref{CRYPTO.StdAlg}) со стандартными параметрами, 
которые определяются неявно по личному ключу.  

%Команда требует предварительной аутентификации по 
%протоколу BPACE c паролем PIN. 

Команда может вызываться для приложения eSign в состояниях 
PS и AS:AT непосредственно после инициализации алгоритма подписи
(см.~\ref{Oper.Descr.SetDST}).
%В состоянии AS:AT команда может вызываться только авторизованным 
%терминалом с правом выработки подписи (см. \ref{DATA.Access}).

Выполнение команды приводит к сбрасыванию статуса подтверждения пароля PIN
после успешного или неуспешного выполнения $N$ команд, где $N$ --- значение, 
задаваемое при инициализации протокола BPACE (см.~\ref{Oper.Descr.SetBPACE}).
Для установки статуса подтверждения пароля PIN 
необходимо либо подтвердить пароль PIN (см.~\ref{Oper.Descr.VerifyPIN}), 
либо повторно выполнить аутентификацию по PIN (см.~\ref{Oper.Seq.BPACE}).


%%%%%%%%%%%%%%%%%%%%%%%%%%%%%%%%%%%%%%%%%%%%%%%%%%%%%%%%%%%%%%%%%%%%%%%
\subsection{Генерация ключевой пары}\label{Oper.Descr.GenKeys}

Для генерации ключевой пары (личного и открытого ключа) прикладной программы eSign
используется команда <Generate Asymmetric Key Pair>. При выполнении команды 
сгенерированный личный ключ сохраняется в КТ,
а открытый ключ возвращается как данные ответа.
После генерации личный ключ может использоваться 
для выработки подписи (см.~\ref{Oper.Descr.Signature}) и
разбора токена ключа (см.~\ref{Oper.Descr.Decipher}).
Команда определяется согласно 
таблице~\ref{Table.Oper.GenKeysCmd}.

\begin{table}[hbt]
\caption{}\label{Table.Oper.GenKeysCmd}
\begin{tabular}{|c|p{14cm}|}
\hline
Компонент & Описание\\
\hline
\hline
INS & $\hex{47}$: сгенерировать ключевую пару \\
\hline
$\text{P1} \parallel\text{P2}$ & $\hex{8200}$:
вернуть открытый ключ \\
\hline
CDF & $\der(\hex{B6},\der(\hex{84}, X))$,
где $X$ определяет идентификатор генерируемого личного ключа
(см. таблицу~\ref{Table.Oper.KeyRef}) \\
\hline 
RDF & $\der(\hex{7F49}, Y)$, где $Y$ является открытым ключом длины $4l$ бит\\
\hline
$\text{SW1} \parallel \text{SW2}$ & 
$\hex{9000}$: ключевая пара сгенерирована успешно \\
  & $\hex{6984}$: ключ уже существует \\
%  & $\hex{6A88}$: cсылочные данные (ключ) не найдены \\
  & другое: см. таблицу~\ref{Table.Errors.General} \\
\hline
\end{tabular}
\end{table}

Допустимые значения идентификаторов личного ключа определяются
в соответствии с таблицей~\ref{Table.Oper.KeyRef} и зависят от
уровня стойкости личного ключа и состояния КТ:
младшя тетрада идентификатора характеризует уровень стойкости ключа,
а старшая~--- состояние, в котором ключ может использоваться.
КТ должен поддерживать генерацию ключей уровня стойкости 
$l=128$ и может поддерживать генерацию ключей 
уровней стойкости $l=192, 256$.

\begin{table}[hbt]
\caption{Допустимые значения идентификаторов личного ключа}
\label{Table.Oper.KeyRef}
\begin{tabular}{|c|c|c|c|}
\hline
Уровень стойкости & \multicolumn{2}{|c|}{Значение } & Поддержка\\
\cline{2-3}
ключа & Состояние PS & Состояние AS & \\
\hline
\hline
128 & $\hex{01}$ & $\hex{11}$ & Обязательна \\
192 & $\hex{02}$ & $\hex{12}$ & Необязательна\\
256 & $\hex{03}$ & $\hex{13}$ & Необязательна\\
\hline
\end{tabular}
\end{table}

При генерации ключей используется алгоритм \texttt{bign-sign}
со стандартными параметрами, которые определяются неявно
по идентификатору личного ключа. 

Команда может вызываться для приложения eSign в состояниях 
PS и AS:AT. В состоянии AS:AT команда может вызываться 
только авторизованным терминалом, в сертификате которого задано право
генерации ключей в терминальном режиме (см.~\ref{DATA.Access}).

Команда требует предварительной аутентификации по PIN. При вызове
команды статус подтверждения пароля PIN должен быть 
установлен.

Выполнение команды приводит к сбрасыванию статуса подтверждения пароля PIN.
Для установки статуса подтверждения пароля PIN 
необходимо либо подтвердить пароль PIN (см.~\ref{Oper.Descr.VerifyPIN}), 
либо повторно выполнить аутентификацию по PIN (см.~\ref{Oper.Seq.BPACE}).

Если при выполнении команды личный ключ с указанным в команде идентификатором
уже существует, то должен быть возвращен статус 
$\text{SW1} \parallel \text{SW2} = \hex{6984}$. 
При этом для генерации нового ключа старый ключ должен быть предварительно 
уничтожен командой <Terminate> (см.~\ref{Oper.Descr.Terminate}).


%%%%%%%%%%%%%%%%%%%%%%%%%%%%%%%%%%%%%%%%%%%%%%%%%%%%%%%%%%%%%%%%%%%%%%%
\subsection{Деактивация личного ключа}
\label{Oper.Descr.DeactivateKey} 

Для деактивации личного ключа 
используется команда <Deactivate>,
которая определяется согласно 
таблице~\ref{Table.Oper.DeactivateKeyCmd}.

\begin{table}[hbt]
\caption{}\label{Table.Oper.DeactivateKeyCmd}
\begin{tabular}{|c|p{14cm}|}
\hline
Компонент & Описание\\
\hline
\hline
INS & $\hex{04}$: деактивировать\\
\hline
$\text{P1} \parallel \text{P2}$ & $\hex{2100}$: 
деактивировать ключ, определяемый полем данных\\
\hline
CDF &  $\der(\hex{84}, X)$,   
где $X$ определяет идентификатор личного ключа 
(см. таблицу~\ref{Table.Oper.KeyRef})\\
\hline 
RDF & --- \\
\hline
$\text{SW1} \parallel \text{SW2}$ & 
$\hex{9000}$: ключ деактивирован успено \\
%  & $\hex{6A88}$: cсылочные данные (ключ) не найдены \\
  & другое: см. таблицу~\ref{Table.Errors.General} \\
\hline
\end{tabular}
\end{table}

Команда может вызываться для приложения eSign в состояниях 
PS и AS:AT. В состоянии AS:AT команда может вызываться 
только авторизованным терминалом, в сертификате которого задано право
деактивировать ключи (см.~\ref{DATA.Access}).

Команда требует предварительной аутентификации по PIN. 

Если ключ уже деактивирован, то должен быть возвращен статус
$\text{SW1} \parallel \text{SW2} = \hex{9000}$.

После деактивации личного ключа выполнение любых операций, 
требующих его использования, становится невозможным.

%%%%%%%%%%%%%%%%%%%%%%%%%%%%%%%%%%%%%%%%%%%%%%%%%%%%%%%%%%%%%%%%%
\subsection{Деактивация PIN}
\label{Oper.Descr.DeactivatePIN} 

Для деактивации PIN (см.~\ref{OBJ.PWD}) 
используется команда <Deactivate>, 
которая определяется согласно 
таблице~\ref{Table.Oper.DeactivatePINCmd}.

\begin{table}[hbt]
\caption{}\label{Table.Oper.DeactivatePINCmd}
\begin{tabular}{|c|p{14cm}|}
\hline
Компонент & Описание \\
\hline
\hline
INS & $\hex{04}$: деактивировать \\
\hline
P1 & $\hex{10}$: деактивировать пароль, определяемый P2\\
\hline
P2 & $\hex{03}$: PIN \\
\hline
CDF &  --- \\
\hline 
RDF & --- \\
\hline
$\text{SW1} \parallel \text{SW2}$ & 
$\hex{9000}$: PIN деактивирован успено \\
  & другое: см. таблицу~\ref{Table.Errors.General}\\
\hline
\end{tabular}
\end{table}

Команда может вызываться для приложения eSign в состояниях 
PS, AS:AT и для приложения eID в состоянии AS:AT. 
В состоянии AS:AT команда может вызываться 
только авторизованным терминалом,
в сертификате которого задано право деактивировать пароль PIN
(см.~\ref{DATA.Access}).

Команда требует предварительной аутентификации по PIN или PUK. 

Если PIN уже деактивирован, то должен быть возвращен 
статус $\text{SW1} \parallel \text{SW2} = \hex{9000}$.

После деактивации PIN выполнение любых операций, 
требующих аутентификации по паролю PIN (см. таблицу~\ref{Table.Oper.List}), 
становится невозможным. 
При этом КТ переходит в состояние IS и мастер-файл 
становится текущим выделенным файлом. 


%%%%%%%%%%%%%%%%%%%%%%%%%%%%%%%%%%%%%%%%%%%%%%%%%%%%%%%%%%%%%%%%
\subsection{Изменение PIN}
\label{Oper.Descr.ChangePIN}

Для изменения PIN используется команда <Change reference data>,
которая определяется согласно 
таблице~\ref{Table.Oper.ChangePINCmd}.

\begin{table}[hbt]
\caption{}\label{Table.Oper.ChangePINCmd}
\begin{tabular}{|c|p{14cm}|}
\hline
Компонент & 	Описание \\
\hline
\hline
INS & $\hex{24}$: изменить данные\\
\hline
P1 & $\hex{00}$: задавать старое и новое значение пароля \\
   & $\hex{01}$: задавать только новое значение пароля\\
\hline
P2 & $\hex{03}$: значение PIN \\
\hline
CDF & при $\hex{00}$ старое и новое значение PIN в формате UTF8\\
    & при $\hex{01}$ новое значение PIN в формате UTF8\\
\hline 
RDF & 	 --- \\
\hline
$\text{SW1}\parallel\text{SW2}$ & 
 $\hex{9000}$: PIN изменен успешно \\
  & другое: произошла ошибка (см. таблицу~\ref{Table.Errors.General})\\
\hline
\end{tabular}
\end{table}

Команда может вызываться для приложения eSign в состояниях 
PS, AS:AT и для приложения eID в состоянии AS:AT. 
В состоянии AS:AT команда может вызываться 
только авторизованным терминалом,
в сертификате которого задано право изменения пароля PIN
(см.~\ref{DATA.Access}).

Команда требует предварительной аутентификации по PIN. 

Выполнение команды приводит к сбрасыванию 
статуса подтверждения пароля PIN.
Для установки статуса подтверждения пароля PIN 
необходимо либо подтвердить пароль PIN (см.~\ref{Oper.Descr.VerifyPIN}), 
либо повторно выполнить аутентификацию по PIN (см.~\ref{Oper.Seq.BPACE}).


%%%%%%%%%%%%%%%%%%%%%%%%%%%%%%%%%%%%%%%%%%%%%%%%%%%%%%%%%%
\subsection{Инициализация алгоритма выработки подписи}
\label{Oper.Descr.SetDST}

Для инициализации алгоритма подписи используется команда     
<MSE: Set DST> (аббревиатура от <<Manage Security Environment: Set 
Digital Signature Template>).
Команда задает личный ключ и алгоритм
хэширования, которые будут использоваться при выработке подписи.
Команда определяется согласно таблице~\ref{Table.Oper.SetDSTCmd}.

\begin{table}[hbt]
\caption{}\label{Table.Oper.SetDSTCmd}
\begin{tabular}{|c|p{14cm}|}
\hline
Компонент & Описание \\
\hline
\hline
INS & $\hex{22}$: управление средой безопасности\\ 
\hline
$\text{P1} \parallel\text{P2}$ & $\hex{41B6}$: 
выбрать для алгоритма выработки подписи \\
\hline
CDF & 
$\der(\hex{84}, X) \parallel \der(\hex{80}, Y)$, 
где $X$ определяет идентификатор личного ключа (см. таблицу~\ref{Table.Oper.KeyRef}), 
а $Y$ определяет идентификатор используемого при выработке
подписи алгоритма хэширования 
и принимает значение $\hex{90}$ для алгоритма хэширования \texttt{belt-hash},  
$\hex{B0}$ для алгоритма хэширования \texttt{bash384} и 
$\hex{C0}$ для алгоритма хэширования \texttt{bash512} (см.~\ref{CRYPTO.StdAlg})\\
\hline 
RDF &  --- \\
\hline
$\text{SW1} \parallel \text{SW2}$ & 
$\hex{9000}$: алгоритм инициализирован успешно \\
  & $\hex{6283}$: личный ключ деактивирован \\
  & $\hex{6984}$: личный ключ уничтожен \\
%  & $\hex{6A88}$: cсылочные данные (ключ) не найдены \\
  & другое: см. таблицу~\ref{Table.Errors.General} \\
\hline
\end{tabular}
\end{table}

\if 0

\begin{table}[hbt]
\caption{Допустимые значения идентификаторов алгоритмов хэширования}
\label{Table.Oper.AlgRef}
\begin{tabular}{|c|c|c|}
\hline
Уровень стойкости ключа & Алгоритм хэширования  & Идентификатор  \\
\hline
\hline
128 & \texttt{belt-hash} & $\hex{90}$ \\
192 & \texttt{bash384} & $\hex{B0}$ \\
256 & \texttt{bash512} & $\hex{C0}$ \\
\hline
\end{tabular}
\end{table}

\fi


В компоненте CDF алгоритм хэширования \texttt{belt-hash}
может быть задан только совместно с личным ключом 
уровня стойкости $l=128$,
а алгоритмы хэширования \texttt{bash384} и \texttt{bash512}~--- 
с личными ключами уровней стойкости $l=192$ или $l=256$
соответственно.
Уровень стойкости алгоритма хэширования \texttt{bash} определяется 
неявно по уровню стойкости выбранного личного ключа.

Команда требует предварительной аутентификации по PIN. 
При вызове команды статус подтверждения 
пароля PIN должен быть установлен.
Для установки статуса подтверждения пароля PIN 
необходимо либо подтвердить пароль PIN (см.~\ref{Oper.Descr.VerifyPIN}), 
либо повторно выполнить аутентификацию по паролю PIN (см.~\ref{Oper.Seq.BPACE}).

Команда может вызываться для приложения eSign в 
состояниях PS и AS:AT. В состоянии AS:AT команда 
может вызываться только авторизованным терминалом,
в сертификате которого задано право 
выработки подписи в терминальном режиме (см.~\ref{DATA.Access}).


%%%%%%%%%%%%%%%%%%%%%%%%%%%%%%%%%%%%%%%%%%%%%%%%%%%%%%%%%%%%%%%%%%%%
\subsection{Инициализация алгоритма разбора токена ключа}
\label{Oper.Descr.SetCT}

Для инициализации алгоритма разбора токена ключа
используется команда <MSE: Set CT> 
(аббревиатура от <<Manage Security Environment: Set Confidentialy template>).
Команда задает личный ключ, который будет использоваться при разборе токена ключа.
Команда определяется согласно таблице~\ref{Table.Oper.SetCTCmd}.

\begin{table}[hbt]
\caption{}\label{Table.Oper.SetCTCmd}
\begin{tabular}{|c|p{14cm}|}
\hline
Компонент & Описание \\
\hline
\hline
INS & $\hex{22}$: управление средой безопасности\\ 
\hline
$\text{P1} \parallel\text{P2}$ & $\hex{41B8}$: 
выбрать для алгоритма разбора токена ключа
(расшифрования ключа) \\
\hline
CDF & 
$\der(\hex{84}, X)$, 
где $X$ определяет идентификатор личного ключа
(см. таблицу~\ref{Table.Oper.KeyRef})\\
\hline
RDF &  --- \\
\hline
$\text{SW1} \parallel \text{SW2}$ & 
$\hex{9000}$: алгоритм инициализирован успешно \\
  & $\hex{6283}$: личный ключ деактивирован \\
  & $\hex{6984}$: личный ключ уничтожен \\
%  & $\hex{6A88}$: cсылочные данные (ключ) не найдены \\
  & другое: см. таблицу~\ref{Table.Errors.General} \\
\hline
\end{tabular}
\end{table}

Команда может вызываться для приложения eSign в 
состояниях PS и AS:AT. В состоянии AS:AT команда 
может вызываться только авторизованным терминалом,
в сертификате которого задано право 
разбора токена ключа в терминальном режиме (см.~\ref{DATA.Access}).

Команда требует предварительной аутентификации по PIN. 


%%%%%%%%%%%%%%%%%%%%%%%%%%%%%%%%%%%%%%%%%%%%%%%%%%%%%%%%%%%%%
\subsection{Инициализация протокола BAUTH}
\label{Oper.Descr.SetBAUTH}

Для инициализации протокола BAUTH используется
команда <MSE: Set AT> 
(аббревиатура от <<Manage Security Environment: Set 
Authentication Template>>), 
которая определяется согласно 
таблице~\ref{Table.Oper.SetBAUTHCmd}.

\begin{table}[hbt]
\caption{}\label{Table.Oper.SetBAUTHCmd}
\begin{tabular}{|c|p{14cm}|}
\hline
Компонент & Описание \\
\hline
\hline
INS & $\hex{22}$: управление средой безопасности\\ 
\hline
$\text{P1} \parallel\text{P2}$ & $\hex{C1A4}$: выбрать и 
инициализировать протокол BAUTH с взаимной 
аутентификацией\\ 
 & $\hex{81A4}$: выбрать и инициализировать протокол BAUTH с 
односторонней аутентификацией\\
\hline
CDF & Объект данных 
$\der(\hex{80}, X)$, где~$X$~--- 
закодированный объектный идентификатор (без поля тега и поля 
длины) протокола (см. приложение~\ref{ASN})\\
 & Объект данных $\der(\hex{85}, X)$, 
где~$X$ определяет хэш-значение запроса аутентификации (см.~\ref{FLOW})\\
 & Объект данных~$X$, который является 
закодированным значением типа \verb|AuthAuxData| (см.~\ref{DATA.Optional}). 
Используется в команде <Verify> (см.~\ref{Oper.Descr.VerifyData}) 
для получения значений дополнительных атрибутов DocumentValidity, 
AgeVerification, RegionVerification. В~$X$ 
обязательно должны быть включены параметры атрибута DocumentValidity\\
\hline 
RDF &  --- \\
\hline
$\text{SW1} \parallel \text{SW2}$ & 
$\hex{9000}$: протокол инициализирован успешно \\
  & другое: см. таблицу~\ref{Table.Errors.General}\\
\hline
\end{tabular}
\end{table}

В компоненте CDF все объекты данных являются обязательными, 
они должны передаваться с помощью одной команды, 
т.е. использование цепочки команд <MSE: Set AT> не допускается. 
При этом порядок следования объектов данных в компоненте CDF не важен. 

\if 0
В случае если для команды <MSE: Set AT> в компоненте CDF необходимо 
передать несколько объектов данных, то они должны передаваться с помощью 
одной команды, т.е. использование цепочки команд <MSE: Set AT> не 
допускается. При этом порядок следования объектов данных в компоненте CDF 
не важен. 
\fi

Команда может вызываться на уровне мастер-файла в состоянии PS 
непосредственно после выполнения протокола BPACE (см.~\ref{Oper.Descr.GABPACE}).

%\doubt{Для использования прикладной программы eID  должен инициализироваться
%протокол BAUTH с взаимной  аутентификацией.}

При успешном выполнении протокола BAUTH дата, которая передается в команде 
для проверки срока действия КТ, может использоваться для обновления даты, 
которая хранится на КТ (см.~\ref{OBJ.Date}). 


\subsection{Инициализация протокола BPACE}
\label{Oper.Descr.SetBPACE}

Для инициализации протокола BPACE используется команда
<MSE: Set AT> (аббревиатура от <<Manage Security Environment: Set 
Authentication Template>>), которая определяется согласно 
таблице~\ref{Table.Oper.SetBPACECmd}.

\begin{table}[h]
\caption{}\label{Table.Oper.SetBPACECmd}
\begin{tabular}{|c|p{14cm}|}
\hline
Компонент & Описание \\
\hline
\hline
INS & $\hex{22}$: управление средой безопасности\\ 
\hline
$\text{P1} \parallel\text{P2}$ & $\hex{C1A4}$: выбрать и 
инициализировать протокол BPACE\\ 
\hline
CDF & Обязательный объект данных 
$\der(\hex{80}, X)$, где~$X$~--- 
закодированный объектный идентификатор (без поля тега и поля 
длины) протокола (см. приложение~\ref{ASN})\\
& Обязательный объект данных $\der(\hex{83}, X)$, 
где $X$ определяет, какой пароль будет использоваться в протоколе: 
$\hex{02}$~--- CAN,  $\hex{03}$~--- PIN, 
$\hex{04}$~--- PUK\\
 & Необязательный объект данных~$X$, который является 
закодированным значением типа \verb|CertHAT| (см.~\ref{DATA.Access}). 
Используется для установки владельцем 
КТ ограничений на доступ к данным и сервисам определенной 
прикладной программы КТ.
Для каждой прикладной программы используется свой объект.
Отсутствие объекта для определенной прикладной
программы означает отсутствие прав доступа к ней \\
 & Необязательный объект данных $\der(\hex{53}, X)$, 
где $X$ принимает значения от 0 до 255 и 
определяет количество последовательных подписей, 
которые могут быть выработаны прикладной программы eSign
в текущем сеансе без повторного подтверждения владельца КТ.
Значение 0 означает, что может быть выработано 
неограниченное количество последовательных подписей. 
Если данный объект не задан, то прикладной программой eSign 
испольуется значение по умолчанию, равное 1 \\
\hline 
RDF &  --- \\
\hline
$\text{SW1} \parallel \text{SW2}$ & 
  $\hex{9000}$: протокол инициализирован успешно,
количество возможных попыток аутентификации равно начальному значению \\
 & $\hex{63CX}$: протокол инициализирован успешно,
значение \texttt{X} содержит количество 
оставшихся попыток аутентификации, которое не равно начальному значению
(при $\texttt{X} = 1$ пароль приостановлен, а при $\texttt{X} = 0$~--- заблокирован)\\
& $\hex{6984}$: пароль деактивирован \\
 & другое: см. таблицу~\ref{Table.Errors.General} \\
\hline
\end{tabular}
\end{table}

В случае если для команды <MSE: Set AT> в компоненте CDF необходимо 
передать несколько объектов данных, то они должны передаваться с помощью 
одной команды, т.е. использование цепочки команд <MSE: Set AT> не 
допускается. При этом порядок следования объектов данных в компоненте CDF 
не важен. Например, для протокола BPACE, использующего PIN, компонент CDF, 
содержащий только обязательные объекты данных, может иметь вид: 
$\text{CDF} = 
\der(\hex{83}, \hex{03}) \parallel 
\der(\hex{80}, X)$, где~$X$~--- закодированный (без поля тега и 
поля длины) объектный идентификатор протокола BPACE (см. СТБ 34.101.66).

Закодированные значения типа \verb|CertHAT|, передаваемые в компоненте CDF, 
должны использоваться в протоколе BPACE как приветственное сообщение КП
(см.~\ref{CRYPTO.BPACE}).
В случае, если передается несколько таких значений,
то при формировании приветственного сообщения они 
должны объединяться в порядке следования в компоненте CDF.
Если же не передается ни одного значения, то в качестве 
приветственного сообщения должно использоваться пустое слово. 
 
%Для разблокировки или активации PIN (см.~\ref{OBJ.PWD})
%при инициализации BPACE в команде <MSE: Set AT> требуется использовать PUK 
%(см.~\ref{OBJ.PWD}).
%До выполнения протокола BPACE должна быть выбрана прикладная 
%программа КТ, пароль которой будет передаваться в команде <MSE: Set AT>. 
%Для этого должна быть вызвана команда <Select File> с нужным 
%идентификатором прикладной программы (AID, см. приложение~\ref{FILES}). 

Для возобновления PIN (см.~\ref{OBJ.PWD}) требуется 
использовать при инициализации BPACE пароль CAN. 
Тип пароля, который должен использоваться при 
инициализации BPACE для возможности выполнения различных 
операций, определяется в таблице~\ref{Table.Oper.List}.

При инициализации BPACE задаются разрешения 
владельца на доступ к сервисам и данным  
КТ, а также количество подписей, которые могут быть выработаны 
в текущем сеансе без повторного подтверждения владельцем КТ пароля PIN.

Команда может вызываться в любом из состояний на уровне мастер-файла.


%%%%%%%%%%%%%%%%%%%%%%%%%%%%%%%%%%%%%%%%%%%%%%%%%%%%%%%%%
\subsection{Обновление данных}
\label{Oper.Descr.Update}

Для обновления данных элементарных файлов используется
команда <Update Binary>, 
которая определяется согласно 
таблице~\ref{Table.Oper.UpdateCmd}.

\begin{table}[hbt]
\caption{}\label{Table.Oper.UpdateCmd}
\begin{tabular}{|c|p{14cm}|}
\hline
Компонент & Описание\\
\hline
\hline
INS & $\hex{D6}$: обновление бинарных данных\\
\hline
P1 & Старший байт смещения, с которого бдут перезаписываться данные 
(старший бит должен быть равен нулю) \\
\hline
P2 & Младший байт смещения, с которого будут перезаписываться данные \\
\hline
CDF & Записываемые данные \\
\hline 
RDF &  --- \\
\hline
$\text{SW1} \parallel \text{SW2}$ & 
$\hex{9000}$: данные перезаписаны успешно \\
 & другое: см. таблицу~\ref{Table.Errors.General} \\
\hline
\end{tabular}
\end{table}

Размер данных, которые нужно записать, определяется компонентом Lс команды 
(см.~\cite{APDU}).

Команда может вызываться в состояниях PS, AS:AT 
для прикладной программы eSign и в состоянии 
AS:AT для прикладной программы eID.
В состоянии AS:AT команда может вызываться только 
авторизованным терминалом, в сертификате которого
задано право записи в соответствующий элементарный 
файл (см.~\ref{DATA.Access}).

Для обновления элементарных файлов приложения eSign команда требует 
предварительной аутентификации по PIN.

Для обновления элементарных файлов приложения eID команда требует 
предварительной аутентификации по CAN или PIN.

Файл, в который записываются данные, должен быть предварительно
выбран (см. \ref{Oper.Descr.SelectEF}).
Запись может быть произведена только в те файлы, для которых 
нет ограничений по записи при текущем состоянии КТ. 
При попытке записи в файлы, доступ к которым ограничен, 
должен возвращаться статус $\text{SW1} \parallel\text{SW2} = \hex{6982}$. 


\subsection{Переключение между соединениями}
\label{Oper.Descr.SetCS}

Для переключения между защищенным соединением,
устанавливаемым между КТ и КП после выполнения 
протокола BPACE, и соединением, 
устанавливаемым между КТ и терминалом после выполнения 
протокола BAUTH, используется 
команда <MSE: Set CS> (аббревиатура от <<Manage Security Environment: Set 
Context Switch>),
которая определяется согласно 
таблице~\ref{Table.Oper.SetCSCmd}.

\begin{table}[hbt]
\caption{}\label{Table.Oper.SetCSCmd}
\begin{tabular}{|c|p{14cm}|}
\hline
Компонент & Описание \\
\hline
\hline
INS & $\hex{22}$: управление средой безопасности\\ 
\hline
$\text{P1} \parallel\text{P2}$ & $\hex{01A4}$: 
переключиться между соединениями \\
\hline
CDF & 
$\der(\hex{E1}, \der(\hex{81}, X))$, 
где $X$ определяет идентификатор
соединения и принимает значение $\hex{00}$ для
переключения на защищенное соединение,
установленное между КТ и КП,
и значение $\hex{01}$ для переключения на защищенное
соединение, установленное между КТ и терминалом\\ 
\hline 
RDF &  --- \\
\hline
$\text{SW1} \parallel \text{SW2}$ & 
$\hex{9000}$: переключение между соединениями выполнено успешно \\
 & другое: см. таблицу~\ref{Table.Errors.General} \\
\hline
\end{tabular}
\end{table}

Команда может вызываться для приложений eID и eSign
в состоянии AS.

Команда требует предварительной аутентификации по PIN или PUK.

Переключение между соединениями может понадобиться
при подтверждении (см.~\ref{Oper.Descr.VerifyPIN}), 
изменении (см.~\ref{Oper.Descr.ChangePIN})
или разблокировке (см.~\ref{Oper.Descr.UnblockPIN}) PIN.

После успешного выполнения команды выделенный файл (см.
приложение~\ref{FILES}), 
соответствующий мастер-файлу, становится текущим.

%%%%%%%%%%%%%%%%%%%%%%%%%%%%%%%%%%%%%%%%%%%%%%%%%
\subsection{Подтверждение PIN}
\label{Oper.Descr.VerifyPIN}

Для подтверждения PIN используется команда
<Verify>, которая определяется согласно 
таблице~\ref{Table.Oper.VerifyPINCmd}.

\begin{table}[hbt]
\caption{}\label{Table.Oper.VerifyPINCmd}
\begin{tabular}{|c|p{14cm}|}
\hline
Компонент & Описание \\
\hline
\hline
INS & $\hex{20}$: проверить данные\\
\hline
$\text{P1} \parallel \text{P2}$ & $\hex{0003}$: 
проверка пароля PIN\\
\hline
CDF & пароль PIN в формате UTF8 \\
\hline 
RDF &  --- \\
\hline
$\text{SW1} \parallel \text{SW2}$ & $\hex{9000}$: аутентификация успешна\\
 & $\hex{63CX}$: аутентификация неуспешна, значение \texttt{X} содержит количество 
оставшихся попыток аутентификации (при $\texttt{X} = 1$ пароль 
приостановлен, а при $\texttt{X} = 0$~--- заблокирован)\\
& $\hex{6984}$: пароль деактивирован \\
 & другое: см. таблицу~\ref{Table.Errors.General} \\
\hline
\end{tabular}
\end{table}

Команда может вызываться для приложения eSign
в состояниях PS и AS:CP.

Команда требует предварительной аутентификации по PIN.

Успешное выполнение команды устанавливает 
статус подтверждения пароля PIN.
В свою очередь, неуспешное выполнение команды 
сбрасывает данный статус.

Команду может потребоваться вызвать после сброса статуса
подтверждения пароля PIN, который происходит 
после выработки определенного количества подписей (см.~\ref{Oper.Descr.Signature}),
генерации ключевой пары (см.~\ref{Oper.Descr.GenKeys}),
изменения пароля PIN (см.~\ref{Oper.Descr.ChangePIN}),
уничтожение личного ключа (см.~\ref{Oper.Descr.Terminate}).

%%%%%%%%%%%%%%%%%%%%%%%%%%%%%%%%%%%%%%%%%%%%%%%%%%%%%%%
\subsection{Проверка дополнительного атрибута}
\label{Oper.Descr.VerifyData}

Для проверки дополнительных атрибутов 
DocumentValidity, AgeVerification, RegionVerification (см.~\ref{DATA.Optional}) 
используется команда  <Verify>, которая определяется согласно 
таблице~\ref{Table.Oper.VerifyDataCmd}.

\begin{table}[hbt]
\caption{}\label{Table.Oper.VerifyDataCmd}
\begin{tabular}{|c|p{14cm}|}
\hline
Компонент & Описание \\
\hline
\hline
INS & $\hex{20}$: проверить данные\\
\hline
$\text{P1} \parallel \text{P2}$ & $\hex{8000}$: 
проверить дополнительный атрибут\\
\hline
CDF & Закодированный идентификатор дополнительного атрибута 
(см.~\ref{DATA.Optional}). 
Может принимать одно из значений 
\verb|id-DocumentValidity|, \verb|id-AgeVerification|, \verb|id-PlaceVerification| 
(см. приложение~\ref{ASN})\\
\hline 
RDF &  --- \\
\hline
$\text{SW1} \parallel \text{SW2}$ & $\hex{9000}$: атрибут проверен успешно\\
 & $\hex{6300}$: атрибут проверен неуспешно\\
 & другое: см. таблицу~\ref{Table.Errors.General} \\
\hline
\end{tabular}
\end{table}

Команда может вызываться для приложения eID в состоянии AS:AT
авторизованным терминалом, в сертификате которого задано право
проверки соответствующего атрибута (см.~\ref{DATA.Access}).  

Команда требует предварительной аутентификации по CAN или PIN.

Установка проверяемых командой данных производится 
при инициализации протокола BAUTH (см.~\ref{Oper.Descr.SetBAUTH}).  

Для команды должен использоваться $\text{CLA}=\hex{84}$ 
(прикладной класс для защищенного соединения без использования цепочки 
команд). 




%%%%%%%%%%%%%%%%%%%%%%%%%%%%%%%%%%%%%%%%%%%%%%%%%%%%%%%%%%%%%%
\subsection{Проверка сертификата}
\label{Oper.Descr.VerifyCert}

Для импорта и проверки сертификата при аутентификации терминала 
по протоколу BAUTH используется команда 
<PSO: Verify Certificate> (аббревиатура от <<Perform Security 
Operation: Verify Certificate>>),
которая определяется согласно таблице~\ref{Table.Oper.VerifyCertCmd}.


\begin{table}[hbt]
\caption{}\label{Table.Oper.VerifyCertCmd}
\begin{tabular}{|c|p{14cm}|}
\hline
Компонент & Описание\\ 
\hline
\hline
INS & $\hex{2A}$: управление средой безопасности \\
\hline
$\text{P1} \parallel \text{P2}$ & $\hex{00BE}$: проверить 
сертификат открытого ключа \\ 
\hline
CDF  & Закодированное значение компонента \verb|certificateBody|, определяющего тело 
облегченного сертификата (см.~\ref{CERTS.Format})\\
 & Закодированное значение компонента \verb|signature|, определяющего подпись 
облегченного сертификата (см.~\ref{CERTS.Format})\\
\hline 
RDF &  --- \\
\hline
$\text{SW1} \parallel \text{SW2}$ & $\hex{9000}$: сертификат проверен успешно \\
 & другое: см. таблицу~\ref{Table.Errors.General} \\
\hline
\end{tabular}
\end{table}

В компоненте CDF команды передаются тело и подпись облегченного сертификата.
Для их передачи может использоваться цепочка из двух команд 
<PSO: Verify Certificate>, одна из которых содержит тело 
облегченного сертификата, 
а вторая~--- подпись облегченного сертификата.

Для передачи нескольких сертификатов, составляющих маршрут 
сертификации, должен использоваться 
последовательный вызов команд <PSO: Verify Certificate>. 
При этом если какой-либо сертификат передается цепочкой команд, 
то для него должна использоваться своя цепочка.

Открытый ключ, используемый при проверке сертификата после его передачи в
КТ, извлекается из сертификата, который хранится на КТ или который был 
импортирован в КТ предшествующим успешным вызовом 
команды <PSO: Verify Certificate> (см.~\ref{CERTS.Path}).

% todo?: пояснить процедуру проверки сертификата:
% - на КТ хранится сертификат/ОК доверенного УЦ; ОК помещается в буфер;
% - при проверке следующего сертификата из цепочки используется ОК из буфера,
% - при успехе проверки ОК из сертификата попадает в буфер, 
%   иначе буфер \doubt{очищается};
% - при последующей работе (выполнении шагов протокола BAUTH) используется 
%   ОК из буфера.

Команда может вызываться на уровне мастер-файла 
в состоянии PS непосредственно после 
инициализации протокола BAUTH (см.~\ref{Oper.Descr.SetBAUTH}).

Команда требует предварительной аутентификации по CAN, PIN или PUK. 
Если аутентификация выполнялась по CAN,
то в сертификате терминала должно быть установлено 
право доступа по паролю CAN  (см.~\ref{DATA.Access}).

Если сертификат является недействтельным, то должен 
возвращаться статус 
$\text{SW1} \parallel \text{SW2} = \hex{6A80}$.

%%%%%%%%%%%%%%%%%%%%%%%%%%%%%%%%%%%%%%%%%%%%%%%%%%
\subsection{Проверка статуса подтверждения PIN}
\label{Oper.Descr.VerifyAuth}

Для проверки статуса подтверждения PIN
используется команда <Verify>, 
которая определяется согласно 
таблице~\ref{Table.Oper.VerifyAuthCmd}.

\begin{table}[hbt]
\caption{}\label{Table.Oper.VerifyAuthCmd}
\begin{tabular}{|c|p{14cm}|}
\hline
Компонент & Описание \\
\hline
\hline
INS & $\hex{20}$: проверить данные\\
\hline
$\text{P1} \parallel \text{P2}$ & $\hex{0003}$: проверить статус подтверждения пароля PIN \\
\hline
CDF & --- \\
\hline 
RDF &  --- \\
\hline
$\text{SW1} \parallel \text{SW2}$ & $\hex{9000}$: 
 статус подтверждения пароля PIN установлен\\
 & другое: см. таблицу~\ref{Table.Errors.General} \\
\hline
\end{tabular}
\end{table}

Команда может вызываться для приложения eSign в состояниях PS и AS:AT.

Команда требует предварительной аутентификации по PIN.

Если статус подтверждения пароля PIN не установлен, то должен 
возвращаться статус $\text{SW1} \parallel \text{SW2} = \hex{6982}$.

%%%%%%%%%%%%%%%%%%%%%%%%%%%%%%%%%%%%%%%%%%%%%%%%%%%%%%%%%
\subsection{Разблокировка PIN}
\label{Oper.Descr.UnblockPIN}

Для разблокировки PIN используется команда
<Reset Retry Counter>,
которая определяется согласно 
таблице~\ref{Table.Oper.UnblockPINCmd}.

\begin{table}[hbt]
\caption{}\label{Table.Oper.UnblockPINCmd}
\begin{tabular}{|c|p{14cm}|}
\hline
Компонент & 	Описание \\
\hline
\hline
INS & $\hex{2С}$: разблокировать PIN\\
\hline
%P1 & $\hex{02}$: разблокировать PIN с его изменением\\
P1 & $\hex{03}$: разблокировать PIN без его изменения\\
\hline
P2 & $\hex{00}$: PIN выбран неявно\\
\hline
%CDF & при $\text{P1} = \hex{02}$ содержит новое значение PIN в формате UTF8\\
% & при $\text{P1} = \hex{03}$ не задается \\
  & --- \\
\hline 
RDF & 	 --- \\
\hline
$\text{SW1}\parallel\text{SW2}$ & $\hex{9000}$: 
разблокировка PIN была выполнена успешно\\
& другое: произошла ошибка (см. таблицу~\ref{Table.Errors.General}) \\
\hline
\end{tabular}
\end{table}

Команда может вызываться для приложения eSign в состояниях 
PS, AS:AT и для приложения eID в состоянии AS:AT. 
В состоянии AS:AT команда может вызываться 
только авторизованным терминалом,
в сертификате которого задано право разблокировать 
пароль PIN (см.~\ref{DATA.Access}).

Команда требует предварительной аутентификации по PUK. 

%%%%%%%%%%%%%%%%%%%%%%%%%%%%%%%%%%%%%%%%%%%%%%%%%%%%%%%%%%%%%%%
\subsection{Разбор токена ключа}
\label{Oper.Descr.Decipher}

Для разбора токена ключа используется 
команда <PSO: Decipher> (аббр. от 
<<Perform Security Operation: Decipher>>), 
которая определяется согласно 
таблице~\ref{Table.Oper.DecipherCmd}.

\begin{table}[hbt]
\caption{}\label{Table.Oper.DecipherCmd}
\begin{tabular}{|c|p{14cm}|}
\hline
Компонент & Описание\\ 
\hline
\hline
INS & $\hex{2A}$: управление средой безопасности \\
\hline
$\text{P1} \parallel \text{P2}$ & $\hex{8086}$: расшифровать
данные \\ 
\hline
CDF & токен ключа, длина которого не
меньше 96 октетов и не больше 112 октетов\\
\hline 
RDF &  расшифрованный ключ \\
\hline
$\text{SW1} \parallel \text{SW2}$ & $\hex{9000}$: 
токен разобран успешно\\
& другое: см. таблицу~\ref{Table.Errors.General} \\
\hline
\end{tabular}
\end{table}

При разборе токена используется алгоритм $\texttt{bign-keytransport}^{-1}$
(см.~\ref{CRYPTO.StdAlg})
со стандартными параметрами СТБ 34.101.45 (приложение Б), 
уровень которых определяется неявно по личному ключу,
выбираемому при инициализации алгоритма разбора токена 
(см.~\ref{Oper.Descr.SetCT}). 

В компоненте CDF команды должен передаваться токен ключа, 
сформированный с нулевым заголовком ключа. 

Команда может вызываться для приложения eSign в состояниях 
PS и AS:AT непосредственно после успешной инициализации 
алгоритма разбора токена (см.~\ref{Oper.Descr.SetCT}).

%Порядок, в котором должна вызываться команда, 
%определяется в~\ref{Oper.Seq.Decipher}.





%%%%%%%%%%%%%%%%%%%%%%%%%%%%%%%%%%%%%%%%%%%%
\subsection{Сброс статуса подтверждения PIN}
\label{Oper.Descr.VerifyDeauth}

Для сброса статуса подтверждения пароля PIN
используется команда <Verify>,
которая определяется согласно 
таблице~\ref{Table.Oper.VerifyDeauthCmd}.

\begin{table}[hbt]
\caption{}\label{Table.Oper.VerifyDeauthCmd}
\begin{tabular}{|c|p{14cm}|}
\hline
Компонент & Описание \\
\hline
\hline
INS & $\hex{20}$: проверить данные\\
\hline
$\text{P1} \parallel \text{P2}$ & $\hex{FF03}$: сбросить статуса подтверждения
пароля PIN  \\
\hline
CDF & ---  \\
\hline 
RDF &  --- \\
\hline
$\text{SW1} \parallel \text{SW2}$ & $\hex{9000}$ 
статус подтверждения пароля PIN сброшен успешно\\
& другое: см. таблицу~\ref{Table.Errors.General} \\
\hline
\end{tabular}
\end{table}

Команда может вызываться для приложения eSign  
в состояниях PS и AS:AT.

Команда требует предварительной аутентификации по PIN.

Успешное выполнение команды сбрасывает статус подтверждения пароля PIN,
а неуспешное~--- не изменяет его статус.

Если статус подтверждения пароля PIN уже сброшен, то должен быть возвращен статус
$\text{SW1} \parallel \text{SW2} = \hex{9000}$.

%%%%%%%%%%%%%%%%%%%%%%%%%%%%%%%%%%%%%%%%%%%%%%%%%%%%%%%%%%%%%%%%
\subsection{Чтение данных}
\label{Oper.Descr.Read}

Для чтения данных из элементарных файлов 
используется команда <Read Binary>, 
которая определяется согласно 
таблице~\ref{Table.Oper.ReadCmd}.

\begin{table}[hbt]
\caption{}\label{Table.Oper.ReadCmd}
\begin{tabular}{|c|p{14cm}|}
\hline
Компонент & Описание \\
\hline
\hline
INS & $\hex{B0}$: чтение бинарных данных \\
\hline
P1 & Старший байт смещения, с которого будут читаться данные (старший бит 
должен быть равен нулю) \\
\hline
P2 & Младший байт смещения, с которого будут читаться данные\\
\hline
CDF &  --- \\
\hline 
RDF & 	Прочитанные данные \\
\hline
$\text{SW1} \parallel\text{SW2}$ & 
$\hex{9000}$: данные прочитаны успешно \\
& другое: см. таблицу~\ref{Table.Errors.General} \\
\hline
\end{tabular}
\end{table}

Размер данных, которые нужно прочитать, определяется компонентом 
Le (см.~\cite{APDU}).

Команда может вызываться в состояниях PS, AS:AT 
для прикладной программы eSign и в состоянии 
AS:AT для прикладной программы eID.
В состоянии AS:AT команда может вызываться только 
авторизованным терминалом, в сертификате которого
задано право чтения соответствующего файла или группы данных
(см.~\ref{DATA.Access}).

Для чтения элементарных файлов приложения eSign команда требует 
предварительной аутентификации по PIN.

Для чтения элементарных файлов приложения eID команда требует 
предварительной аутентификации по CAN или PIN.

Элементарный файл, из которого читаются данные, должен быть предварительно 
выбран (см.~\ref{Oper.Descr.SelectEF}). Прочитаны могут быть только те данные, 
для которых нет ограничений по чтению при текущем состоянии КТ (см.~\ref{DATA.Access}). 
При попытке чтения данных, доступ к которым ограничен, должен возвращаться 
статус $\text{SW1} \parallel \text{SW2} = \hex{6982}$.

%%%%%%%%%%%%%%%%%%%%%%%%%%%%%%%%%%%%%%%%%%%%%%%%%%%%%%%%%%%%%%%%
\subsection{Уничтожение личного ключа}
\label{Oper.Descr.Terminate}

Для уничтожения личного ключа используется команда <Terminate>,
которая определяется согласно 
таблице~\ref{Table.Oper.TerminateCmd}.

\begin{table}[ht]
\caption{}\label{Table.Oper.TerminateCmd}
\begin{tabular}{|c|p{14cm}|}
\hline
Компонент & Описание\\
\hline
\hline
INS & $\hex{E6}$: уничтожить личный ключ \\
\hline
$\text{P1} \parallel\text{P2}$ & $\hex{2100}$:
идентификатор личного ключа задается в поле данных\\
\hline
CDF &  $\der(\hex{B6},\der(\hex{84}, X))$,
%$\der(\hex{84}, X)$,  
где $X$ определяет идентификатор личного ключа
(см. таблицу~\ref{Table.Oper.KeyRef})\\ 
\hline 
RDF & ---  \\
\hline
$\text{SW1} \parallel \text{SW2}$ & 
$\hex{9000}$: ключ уничтожен успешно\\
% & $\hex{6A88}$: ссылочные данные (ключ) не найдены\\
 & другое: см. таблицу~\ref{Table.Errors.General} \\
\hline
\end{tabular}
\end{table}

Команда может вызываться для приложения eSign в состояниях PS и AS:AT.
В состоянии AS:AT команда может вызываться только
авторизованным терминалом, в сертификате которого задано право
уничтожать ключи (см.~\ref{DATA.Access}).
В каждом из состояний могут быть уничтожены только те ключи, которые
были сгенерированы в данном состоянии (см.~\ref{Oper.Descr.GenKeys}).

Команда требует предварительной аутентификации по PIN. 
При вызове команды статус подтверждения пароля PIN 
должен быть установлен.

Выполнение команды приводит к сбрасыванию статуса подтверждения пароля PIN.
Для установки статуса подтверждения пароля PIN 
необходимо либо подтвердить пароль PIN (см.~\ref{Oper.Descr.VerifyPIN}), 
либо повторно выполнить аутентификацию по PIN (см.~\ref{Oper.Seq.BPACE}).

При попытке уничтожить ключ, который не был сгенерирован
или который был уничтожен ранее, должен возвращаться 
статус $\text{SW1} \parallel \text{SW2} = \hex{9000}$.