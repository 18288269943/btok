\newpage
\setcounter{page}{1}
\pagestyle{headings}
%\thispagestyle{empty}

\begin{center}
{\bfseries
ГОСУДАРСТВЕННЫЙ СТАНДАРТ РЕСПУБЛИКИ~БЕЛАРУСЬ
\vskip 2pt
\hrule width\textwidth

\vskip 9pt

Информационные технологии и безопасность

КРИПТОГРАФИЧЕСКИЕ ТОКЕНЫ

\vskip 9pt

Iнфармацыйныя тэхналогii i бяспека

КРЫПТАГРАФIЧНЫЯ ТОКЕНЫ
} % bfseries

\vskip 9pt

Information technology and security

Cryptographic tokens

\vskip 4pt                
\hrule width \textwidth
\end{center}

\mbox{}\hfill{\bfseries Дата введения 2018-XX-XX}

\chapter{Область применения}

Настоящий стандарт устанавливает архитектуру
криптографического токена (далее~--- КТ) и правила 
его использовании в информационных системах 
для аутентификации владельца КТ, предъявления идентификационных данных 
владельца, выработки электронной цифровой подписи, расшифрования ключей защиты 
данных.
%
В стандарте определяются объекты, которые размещаются на КТ,
сервисы, которые предоставляет КТ, протоколы работы с КТ.

Настоящий стандарт применяется при разработке средств криптографической 
защиты информации.

