\chapter{Общие положения}\label{COMMON}

\section{Назначение}

Для КТ устанавливается два режима использования: базовый и терминальный.

В базовом режиме владелец посредствам клиентской 
программы (далее~--- КП) взаимодействует с КТ, 
управляя паролями, генерируя личные ключи, подписывая данные, 
производя разшифрования ключа шифрования данных.

В терминальном режиме КТ взаимодействует с терминалом, представляя владельца.
В этом режиме КТ обеспечивает решение следующих задач: 
\begin{itemize}
\item[1)]
аутентификация владельца на доступ к ресурсам и сервисам токена; 
\item[2)]
управление объектами, размещенными в пределах криптографической границы токена;
\item[3)]
аутентификация терминала и аутентификация перед ним;
\item[4)]
установка защищенного соединения с терминалом;
\item[5)]
проверка полномочий терминала и передача ему ресурсов владельца;
\item[6)]
проверка полномочий терминала и открытие ему доступа к прикладным программам 
КТ. В частности, КТ может открывать доступ к программе eSign, реализующей
генерацию ключей, подпись запроса на получение сертификата, запись 
сертификата, подписывание данных, разшифрование ключа шифрования данных.
\end{itemize}

%\doubt{Терминал, который решает последнюю задачу, называется терминалом подписи.}

%\section{Реализация} 
\section{Вид исполнения} 

КТ может быть выполнен в виде смарт-карты, USB-токена, программы, 
выполняемой на защищенном сервере. Программные токены могут быть полезны 
для организации криптографических сервисов серверов и служб. Определенный 
в разделе~\ref{CMDS} командный интерфейс унифицирует работу с токенами 
различных типов.  

Программное обеспечение КТ должно соответствовать требованиям СТБ~34.101.27 
или аналогичных ТНПА. На программно-аппаратном КТ должны быть 
предусмотрены средства инженерной защиты от злоумышленника, завладевшего 
токеном. Эти средства должны препятствовать чтению критических данных 
(личных и секретных ключей, паролей, атрибутов владельца) и переводу КТ в 
необнаружимое небезопасное состояние.  

\section{Терминалы}

КТ взаимодействует с информационными системами посредством терминалов.

Терминалы могут быть двух типов: локальный и удаленный.

Локальный терминал взаимодействует с КТ непосредственно. КП является 
частью локального терминала. Локальный терминал должен быть 
программно-аппаратным.

Удаленный терминал взаимодействует с КТ через коммуникационные сети,
как правило, сети Интернет. Посредником при взаимодействии 
\doubt{кроме КП может быть браузер}. 

Удаленный терминал, как правило, представляет сервер или службу.
Например в СТБ 34.101.78 терминал представляет РЦ (для организации 
выпуска сертификатов) и СИ (для организации аутентификации). 

Для организации взаимодействия с КТ терминалы снабжаются сертификатами.
Сертификаты должны быть выпушены в инфраструктуре открытых ключей, 
определенной в СТБ 34.101.78.

\section{Клиентская программа}

КП организует взаимодействие КТ с владельцем и терминалом.

КП может обрабатывать критические данные
(например, пароль), а также открытые данные, 
целостность и подлинность которых определяют надежность  
взаимодействия.

КП не может самостоятельно 
обеспечить полную защиту обрабатываемых данных. 
Защита организуется также средствами среды эксплуатации, 
в том числе через настройки операционной системы.  
%
Основные задачи защиты средствами среды: 
невозможность чтения критических областей памяти
вредоносными программами; 
невозможность перехвата данных, передаваемым по каналам управления; 
защита канала <<браузер~--- приложение>>; 
невозможность подмены открытых данных.
%
Организация защиты выходит за рамки настоящего стандарта.

При использовании программно-аппаратного КТ единственными критическими данными, 
которые обрабатывает КП, является пароль доступа к КТ. Поэтому КТ 
желательно использовать в тех ситуациях, когда КП пользователя 
выполняется в потенциально агрессивной среде, например, 
на компьютере, подключенном к сети общего пользования.
%незащищенном компьютере общего пользования.

\section{Прикладные программы} 

На токене должны быть реализованы прикладные программы eID (см.~\ref{OBJ.eID}) 
и eSign (см.~\ref{OBJ.eSign}). Могут поддерживаться дополнительные программы, 
например, программа ePassport~--- электронный паспорт. 

Прикладная программа eSign может использоваться в двух режимах:
1) в локальном режиме после успешной аутентификации владельца;
2) в терминальном режиме, после успешной аутентификации терминала.

Прикладная программа eID может использоваться только
в терминальном режиме после успешной аутентификации терминала.

