\chapter{Общие положения}\label{COMMON}

\section{Назначение}

КТ используется в двух режимах: базовом и терминальном.

В базовом режиме владелец настраивает КТ и выполняет с его помощью различные
криптографические операции. Взаимодействие с КТ поддерживает КП. 
Используя КП, владелец устанавливает пароли, генерирует личные ключи, 
подписывает данные, снимает защиту с транспортируемых ключей и др.
Связанные операции и объекты КТ оформляются в виде прикладных программ.

\doubt{
КТ хранит идентификационные данные владельца и другие объекты, 
описанные в~\ref{DATA}, \ref{OBJ}. В частности, на токене размещаются 
сертификаты, описанные в~\ref{CERTS}.
%
КТ реализует криптографические алгоритмы и протоколы, заданные в~\ref{CRYPTO}.
}

Важно, что операции с личным ключом (выработка ЭЦП, снятие защиты) выполняются  
в пределах защищенной криптографической границы КТ после предъявления 
корректного пароля. Личный ключ не покидает пределов границы, при аппаратной 
защите границы секретность ключа сохраняется даже при эксплуатации КТ в 
агрессивной среде.

В терминальном режиме КТ представляет владельца при его взаимодействии с 
информационными системами. Представителями систем при этом взаимодействии 
являются терминалы, взаимодействие со стороны владельца снова поддерживает КП. 

В терминальном режиме КТ обеспечивает решение следующих задач: 
\begin{enumerate}
\item[1)]
аутентификация владельца на доступ к ресурсам и сервисам токена; 
\item[2)]
управление объектами, размещенными в пределах криптографической границы токена;
\item[3)]
аутентификация терминала и аутентификация перед ним;
\item[4)]
установка защищенного соединения с терминалом;
\item[5)]
проверка полномочий терминала и передача ему ресурсов владельца;
\item[6)]
проверка полномочий терминала и открытие ему доступа к прикладным программам КТ. 
\end{enumerate}

Получив доступ к прикладной программе КТ, терминал может выполнять \doubt{те же}
криптографические операции, что и владелец в базовом режиме. Существенным
отличием базового режима является то, несмотря на защиту личного ключа,
сохраняется угроза подмены данных, которые на нем обрабатываются (например,
подписываются). Для сравнения, в терминальном режиме КТ получает данные от
терминала, предварительно прошедшего аутентификацию перед токеном. Данные
передаются по защищенному соединению, и их невозможно раскрыть или подменить
даже при полном контроле среды эксплуатации токена. Таким образом, терминальный
режим обеспечивает большие гарантии безопасности при условии защиты среды 
эксплуатации терминала.

\doubt{todo: общая аутентификация: билета, утверждения, согласие владельца, 
ссылка на \ref{FLOW}}.

\section{Исполнение} 

КТ может быть выполнен в виде смарт-карты, USB-токена 
или программы, которая работает на защищенном сервере. 
Программные токены могут быть полезны для организации криптографических 
сервисов информационных систем. 

\doubt{
КТ устанавливает соединения с КП и терминалом.
В зависимости от состояния аутентификации по каждому соединению,
КТ может находиться в одном из трех состояний,
описанных в~\ref{STATES}.
%
Определенный в~\ref{CMDS} командный интерфейс 
унифицирует работу с токенами различных типов.
}

Программное обеспечение КТ должно соответствовать требованиям СТБ~34.101.27  или
аналогичных ТНПА. На аппаратном КТ должны быть предусмотрены средства инженерной
защиты от злоумышленника, завладевшего токеном. Эти средства должны
препятствовать чтению критических данных (личных ключей, паролей,
идентификационных данных) и переводу КТ в необнаружимое небезопасное состояние.

\section{Терминалы}

Терминалы могут быть двух типов: локальные и удаленные.

Локальный терминал взаимодействует с КТ непосредственно. КП является 
частью локального терминала. Локальный терминал должен быть аппаратным.

Удаленный терминал взаимодействует с КТ через коммуникационные сети,
как правило, сети Интернет. Посредником при взаимодействии 
кроме КП может быть еще и браузер. 

Для организации взаимодействия с КТ терминалы снабжаются сертификатами.
Сертификаты должны выпускаться в инфраструктуре открытых ключей, 
определенной в СТБ 34.101.78.

\section{Клиентская программа}

КП, организуя взаимодействие между КТ, владельцем и терминалом,
обрабатывает критические данные (пароли), а также открытые данные, 
целостность и подлинность которых определяют надежность  
взаимодействия.

КП не может самостоятельно 
обеспечить полную защиту обрабатываемых данных. 
Защита организуется также средствами среды эксплуатации, 
в том числе через настройки операционной системы.  
%
Основные задачи защиты средствами среды: 
невозможность чтения критических областей памяти вредоносными программами; 
невозможность подмены открытых данных.
защита канала <<браузер~--- КП>>.
%
Организация защиты выходит за рамки настоящего стандарта.

При работе с аппаратным КТ к уязвимым критическими данным
относятся только пароли доступа к КТ. Поэтому аппаратным КТ следует отдавать 
предпочтение в тех ситуациях, когда КП выполняется в потенциально агрессивной 
среде, например, на компьютере, подключенном к сети общего пользования.

\doubt{
С помощью КП КТ может быть вписан в сложную инфраструктуру аутентификации,
описанную в~\ref{FLOW}.
}

\section{Прикладные программы} 

На КТ должны быть установлены прикладные программы eID и eSign,
описанные в~\ref{OBJ.eID}, \ref{OBJ.eSign}. Могут устанавливаться 
дополнительные программы, например, ePassport~--- электронный 
паспорт.

Прикладная программа eID предназначена для управления идентификационными
данными владельца КТ со стороны терминала, представляющего некоторую 
информационную систему. 
%
Программа eSign поддерживает алгоритмы ЭЦП и транспорта ключа,
определенные в СТБ 34.101.45. С помощью цепочек команд eID, eSign
организуются сложные макрооперации, например, 
процесс выпуска сертификатов открытых ключей алгоритмов ЭЦП / транспорта.

Программа eSign может использоваться как в базовом режиме 
после успешной аутентификации владельца, так и в терминальном режиме 
после успешной аутентификации терминала (требуется предварительная 
аутентификация владельца). 
%
Программа eID может использоваться только в терминальном режиме
после успешной аутентификации терминала. 


\if0
Например, терминалу может быть открыт доступ к прикладной программе eSign.
Получив доступ, терминал генерирует личный ключ в пределах криптографической 
границы КТ, извлекает открытый ключ и идентификационные данные, формирует по 
ним запрос на получение сертификата, подписывает запрос на получение 
сертификата на личном ключе КТ, обращается с запросом к УЦ, дожидается 
выпуска сертификата, записывает сертификат на КТ.
\fi

