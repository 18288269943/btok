\chapter{Нормативные ссылки}

В настоящем стандарте использованы ссылки на следующие технические 
нормативные правовые акты в области технического нормирования и 
стандартизации (далее~--- ТНПА): 

СТБ 34.101.17 2012 Информационные технологии и безопасность. 
Синтаксис запроса на получение сертификата
 
СТБ 34.101.19-2012 Информационные технологии и безопасность. Форматы 
сертификатов и списков отозванных сертификатов инфраструктуры открытых 
ключей 

СТБ 34.101.27-2011 Информационные технологии и безопасность. Требования 
безопасности к программным средствам криптографической защиты информации 

СТБ 34.101.31-2011 Информационные технологии. Защита информации. 
Криптографические алгоритмы шифрования и контроля целостности 

СТБ 34.101.45-2013 Информационные технологии и безопасность. 
Алгоритмы электронной цифровой подписи и транспорта 
ключа на основе эллиптических кривых 

СТБ 34.101.47-2017 Информационные технологии и безопасность. 
Криптографические алгоритмы генерации псевдослучайных чисел 

СТБ 34.101.66-2014 Информационные технологии и безопасность. Протоколы 
формирования общего ключа на основе эллиптических кривых 

СТБ 34.101.77-2016 Информационные технологии и безопасность. 
Алгоритмы хэширования

СТБ 34.101.78-201\addendum{X} Информационные технологии и безопасность. 
Профиль инфраструктуры открытых ключей

ГОСТ 34.973-91 (ИСО 8824-87) Информационная технология. Взаимосвязь 
открытых систем. Спецификация абстрактно-синтаксической нотации версии 1 
(АСН.1) 

ГОСТ 34.974-91 (ИСО 8825-87) Информационная технология. Взаимосвязь 
открытых систем. Описание базовых правил кодирования для 
абстрактно-синтаксической нотации версии 1 (АСН.1) 

\begin{note}
Примечание -- 
При пользовании настоящим стандартом целесообразно проверить действие 
ТНПА по каталогу, составленному по состоянию на 1 января текущего года, 
и по соответствующим информационным указателям, опубликованным в текущем году. \\
Если ссылочные ТНПА заменены (изменены), то при пользовании настоящим
стандартом следует руководствоваться действующими взамен ТНПА.
Если ссылочные ТНПА отменены без замены, то положение, в котором дана ссылка
на них, применяется в части, не затрагивающей эту ссылку.
\end{note}
