\chapter{Объекты}\label{OBJ}

\section{Генератор случайных или псевдослучайных чисел}\label{OBJ.RNG}

В состав КТ должен входить физический генератор случайных чисел.
Генератор должен удовлетворять требованиям СТБ 34.101.27 или другого 
профильного ТНПА. Генератор должен использоваться для создания личных и 
секретных ключей и может использоваться для создания синхропосылок. 

Вместо генератора случайных чисел может использоваться его аналог~--- алгоритм 
генерации псевдослучайных чисел, определенный в СТБ 34.101.47 или в другом 
ТНПА. Входные данные алгоритма генерации должны включать долговременный 
секретный ключ и уникальную синхропосылку.
%
Уровень стойкости алгоритмов, в которых планируется использовать 
генерируемые ключи, должен соответствовать длине ключа алгоритма генерации.

\section{Таймер}\label{OBJ.Date}

Для определения даты на КТ может быть установлен аппаратный таймер, либо, 
если аппаратного таймера нет, текущая дата может приближенно оцениваться по 
реквизитам входящих сертификатов или по другим данным, передаваемым на КТ 
от терминала. Например, текущая дата всегда позже даты выпуска очередного 
сертификата терминала, признанного КТ действительным.

При отсутствии таймера оценка текущей даты должна устанавливаться при выпуске 
КТ в обращение равной дате выпуска. 

\section{Пароли}\label{OBJ.PWD}

КТ должен поддерживать три пароля: PIN, CAN, PUK.
Пароли используются в протоколе BPACE (см.~\ref{CRYPTO.BPACE})
и являются общими для всех прикладных программ КТ.

Пароль PIN (от Personal Identification Number)
представляет собой случайное число из 6 десятичных цифр,
известное только владельцу КТ. Используется для контроля доступа к данным и 
прикладным программам КТ.  

PIN снабжается счетчиком попыток, который первоначально равен 3. При 
неверном вводе PIN счетчик уменьшается на 1. Если счетчик достигает 
значения 1, то PIN приостанавливается и далее требуется ввести CAN. 
Ввод CAN не изменяет счетчик. При верном CAN пароль PIN возобновляется~--- его 
снова можно ввести. При неверном CAN доступ к КТ блокируется \doubt{на 
1~секунду}. Если счетчик попыток достигает значения 0, то PIN блокируется.

PIN может быть разблокирован вводом верного PUK. Однако если при 
заблокированном PIN пароль PUK вводится неверно 10 раз, то PIN блокируется 
навсегда.

При разблокировке PIN и вводе верного PIN значение счетчика попыток 
устанавливается в первоначальное. После ввода верного PIN он  
может быть изменен. 

PIN может быть деактивирован и повторно активирован.
Первоначально PIN является активированным. При деактивации
PIN доступ к операциям и данным, требующим аутентификации по PIN, невозможен.

Пароль CAN (от Card Access Number) представляет собой число из 6 десятичных 
цифр, которое не может быть вычислено на основании общей информации о КТ 
(например, серийном номере) или его владельце. Может быть напечатан на корпусе 
КТ или указан в сопроводительных документах. 

Пароль CAN не может быть заблокирован или изменен. Он используется для защиты 
от атак типа <<отказ в обслуживании>>: защита состоит в требовании ввести CAN перед 
последней проверкой PIN. Дополнительно CAN может использоваться для 
получения доступа к функциям и данным прикладной программы eID 
авторизованным терминалом, т.~е. терминалом, который был успешно 
аутентифицирован с помощью протокола BAUTH (см.~\ref{CRYPTO.BAUTH}) и в 
сертификате которого установлено соответствующее право (см.~\ref{DATA.Access}). 

Пароль PUK (от PIN Unlock Key) представляет собой случайное число из 10 
десятичных цифр, известное только владельцу КТ. PUK не может быть заблокирован 
или изменен. Используется для разблокировки PIN. Дополнительно может использоваться 
для деактивации и активации PIN.

\section{Личные ключи}\label{OBJ.Keys}

На КТ обязательно хранится личный ключ КТ. Этот ключ используется в протоколе
BAUTH для аутентификации КТ перед терминалом. Ключ устанавливается при выпуске
КТ в обращение вместе с сертификатом соотвествующего открытого. 
Процедуры выпуска КТ должны гарантировать, что ключ сохраняется только в 
пределах криптографической границы. К личному ключу КТ можно обратиться только 
косвенно, через выполнение BAUTH, он не может быть прочитан с КТ или изменен.

Дополнительно, на КТ могут хранится личные ключи владельца КТ, 
которые используются в прикладной программе eSign для выработки 
подписи и/или разбора токена ключа.
Личные ключи владельца генерируются средствами КТ при эксплуатации
КТ и никому не известны. 
Поддерживается одновременное 
использование нескольких личных ключей владельца, 
соответствующих различным уровням стойкости ключа и режимам
использования КТ.
Ключи, сгенерированные в локальном режиме, не могут использоваться 
в терминальном и наоборот.
К личному ключу владельца можно обратиться только по идентификатору, 
личный ключ владельца не может быть прочитан с КТ, 
он может быть уничтожен и изменен.

\section{Сертификаты}\label{OBJ.Certs}

При выпуске КТ в обращение на него обязательно записываются сертификат 
СИ и сертификат КТ. 
Сертификаты не связаны с конкретной 
прикладной программой, они относятся ко всему КТ. 
%$\doubt{Сертификат КТ и сертификат СТ должны быть
%облегченными (см.~\ref{CERTS.Light}).}

Дополнительно, при иcпользовании КТ на него могут быть записаны
сертификаты владельца, которые соответствуют личным ключам, используемым
в прикладной программе eSign.
%Для каждого личного ключа после его генерации в КТ может быть установлен 
%соответствующий ему сертификат. 
%Сертификаты должны соответствовать требованим СТБ 34.101.19.
При формирования запроса на выпуск сертификата могут использоваться
данные eSign, специально для этого предназначенные и предварительно
установленные при выпуске или эксплуатации КТ.

% todo: Нужен ли сертификат УЦ записывать?

Формат сертификатов и правила управления ими описаны в разделе~\ref{CERTS}.

\section{Прикладная программа eID}\label{OBJ.eID}

Прикладная программа eID обеспечивает управление атрибутами владельца КТ. 
Доступ к атрибутам осуществляется только после успешной аутентификации 
терминала. Атрибуты передаются по защищенному соединению в соответствии с 
разрешениями, установленными владельцем КТ, и правами, заданными в маршруте 
сертификации терминала (см.~\ref{CERTS.Path}).  

Прикладной программе eID назначается идентификатор \verb|id-eID|, 
определенный в приложении~\ref{ASN}. В приложении~\ref{FILES}
даны рекомендации по хранению объектов eID.

\section{Прикладная программа eSign}\label{OBJ.eSign}

Прикладная программа eSign обеспечивает генерацию личных и открытых 
ключей владельца КТ, управление сертификатами владельца, 
выработку подписи и разбор токена ключа.
%
Криптографические операции eSign основаны на алгоритмах СТБ 34.101.45.


\if 0
Прикладная программа eSign поддерживает одновременное 
использование нескольких личных ключей, 
которые могут применяться при выработке подписи и разборе токена ключа.
Личные ключи генерируются средствами КТ и никому не известны. 
К личному ключу можно обратиться только по идентификатору, 
он не может быть прочитан с КТ. 
Ключи, сгенерированные в локальном режиме, не могут использоваться 
в терминальном и наоборот.
\fi

Прикладной программе eSign назначается идентификатор \verb|id-eSign|, 
определенный в приложении~\ref{ASN}. В приложении~\ref{FILES}
даны рекомендации по хранению объектов eSign.


