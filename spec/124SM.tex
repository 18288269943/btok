\section{Управление защищенным соединением}
\label{CMDS.SM}

\subsection{Форматы сообщений}\label{CMDS.SM.Format}

Команда и ответ на команду передаются в виде двоичных слов 
$\text{cmd} = \text{CLA} \parallel \text{INS} \parallel \text{P1} \parallel 
\text{P2} \parallel \text{Lc} \parallel \text{CDF} \parallel \text{Le}$ и 
$\text{res} = \text{RDF} \parallel \text{SW1} \parallel \text{SW2}$ 
соответственно (см.~таблицу~\ref{Table.CMDS.Fmt}). 
%
Компоненты Lc, CDF, Le и RDF могут быть пустыми словами, т.~е. отсутствовать.

При передаче по защищенному соединению команда cmd преобразуются в защищенную 
команду
$\text{cmd*} = \text{CLA*} \parallel \text{INS} \parallel \text{P1} 
\parallel \text{P2} \parallel \text{Lс*} \parallel \text{CDF*} 
\parallel \text{Le*}$, а ответ~res~--- в защищенный ответ 
$\text{res*} = \text{RDF*} \parallel \text{SW1} \parallel \text{SW2}$. 
%
В защищенных командах и ответах все компоненты имеют ненулевую длину. 

Компонент CLA* защищенной команды получается из CLA установкой признака 
(бита) защиты, Lc* кодирует длину компонента CDF*, 
а Le* всегда устанавливается в $\hex{00}$. 
%
В CDF* включаются зашифрованный компонент CDF (если он непуст), 
компонент Le (если непуст) и имитовставка (обязательно). 

Аналогично компонент RDF* защищенного ответа включает зашифрованный 
компонент RDF (если он непуст) и имитовставку (обязательно). 

При формировании компонентов CDF* и RDF* включаемые в них объекты данных
кодируются с использованием отличительных правил (см.~\ref{CMDS.Intro}). 
В таблице~\ref{Table.CMDS.CDFRDF} приводятся допустимые объекты, 
указываются их длины и теги, используемые при кодировании. 

\begin{table}[h]
\caption{Объекты данных компонентов CDF* и RDF*}
\label{Table.CMDS.CDFRDF}
\begin{tabular}{|c|c|c|}
\hline
Объект & Длина (в октетах) & Тег \\
\hline
\hline
Зашифрованные данные (с индикатором защиты) & Не менее 2 & $\hex{87}$ \\
\hline
Компонент Le & 1 -- 3 & $\hex{97}$\\
%\hline
%Статусы SW1 и SW2 & 2 & $\hex{99}$ \\
\hline      
Имитовставка & 8 & $\hex{8E}$ \\
\hline
\end{tabular}
\end{table}

\subsection{Защита команды}\label{CMDS.SM.EncrCmd}

Команда cmd защищается с помощью алгоритма~\ref{CRYPTO.SM.Algs.Encr}. 
%
В качестве заголовка $I$ выступает слово 
$\text{CLA*} \parallel \text{INS} \parallel \text{P1} \parallel \text{P2}$ 
(4 октета), а в качестве критического сообщения $X$~--- компонент CDF. 
%
Слово~CLA* получается из CLA изменением одного бита:
$\text{CLA*}\leftarrow\text{CLA}\vee\hex{04}$.

Алгоритм~\ref{CRYPTO.SM.Algs.Encr} настраивается следующим образом.
\begin{enumerate}
\item
После зашифрования на шаге~1 по защищенному сообщению~$Y$
(может быть пустым) и компоненту Le (может быть пустым)
строится слово~$Z$. Для этого выполняется следующая 
последовательность шагов:
\begin{enumerate}
\item
$Z \gets\perp$;
\item
если $|Y|>0$, то $Z\gets Z\parallel\der(\hex{87},\hex{02}\parallel Y)$;
\item
если $|\text{Le}|>0$, то $Z\gets Z\parallel\der(\hex{97}, \text{Le})$.
\end{enumerate}
\item
На шаге~2 кодовое представление $\langle\langle I,Y\rangle\rangle$
определяется как $I\parallel Z$.
\item
После вычисления имитовставки~$T$ на шаге~3 строится слово
$\text{CDF*}\gets Z\parallel \der(\hex{8E}, T)$.
Затем по правилам, заданным в~\ref{CMDS.Intro}, определяется длина Lc* этого 
слова.
\item
На шаге~4 кодовое представление $\langle\langle I,Y,T\rangle\rangle$
определяется как $I\parallel\text{Le*}\parallel\text{CDF*}\parallel\hex{00}$.
Это и есть защищенная команда cmd*.
\end{enumerate}

Защита команды схематически представлена на рис.~\ref{Fig.CMDS.CmdEncr}. 
Необязательные части сообщений взяты на рисунке в квадратные скобки.

\begin{figure}[!h]
\begin{center}
\begin{tabular}{|c|c|c|c|c|c|c|}
\hline
CLA & INS & P1 & P2 & Lc & CDF & Le \\
\hline
\hline
\multicolumn{4}{|c|}{} & [длина CDF] & [сообщение $X$] & [длина ответа]\\
\hline
\end{tabular}

\vskip3pt$\Downarrow$\vskip3pt

\begin{tabular}{|c|c|c|c|c|c|c|}
\hline
CLA* & INS & P1 & P2 & Lc* & CDF* & Le*\\
\hline
\hline
\multicolumn{4}{|c|}{заголовок $I$} & длина CDF* & 
$\left[\der(\hex{87},\hex{02}\parallel Y)\parallel\right]
\left[\der(\hex{97},\text{Le})\parallel\right]$ & $\hex{00}$\\
\multicolumn{4}{|c|}{} & & 
$\phantom{[}\der(\hex{8E},T)$\hfill\mbox{} &\\
\hline
\end{tabular}
\end{center}
\caption{Защита команды}\label{Fig.CMDS.CmdEncr}
\end{figure}

\subsection{Защита ответа}\label{CMDS.SM.EncrRez}

Ответ защищается с помощью алгоритма~\ref{CRYPTO.SM.Algs.Encr}. 
В качестве заголовка~$I$ выступает слово $\text{SW1} \parallel\text{SW2}$ 
(2 октета), а в качестве критического сообщения $X$~--- компонент RDF. 

Алгоритм~\ref{CRYPTO.SM.Algs.Encr} настраивается следующим образом.
\begin{enumerate}
\item
После зашифрования на шаге~1 по защищенному сообщению~$Y$
(возможно пустому) строится слово~$Z$. Для этого выполняется следующая 
последовательность шагов:
\begin{enumerate}
\item
$Z\gets\perp$;
\item
если $|Y|>0$, то $Z\gets Z\parallel\der(\hex{87},\hex{02}\parallel Y)$.
\end{enumerate}
\item
На шаге~2 кодовое представление $\langle\langle I,Y\rangle\rangle$
определяется как $Z\parallel I$.
\item
После вычисления имитовставки~$T$ на шаге~3 строится слово
$\text{RDF*}\gets Z\parallel \der(\hex{8E}, T)$.
\item
На шаге~4 кодовое представление $\langle\langle I,Y,T\rangle\rangle$
определяется как $\text{RDF*}\parallel I$.
Это и есть защищенный ответ res*.
\end{enumerate}

Защита ответа схематически представлена на рис.~\ref{Fig.CMDS.ResEncr}. 

\begin{figure}[!h]
\begin{center}
\begin{tabular}{|c|c|c|}
\hline
RDF & SW1 & SW2 \\
\hline
\hline
[сообщение $X$] & \multicolumn{2}{|c|}{заголовок $I$} \\
\hline
\end{tabular}

\vskip3pt$\Downarrow$\vskip3pt

\begin{tabular}{|c|c|c|}
\hline
RDF* & SW1 & SW2 \\
\hline
\hline
$\left[\der(\hex{87},\hex{02}\parallel Y)\parallel\right]\ \der(\hex{8E},T)$ & 
\multicolumn{2}{|c|}{заголовок $I$} \\
\hline
\end{tabular}
\end{center}
\caption{Защита ответа}\label{Fig.CMDS.ResEncr}
\end{figure}

КТ должен защищать ответ в рамках того защищенного соединения,
по которому поступила защищенная команда (даже если это команда 
переключения между соединениями). 

\subsection{Снятие защиты}\label{CMDS.SM.Decr}

Снятие защиты с команды и ответа производится с помощью 
алгоритма~\ref{CRYPTO.SM.Algs.Decr}.  
При снятии защиты выполняются обратные к установке защиты действия: 
защищенные команда cmd* и ответ res* преобразуются в исходные команду cmd  
и ответ res.  

При снятии защиты должен проверяться формат сообщений.
В частности, должно быть проверено, что в компоненте CLA* 
защищенной команды установлен бит защиты:
$\text{CLA*}\wedge\hex{04}=\hex{04}$.

\subsection{Принудительное закрытие защищенного соединения}
\label{CMDS.SM.Stop}

КТ должен принудительно закрыть текущее защищенное соединение только в том 
случае, когда при снятии защиты с команды обнаружено, что: 
\begin{enumerate}
\item[1)] команда представлена в незащищенном виде;
\item[2)] отсутствует необходимый объект данных;
\item[3)] объект данных является некорректным.
\end{enumerate}

В первом и втором случаях КТ должен вернуть 
статус~$\text{SW1} \parallel \text{SW2} = \hex{6987}$, 
а в третьем случае~--- статус~$\text{SW1} \parallel \text{SW2} = 
\hex{6988}$.  

При принудительном закрытии защищенного соединения КТ должен уничтожить 
ключи, используемые для защиты. 
