\chapter{Соединения и состояния}\label{STATES}

\section{Защищенные соединения}\label{STATES.SM}

КТ поддерживает следующие защищенные соединения:
%
\begin{itemize}
\item[1)]
защищенное соединение между КТ и КП;
\item[2)]
защищенное соединение между КТ и терминалом.
\end{itemize}
%
Защищенные соединения обеспечивают защиту передаваемых сообщений
(см.~\ref{CRYPTO.SM}). Формат сообщений защишенного
соединения определяется в~\ref{CMDS.SM}.

Защищенное соединение между КТ и КП устанавливается 
после успешной аутентификация по протоколу BPACE (см.~\ref{CRYPTO.BPACE}).
В качестве ключа защиты данного соединения используется 
ключ, формируемый протоколом BPACE.

Защищенное соединение между КТ и терминалом устанавливается 
после успешной аутентификация по протоколу BAUTH (см.~\ref{CRYPTO.BAUTH.Vars}).
В качестве ключа защиты данного соединения используется 
ключ, формируемый протоколом BAUTH.
После установки защищенного соединения между КТ и терминалом
ранее установленное защищенное соединение между КТ и КП
не закрывается. 

КТ поддерживает переключение между защищенными соединениями.
Переключение с защищенного соединения, установленного 
между КТ и терминалом, на защищенное соединение,
установленное между КТ и КП, 
может потребоваться, например,
для подтверждения пароля PIN, при котором ввод 
пароля осуществляется через КП. 

Защищенные соединения автоматически закрываются 
после окончания работы с КТ 
(например, после отключения питания аппаратного КТ).
В некоторых случаях защищенные соединение закрываются
принудительно (см.~\ref{CMDS.SM}).

\section{Состояния криптографического токена}\label{STATES.ST}

КТ может находится в следующих состояниях:
%
\begin{itemize}
\item[1)]
IS~--- начальное состояние;
\item[2)]
PS~--- состояние аутентификации по паролю;
\item[3)]
AS~--- состояние аутентификации терминала.
\end{itemize}

В состояние IS токен переходит сразу после включения.
При определенных условиях КТ может переходить в состояние
IS из состояний PS и AS.
В данном состоянии взаимодействие с КТ осуществляется без использования 
защищенного соединения. При переходе в IS все защищенные соединения, 
которые были ранее установлены, закрываются.
В состоянии IS доступ к прикладным программам eID и eSign запрещен.

В состояние PS токен может перейти из любого состояния 
после успешной аутентификации по паролю, 
т.е. после успешного выполнения протокола BPACE. 
%
%В зависимости от используемого при аутентификации пароля 
%для состояния PS выделяются подсостояния: 
%состояния аутентификации по паролю CAN, PIN и PUK. 

Взаимодействие с КТ в состоянии PS осуществляется по 
защищенному соединению, установленному между КТ и КП. 
Если при переходе в состояние PS защищенное соединение между КТ и ПС уже было 
установлено ранее, то оно переустанавливается. 
Если же при переходе в состояние PS дополнительно было установлено 
защищенное соединение между КТ и терминалом, то оно закрывается.

В состоянии PS может использоваться только прикладная программа eSign.
В данном состоянии при выработке подписи и разборе токена ключа
могут использоваться только ключи, которые были 
сгенерированы в состоянии PS.

В состояние AS токен переходит из состояния PS 
после успешной аутентификации терминала, т.е. 
после успешного выполнения протокола BAUTH 
(с односторонней или взаимной аутентификацией). 

Взаимодействие с КТ в состоянии AS осуществляется по 
защищенному соединению, установленному между КТ и терминалом.
Дополнительно, в данном состоянии 
взаимодействие с КТ может осуществляться 
по защищенному соединению, установленному между КТ и КП.
Для первого случая используется обозначение AS:TA, 
а для второго --- AS:CP. 
После успешной аутентификации терминала
активным является защищенное соединение между КТ и терминалом.

%Защищенное соединение между КТ и терминалом
%устанавливается как активное сразу после успешного выполнения 
%протокола BAUTH.

В состоянии AS могут использоваться прикладные программы eSign и eID. 
Программы, которые могут использоваться конкретным терминалом,
определяются правами доступа (см.~\ref{DATA.Access}),
задаваемыми в сертификате терминала (см.~\ref{CERTS}).


