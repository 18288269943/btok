\chapter{Соединения и состояния}\label{STATES}

\section{Защищенные соединения}\label{STATES.SM}

КТ поддерживает защищенные соединения 
<<КП~--- КТ>> и <<терминал~--- КТ>> (см.~\ref{CRYPTO.SM}). 
Формат сообщений защишенного соединения определяется в~\ref{CMDS.SM}.

Защищенное соединение <<КП~--- КТ>> устанавливается 
после успешной аутентификация по протоколу BPACE (см.~\ref{CRYPTO.BPACE}).
В свою очередь, защищенное соединение <<терминал~--- КТ>> устанавливается 
после успешной аутентификация по протоколу BAUTH (см.~\ref{CRYPTO.BAUTH}).
%После установки защищенного соединения <<терминал~--- КТ>>
%ранее установленное защищенное соединение <<КП~--- КТ>> 
%не закрывается. 

КТ поддерживает переключение между защищенными соединениями.
Переключение с одного защищенного соединения на другое 
может потребоваться в терминальном режиме, например,
для подтверждения пароля PIN, при котором ввод пароля 
осуществляется через КП. 

Защищенные соединения автоматически закрываются 
после окончания работы с КТ (например, 
после отключения питания аппаратного КТ).
В некоторых случаях защищенные соединение 
закрываются принудительно (см.~\ref{CMDS.SM}).

\section{Состояния криптографического токена}\label{STATES.ST}

КТ может находится в следующих состояниях:
%
\begin{itemize}
\item[1)]
начальное состояние (IS);
\item[2)]
состояние после успешной аутентификации по паролю (PS);
\item[3)]
состояние после успешной аутентификации терминала (AS).
\end{itemize}
В состоянии IS взаимодействие с КТ осуществляется без использования 
защищенного соединения. 
В свою очередь, в состояниях PS и AS взаимодействие 
с КТ производится с использованием защищенных соединений.
 
В состояние IS токен переходит сразу после включения.
КТ может переходить в состояние IS из состояний PS и AS
при принудительном закрытии защищенного соединения 
<<КП~--- КТ>> (см.~\ref{CMDS.SM}).
При переходе в IS защищенное соединения <<терминал~--- КТ>>, 
если оно было установлено ранее, закрывается.
В состоянии IS доступ к прикладным программам eID и eSign запрещен.

В состояние PS токен может перейти из любого состояния 
после успешной аутентификации по паролю, 
т.е. после успешного выполнения протокола BPACE. 
Взаимодействие с КТ в состоянии PS осуществляется по 
защищенному соединению <<КП~--- КТ>>. 
Если при переходе в состояние PS защищенное соединение <<КП~--- КТ>> уже было 
установлено ранее, то оно переустанавливается. 
Если же при переходе в состояние PS дополнительно было установлено 
защищенное соединение <<терминал~--- КТ>>, то оно закрывается.

В состоянии PS может использоваться только прикладная программа eSign.
В данном состоянии при выработке подписи и разборе токена ключа
могут использоваться только ключи, которые были сгенерированы 
в состоянии PS.
В состояние AS токен переходит из состояния PS 
после успешной аутентификации терминала, т.е. 
после успешного выполнения протокола BAUTH 
(с односторонней или взаимной аутентификацией). 
Взаимодействие с КТ в состоянии AS осуществляется по 
защищенному соединению <<терминал~--- КТ>>.
Дополнительно, в данном состоянии 
взаимодействие с КТ может осуществляться 
по защищенному соединению <<КП~--- КТ>> (при переключении).
Для первого случая используется обозначение AS:TA, 
а для второго --- AS:CP. 
После успешной аутентификации терминала
активным является защищенное соединение <<терминал~--- КТ>>.

%Защищенное соединение между КТ и терминалом
%устанавливается как активное сразу после успешного выполнения 
%протокола BAUTH.

В состоянии AS могут использоваться прикладные программы eSign и eID. 
Программы, которые могут использоваться конкретным терминалом,
определяются правами доступа (см.~\ref{DATA.Access}),
задаваемыми в сертификате терминала (см.~\ref{CERTS}).

