\chapter{Сертификаты}\label{CERTS}

\section{Общие требования}

Стороны характеризуются отличительными идентификаторами $Id\in\{0,1\}^*$. 
Долговременный открытый ключ~$Q$ стороны связывается с ее идентификатором и 
распространяется в виде сертификата~$\Cert(Id, Q)\in\{0,1\}^*$. Сертификат 
представляет собой подписанные данные. Подпись вырабатывается на личном 
ключе СИ: $\Cert(Id, Q) = \Signed(\langle\langle Id, Q\rangle\rangle, d_\text{СИ})$. 
Владелец сертификата называется субъектом, а подписавшая сертификат 
сторона~--- эмитентом.  

При формировании~$\Cert(Id, Q)$ может проверяться, что предполагаемый 
владелец сертификата владеет личным ключом~$d$, которому соответствует~$Q$. 
Если такая проверка предусмотрена, то она должна состоять в проверке 
запроса на выпуск сертификата, содержащего~$Id$ и~$Q$. Для выработки ЭЦП 
должен использоваться алгоритм \texttt{bign-sign} и личный ключ~$d$, 
соответствующий~$Q$. 

Эмитентом сертификата может быть не сам СИ, а подчиненный ему (прямо или 
косвенно) удостоверящий центр. В этом случае~$\Cert(Id, Q)$ должен 
сопровождаться цепочкой подтверждающих друг друга сертификатов вплоть до 
сертификата, выпущенного непосредственно СИ. Эта цепочка называется 
маршрутом сертификации. Маршрут сертификации дополнительно может включать 
списки отозванных сертификатов, определенные в СТБ 34.101.19 (раздел 7). 
Проверка $\Cert(Id, Q)$ должна включать проверку сертификатов маршрута. 

Кроме~$Id$ и~$Q$ в сертификате указывается срок действия, идентификатор 
эмитента, права владельца сертификата, назначение открытого ключа и 
соответствующего ему личного ключа, различного рода ограничения, 
расширения и др.  

Для аутентификации терминала и КТ,
выполняемой по протоколу BAUTH (см.~\ref{CRYPTO.BAUTH}), должны использоваться 
стандартные сертификаты, определенные в СТБ 34.101.19, либо облегченные сертификаты, 
введенные в~\cite{LightCerts} и описанные в~\ref{CERTS.Light}. 
Для ключей, используемых в прикладной программе eSign,
должны выпускаться сертификаты, определенные в \doubt{СТБ 34.101.19}.

\section{Стандартный сертификат}\label{CERTS.Std}

Стандартный сертификат должен соответствовать требованиям СТБ 34.101.19, 
а также следующим дополнительным требованиям ({см. также СТБ 34.101.78}):  

\begin{enumerate}
\item
Сертификат должен быть версии 3.

\item
Сертификаты, которые используются для доступа к КТ, должны содержать 
расширение \verb|SubjectDirectoryAttributes|,
определенное в СТБ 34.101.19 (подпункт 6.2.1.8). 
В этом расширении должны быть указаны права доступа субъекта сертификата к 
прикладным программам КТ и их объектам. Права доступа задаются списком 
атрибутов субъекта. Идентификатор атрибута должен указывать на прикладную 
программу, в которой он используется, и определять содержание атрибута. 
При указании права доступа к прикладной программе eID должен использоваться атрибут с 
идентификатором \verb|id-eIdAccess| и значением, 
заданным типом \verb|EIDAccess| (см.~\ref{DATA.Access}). 
При указании права доступа к прикладной программе eSign 
должен использоваться атрибут с идентификатором \verb|id-eSignAccess| и значением, 
заданным типом \verb|ESignAccess| (см.~\ref{DATA.Access}). 
Каждый атрибут должен встречаться в расширении не более одного раза.

\item
Значения компонент строковых типов (см. тип <<строка знаков>> в ГОСТ 34.973) 
не должны содержать незначащих пробелов. 

\item
Сертификат СИ должен определять параметры открытого ключа в компоненте 
\verb|subjectPublicKeyInfo| путем указания их идентификатора \verb|bign-curveXXXv1|. 
В остальных сертификатах параметры должны либо также задаваться 
идентификатором \verb|bign-curveXXXv1|, либо задаваться ссылкой на 
параметры эмитента. 
\end{enumerate}

Для сокращения объема сертификата рекомендуется в компонентах \verb|subject| и 
\verb|issuer| использовать минимальное число атрибутов субъекта и эмитента. 
Необязательные расширения включать в сертификат не рекомендуется. 

Проверка владения личным ключом при выпуске сертификата должна проводиться 
в соответствии с СТБ 34.101.17. 

\section{Облегченный сертификат}\label{CERTS.Light}

Облегченные сертификаты (card verifiable certificate в~\cite{LightCerts}) 
имеют меньший объем и более простой формат, чем стандартные сертификаты. 
Облегченный сертификат описывается следующими типами АСН.1,
\doubt{для которых применяется тегирование} \verb|IMPLICIT|: 

\begin{verbatim}
CVCertificate ::= [APPLICATION 33] SEQUENCE {
  certificateBody ::= [APPLICATION 78] SEQUENCE {
   certProfileIdentifier			[APPLICATION 41] INTEGER { v1(0) },
   certAuthorityReference			[APPLICATION 2]  CharString,
   publicKey					[APPLICATION 73] PubKey,
   certHolderReference			[APPLICATION 32] CharString,
   certHolderAuthorizationTemplate	[APPLICATION 76] CertHAT  OPTIONAL,
   certEffectiveDate				[APPLICATION 37] CVDate,
   certExpirationDate			[APPLICATION 36] CVDate,
   certExtensions				[APPLICATION 5]  CVExt  OPTIONAL 
  },
  signature					[APPLICATION 55] OCTET STRING
}

CharString ::= OCTET STRING (SIZE (8..16))

PubKey ::= SEQUENCE {
  objIdentifier			OBJECT IDENTIFIER,
  pubKeyandParameters		ANY DEFINED BY objIdentifier
}

CVDate :: = NumericString (SIZE(6))

CVExt ::= SEQUENCE OF DiscretionaryDataTemplate
\end{verbatim}

Компоненты типа \verb|CVCertificate| имеют следующее значение:

\begin{itemize}
\item[--]
\verb|certProfileIdentifier| 
определяет текущую версию формата сертификата и 
должен быть установлен в 0 (соответствует первой версии); 

\item[--] 
\verb|certAuthorityReference| определяет эмитента сертификата;

\item[--]
\verb|publicKey| определяет открытый ключ и его параметры;

\item[--]
\verb|certHolderReference| определяет субъект сертификата;

\item[--]
\verb|certHolderAuthorizationTemplate| определяет роль и права субъекта 
сертификата. 
Значение этого компонента должно формироваться в 
соответствии с правилами, заданными в~\ref{DATA.Access}, 
и должно определять права доступа к прикладной программе eID;

\item[--]
\verb|certEffectiveDate| определяет время выпуска сертификата;

\item[--]
\verb|certExpirationDate| определяет время окончания действия сертификата;

\item[--]
\verb|certExtensions| определяет список расширений, которые содержат 
дополнительную информацию о субъекте сертификата. 
Расширения являются некритическими.
% и поэтому КТ не обязан их распознавать и использовать.
Права доступа к прикладной программе 
eSign должны определяться расширением с идентификатором
\verb|id-SignAuthExt| и значением, содержащим 
значение типа \verb|CertHAT|, закодированное 
согласно правилам, заданным в~\ref{DATA.Access}
для прав доступа к прикладной программе eSign.  

\item[--]
\verb|signature| содержит ЭЦП сертификата.
\end{itemize}

Компонент \verb|certHolderReference| должен представлять собой объединение 
следующих последовательных полей: 

\begin{itemize}
\item[--]
двухбуквенный код страны субъекта сертификата, определенный в 
соответствии с~\cite{CountryCodes} и представленный 2-мя октетами 
по правилам UTF-8, заданным в~\cite{UTF8}; 

\item[--]
уникальное мнемоническое имя. Мнемоническое имя должно однозначно 
идентифицировать субъект и состоять не более чем из девяти октетов. Оно 
выбирается и назначается субъекту сертификата эмитентом; 

\item[--]
идентификационный номер, который состоит из пяти октетов. 
Идентификационный номер может выбираться субъектом сертификата 
самостоятельно. Идентификационный номер должен являться кодовым 
представлением текстовой строки. Кодирование должно выполняться по 
правилам UTF-8. 
\end{itemize}

Компонент \verb|certAuthorityReference| должен формироваться по тем же 
правилам, что и компонент \verb|certHolderReference|. 

Компоненты типа \verb|PubKey| имеют следующее значение:

\begin{itemize}
\item[--]
\verb|objIdentifier| определяет идентификатор открытого ключа. 
Этот идентификатор должен принимать значение \texttt{bign-pubkey}, 
определенное в СТБ 34.101.45 (приложение Д); 

\item[--]
\verb|pubKeyandParameters| определяет значение открытого ключа. 
Значение должно задаваться типом \verb|PublicKey|, 
определенным в СТБ 34.101.45 (приложение Д), с тегом $\texttt{80}_{16}$. 
\end{itemize}

Тип \verb|CVDate| определяет формат даты как YYMMDD. Здесь YY означает год 20YY, 
т.~е. год задан в интервале от 2000 до 2099, MM~--- месяц, DD~--- день. 
Каждый октет кодирует цифру от 0 до 9. 

\subsection{Проверка маршрута сертификации на КТ}\label{CERTS.Path}

В протоколе BAUTH сертификат терминала может передаваться КТ вместе с маршрутом 
сертификации. При этом сертификаты маршрута должны передаваться 
последовательно, начиная с сертификата, выпущенного СИ, и заканчивая 
сертификатом терминала.  

КТ проверяет маршрут сертификации следующим образом:

\begin{enumerate}
\item
Проверяется соответствие эмитента текущего сертификата и субъекта 
предыдущего. Для первого сертификата маршрута предыдущим является 
сертификат СИ, установленный на КТ. 

\item
При сравнении идентификаторов субъекта и эмитента используется кодовое 
представление идентификаторов. 

\item
Проверяется срок действия каждого сертификата маршрута. При проверке 
используется текущая дата, которая хранится на КТ (см.~\ref{OBJ.Date}). 
Текущая дата может обновляться после успешного выполнения BAUTH, и для ее 
обновления может использоваться дата выпуска сертификата терминала. 

\item
Нераспознанные расширения сертификатов (в том числе критические 
расширения стандартных сертификатов) игнорируются. 

\item
Маршрут сертификации не должен содержать списки отозванных сертификатов, 
КТ не использует их при проверке. Поэтому ответственность за то, что в 
предоставленном маршруте сертифика-ции присутствуют отозванные 
сертификаты, ложится на сторону, предъявившую данный маршрут для проверки. 

\item
Проверяется ЭЦП каждого сертификата.
\end{enumerate}

Одновременно с проверкой маршрута сертификации могут задаваться права 
доступа терминала к прикладным программам КТ.
Для определения прав доступа к конкретной 
прикладной программе соответствующие слова прав доступа 
всех субъектов сертификатов маршрута накладываются друг на 
друга с помощью операции $\wedge$ (логическое И). 
В стандартном сертификате слова прав доступа размещаются в 
атрибуте \verb|id-eIdAccess| расширения \verb|SubjectDirectoryAttributes|, 
а в облегченном сертификате~--- в компоненте 
\verb|certHolderAuthorizationTemplate| для прав доступа к прикладной
программе eID и в расширении с идентификатором
\verb|id-SignAuthExt| для прав доступа к прикладной
программе eSign.



