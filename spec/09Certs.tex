\chapter{Облегченные сертификаты}\label{CERTS}

\section{Формат}\label{CERTS.Format}

Облегченные сертификаты (card verifiable certificates в~\cite{LightCerts}) 
имеют меньший объем и более простой формат, чем стандартные сертификаты,
определенные в СТБ 34.101.19.

Облегченный сертификат описывается следующими типами АСН.1.
В типах применяется неявное (\verb|IMPLICIT|) тегирование. 

\begin{verbatim}
CVCertificate ::= [APPLICATION 33] SEQUENCE {
  certificateBody ::= [APPLICATION 78] SEQUENCE {
   certProfileIdentifier           [APPLICATION 41] INTEGER { v1(0) },
   certAuthorityReference          [APPLICATION 2]  CharString,
   publicKey                       [APPLICATION 73] PubKey,
   certHolderReference             [APPLICATION 32] CharString,
   certHolderAuthorizationTemplate [APPLICATION 76] CertHAT OPTIONAL,
   certEffectiveDate               [APPLICATION 37] CVDate,
   certExpirationDate              [APPLICATION 36] CVDate,
   certExtensions                  [APPLICATION 5]  CVExt OPTIONAL 
  },
  signature [APPLICATION 55] OCTET STRING
}

CharString ::= OCTET STRING (SIZE (8..16))

PubKey ::= SEQUENCE {
  objIdentifier         OBJECT IDENTIFIER,
  pubKeyandParameters   PublicKey
}

CVDate :: = NumericString (SIZE(6))

CVExt ::= SEQUENCE OF DiscretionaryDataTemplate
\end{verbatim}

Компоненты типа \verb|CVCertificate| имеют следующее значение.
\begin{enumerate}
\item
Компонент \verb|certProfileIdentifier| определяет версию формата. 

\item
Компонент \verb|certAuthorityReference| идентифицирует УЦ, выпустивший 
сертификат. Представляет собой строку \str{BYCVCA0-BYXXX} для корневых УЦ либо  
строку \str{BYCVCA1-BYXXX} для подчиненных. Здесь \str{XXX}~--- 
номер УЦ (см.~\ref{OBJ.Certs}).

\item
Компонент \verb|publicKey| описывает открытый ключ.
Вложенный компонент \verb|objIdentifier| 
должен принимать значение \texttt{bign-pubkey}, 
заданное в СТБ 34.101.45 (приложение Д).
Вложенный компонент \verb|pubKeyandParameters| определяет значение открытого 
ключа. Тип \verb|PublicKey| также определен в СТБ 34.101.45. 
Длина открытого ключа определяет уровень стойкости и стандартные параметры  
эллиптической кривой этого уровня (см.~\ref{CRYPTO}).

\item
Компонент \verb|certHolderReference| идентифицирует субъекта сертификата.
Если субъектом является УЦ, то при формировании компонента повторяется логика 
\verb|certAuthorityReference|. Если же субъектом является КТ или терминал, то 
идентификационная строка должна быть получена добавлением к серийному номеру 
субъекта префикса~\str{BY}. Например, \str{BY590082394654}. Длина серийного 
номера не должна превышать 14 символов.  

\item
Компонент \verb|certHolderAuthorizationTemplate| определяет права доступа 
субъекта сертификата к прикладной программе eID. Правила формирования 
компонента определены в~\ref{DATA.Access}.
%
Компонент должен отсутствовать в сертификате КТ.
%
Отсутствие компонента в других сертификатах означает отсутствие прав 
доступа к eID. 

\item
Компонент \verb|certEffectiveDate| определяет время выпуска сертификата.
%
Время задается строкой формата \texttt{YYMMDD}, 
в которой \texttt{YY}~--- год текущего века, 
\texttt{MM}~--- месяц, \texttt{DD}~--- день.

\item
Компонент \verb|certExpirationDate| определяет время окончания действия 
сертификата.

\item
Компонент \verb|certExtensions| определяет список расширений, 
которые содержат дополнительную информацию о субъекте сертификата. 
%
Настоящий стандарт допускает только одно расширение, которое описывает 
права доступа к прикладной программе eSign. 
Расширению назначается идентификатор~\verb|id-SignAuthExt|,
определенный в приложении~\ref{ASN}. Этот идентификатор указывается
в компоненте~\texttt{objIdentifier} типа \texttt{DiscretionaryDataTemplate},
а в компоненте~\texttt{dataObjects} указывается значение типа \verb|CertHAT|.
Правила формирования значения определены в~\ref{DATA.Access}. 
%
Компонент \verb|certExtensions| должен отсутствовать в сертификате КТ.
%
Отсутствие компонента в других сертификатах означает отсутствие прав доступа к 
eSign. 

\item
Компонент \verb|signature| содержит ЭЦП сертификата, выработанную УЦ.
Уровень стойкости ключей УЦ не должен быть ниже уровня стойкости 
ключей субъекта сертификата.
\end{enumerate}

\section{Проверка}\label{CERTS.Path}

В протоколе BAUTH сертификат терминала может передаваться КТ вместе с маршрутом 
сертификации. При этом сертификаты маршрута должны передаваться 
последовательно, начиная с сертификата, выпущенного \doubt{СИ}, и заканчивая 
сертификатом терминала. 

\doubt{todo: 
В протоколе BAUTH сертификат терминала может передаваться КТ в виде 
маршрута сертификации.
Маршрут должен передаваться, если сертификат терминала выпускается 
с использованием УЦ, который подчинен УЦ, 
отвечающему за управление инфраструктурой КТ и терминалов.
При этом в маршрут должны включаться сертификаты всех подчиненных
УЦ, а также сертификат терминала.
Сертификаты маршрута должны передаваться 
последовательно, начиная с сертификата УЦ, непосредственно подчиненного УЦ, 
отвечающему за управление инфраструктурой КТ и терминалов, и заканчивая 
сертификатом терминала.
} 

КТ проверяет маршрут сертификации следующим образом:

\begin{enumerate}
\item
Проверяется соответствие эмитента текущего сертификата и субъекта 
предыдущего. Для первого сертификата маршрута предыдущим является 
сертификат СИ, установленный на КТ. 

\item
При сравнении идентификаторов субъекта и эмитента используется кодовое 
представление идентификаторов. 

\item
Проверяется срок действия каждого сертификата маршрута. При проверке 
используется текущая дата, которая хранится на КТ (см.~\ref{OBJ.Date}). 
Текущая дата может обновляться после успешного выполнения BAUTH, и для ее 
обновления может использоваться дата выпуска сертификата терминала. 

\item
Нераспознанные расширения сертификатов игнорируются. 

\item
Маршрут сертификации не должен содержать списки отозванных сертификатов, 
КТ не использует их при проверке. Поэтому ответственность за то, что в 
предоставленном маршруте сертифика-ции присутствуют отозванные 
сертификаты, ложится на сторону, предъявившую данный маршрут для проверки. 

\item
Проверяется ЭЦП каждого сертификата.
\end{enumerate}

Одновременно с проверкой маршрута сертификации могут задаваться права 
доступа терминала к прикладным программам КТ.
Для определения прав доступа к конкретной 
прикладной программе соответствующие слова прав доступа 
всех субъектов сертификатов маршрута накладываются друг на 
друга с помощью операции $\wedge$ (логическое И). 
В стандартном сертификате слова прав доступа размещаются в 
атрибуте \verb|id-eIdAccess| расширения \verb|SubjectDirectoryAttributes|, 
а в облегченном сертификате~--- в компоненте 
\verb|certHolderAuthorizationTemplate| для прав доступа к прикладной
программе eID и в расширении с идентификатором
\verb|id-SignAuthExt| для прав доступа к прикладной
программе eSign.

\doubt{todo: Учитывая, что при проверке маршрута
не используются списки отозванных сертификатов (сравни с СТБ 34.101.19) 
и не применяются онлайновые протоколы проверки статуса сертификатов 
(см., например, СТБ 34.101.26),то для сертификатов терминалов рекомендуется 
задавать срок действия, который не превышает 1 месяца.}



