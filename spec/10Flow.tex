\chapter{Выпуск билета аутентификации}\label{FLOW}

\section{Схема}\label{FLOW.Common}

В настоящем разделе определяется схема взаимодействия ПС, КП, КТ и его
владельца, СИ и его терминала для аутентификации КТ с выдачей ПС билета
аутентификации.
%
Аутентификация и выпуск билета проводятся СИ и терминалом по запросу ПС при 
помощи КП и с согласия владельца.
%
Непосредственную аутентификацию КТ выполняет терминал с помощью протокола BAUTH.
Перед этим КТ проводит парольную аутентификацию владельца (протокол BPACE), а
затем аутентификацию терминала (первая часть BAUTH).

Утверждения, которые приводятся в билете, могут быть известны СИ
(например, утверждения аутентификации) или могут храниться на КТ в виде 
идентификационных атрибутов, и тогда терминалу требуется их получить.
%
Перед передачей атрибутов пользователь дает согласие на передачу, а КТ 
проверяет, что запрос атрибутов подкреплен правами в сертификате 
терминала. Атрибуты пересылаются по защищенному соединению.

Определяемая схема может использоваться для построения систем массовой 
аутентификации. Схема может адаптироваться к нуждам систем в части 
организации пересылок между сторонами, выполнения дополнительных действий. 
При адаптации должна сохраняться базовая криптографическая логика.

\section{Объекты}

При выпуске билета аутентификации используются следующие объекты:
\begin{itemize}
\item[--]
$\Req_\text{ПС}$~--- запрос аутентификации;
\item[--]
$H_\text{КП}, H_\text{СИ}$~--- хэш-значения $\Req_\text{ПС}$, 
которые вычисляют КП и СИ соответственно;
\item[--]
$\Chart_\text{ПС}$~--- перечень утверждений, запрашиваемых ПС в~$\Req_\text{ПС}$;
\item[--]
$\Chart_\text{В}$~--- перечень идентификационных атрибутов, на предоставление 
которых дает согласие владелец КТ. Перечень $\Chart_\text{В}$ является 
сужением~$\Chart_\text{ПС}$, которое не включает не одобренные владельцем 
необязательные атрибуты.  
%
Перечень кодируется словом прав доступа к прикладной  программе eID. Это слово, 
в свою очередь, представляется объектом типа \texttt{CertHAT} 
(см.~\ref{DATA.Access});
\item[--]
$\Cert(Id_T,Q_T)$~--- сертификат терминала. Неявно сопровождается сертификатом
корневного УЦ  и может явно сопровождаться сертификатом промежуточного УЦ 
(см.~\ref{OBJ.Certs}). Маршрут сертификации определяет слово прав доступа к 
прикладной программе eID (см.~\ref{CERTS.Path}); 
\item[--]
$\Attr_\text{КТ}$~--- идентификационные атрибуты КТ (группы данных и 
дополнительные атрибуты) в соответствии с перечнем~$\Chart_\text{В}$. 
\end{itemize}

\section{Сообщения}\label{FLOW.Msgs}

Билет аутентификации выпускается за 8 шагов, определенных в~\ref{FLOW.Steps}.
В таблице~\ref{Table.FLOW.Msgs} схематически представлены сообщения,
которыми обмениваются стороны на каждом шаге.

В таблице сообщения вспомогательных протоколов BPACE и BAUTH 
маркируются индексом, в котором указывается 
название протокола. Например, $\text{M1}_\text{BAUTH}$~--- 
это сообщение M1 протокола BAUTH. 

Двойная стрелка ($\Rightarrow$) в таблице означает пересылку данных по 
защищенному соединению. Сначала это соединение <<КТ~--- КП>>, затем 
<<КТ~--- терминал>>. Хотя КП участвует в продолжении соединения и является 
посредником при передаче по нему сообщений, она не может раскрыть содержимое
сообщений.

Формат сообщений отправляемых и возвращаемых КТ детализируется в~\ref{CMDS}.

Стороны могут обмениваться дополнительными информационными сообщениями, 
например, характеристиками КТ или параметрами дополнительных идентификационных 
атрибутов. 

\begin{table}[bht]
\caption{Выпуск билета аутентификации: сообщения}\label{Table.FLOW.Msgs}
\begin{tabular}{|c|c|c|l|}
\hline
Шаг & Направление & Сообщение & Примечание\\
\hline
%
\hline
1   & $\text{ПС}\rightarrow\text{КП}$ & $\Req_\text{ПС}$ &
$\Req_\text{ПС}=\langle\langle\Chart_\text{ПС}\rangle\rangle$\\
\hline
%
2   & $\text{КП}\rightarrow\text{СИ}$ & 
$\langle\langle\Req_\text{ПС},\Chart_\text{В}\rangle\rangle$ &\\
\hline
%
3   & $\text{КП}\rightarrow\text{КТ}$ & 
$\langle\langle\text{M0}_\text{BPACE},\text{M1}_\text{BPACE}\rangle\rangle$ &
$\text{M0}_\text{BPACE}=\langle\langle\Chart_\text{В}\rangle\rangle$\\
    & $\text{КТ}\rightarrow\text{КП}$ & 
$\langle\langle\text{M2}_\text{BPACE}\rangle\rangle$ &\\
    & $\text{КП}\rightarrow\text{КТ}$ & 
$\langle\langle\text{M3}_\text{BPACE}\rangle\rangle$ &\\
    & $\text{КТ}\rightarrow\text{КП}$ & 
$\langle\langle\text{M4}_\text{BPACE}\rangle\rangle$ &\\
\hline
%
4   & $\text{СИ}\rightarrow\text{КП}$ & 
$\langle\langle\Cert(Id_\text{Т},Q_\text{Т})\rangle\rangle$ &\\
\hline
%
5   & $\text{КП}\Rightarrow\text{КТ}$ & 
$\langle\langle\text{M0}_\text{BAUTH}\rangle\rangle$ &
$\text{M0}_\text{BAUTH}=\langle\langle
H_\text{КП},\Cert(Id_\text{Т},Q_\text{Т})\rangle\rangle$\\
    & $\text{КТ}\Rightarrow\text{КП}$ & 
$\langle\langle\text{M1}_\text{BAUTH}\rangle\rangle$ &\\
\hline
%
6   & $\text{КП}\rightarrow\text{Т}$ &
$\langle\langle\text{M1}_\text{BAUTH}\rangle\rangle$ &\\
    & $\text{Т}\rightarrow\text{КП}\Rightarrow\text{КТ}$ & 
$\langle\langle\text{M2}_\text{BAUTH}\rangle\rangle$ &\\
    & $\text{КТ}\Rightarrow\text{КП}\rightarrow\text{Т}$ & 
$\langle\langle\text{M3}_\text{BAUTH}\rangle\rangle$ &\\
    & $\text{Т}\rightarrow\text{КП}\Rightarrow\text{КТ}$ & 
$\langle\langle\text{M4}_\text{BAUTH}\rangle\rangle$ &\\
\hline
%
7   & $\text{Т}\Rightarrow\text{КТ}$ &
$\langle\langle\Chart_\text{В}\rangle\rangle$ & по частям\\
    & $\text{КТ}\Rightarrow\text{Т}$ & 
$\langle\langle\Attr_\text{КТ}\rangle\rangle$ & по частям\\
\hline
%
8   & $\text{СИ}\rightarrow\text{ПС}$ &
билет аутентификации & 
$\text{билет}=\langle\langle\Attr_\text{КТ}\rangle\rangle$\\
\hline
\end{tabular}
\end{table}

\section{Шаги}\label{FLOW.Steps}

Выпуск билета аутентификации выполняется за 8 шагов, определяемых
ниже. При ошибке на любом шаге, в том числе нарушении условий проверок,
выпуск билета прекращается. Информирование сторон об ошибке и правила
обработки ошибок в настоящем стандарте не оговариваются.

{\bf Шаг 1: отправка запроса аутентификации}.
%
ПС отправляет КП запрос аутентификации~$\Req_\text{ПС}$. 
Этот запрос содержит перечень~$\Chart_\text{ПС}$.

Запрос должен быть волатильным: запросы с одинаковым содержимым должны 
отличаться. Например, запрос может включать отметку времени или уникальную
синхропосылку.

Может требоваться, чтобы ПС подписывала запрос или включать в него аттестаты,
подтверждающие права доступа к запрошенным атрибутам. В таких случаях
проверка запроса включает проверку подписи и прав доступа.

\subsection{Шаг 2: обработка запроса аутентификации}

КП проверяет запрос аутентификации, выделяет в нем перечень $\Chart_\text{ПС}$.
%
КП согласует перечень с владельцем КТ.
%
Владелец выбирает устраивающие его пункты перечня, в результате чего 
формируется перечень~$\Chart_\text{КП}$. Кроме пунктов, запрошенных ПС,
перечень может включать технические пункты, необходимые для взаимодействия с 
СИ. 

Окончательный перечень оформляется в виде слов прав доступа к прикладной 
программе \texttt{eID}. Эти права представляются 
объектом типа \texttt{CertHAT}, определенным в~\ref{DATA.Access}.

КП вычисляет хэш-значение $H_\text{КП}=\texttt{belt-hash}(\Req_\text{ПС})$ 
и отправляет запрос СИ вместе с $\Chart_\text{КП}$. СИ может быть заранее
известен, либо КП может выбрать СИ из списка с участием владельца КП. 
Спиоск подходящих СИ может быть дан в~$\Req_\texttt{ПС}$.

\subsection{Шаг 3: парольная аутентификация}

КП и КТ выполняют протокол BPACE (обмениваясь сообщениями 
$\text{M1}_\text{BPACE}$~---  $\text{M4}_\text{BPACE}$) и формируют общий ключ 
$K_0$. Стороны используют $P$ в качестве пароля и $\Chart_\text{КП}$ в качестве 
приветственного сообщения $\hello_\text{КП}$. КТ должен сохранить 
$\Chart_\text{КП}$ для дальнейшего использования.

КП и КТ создают защищенное соединение на ключе $K_0$.

\subsection{Шаг 4: обработка запроса аутентификации}

СИ получает запрос $\Req_\text{ПС}$ и перечень $\Chart_\text{КП}$.

СИ проверяет присланный запрос, выделяет в нем перечень 
$\Chart_\text{ПС}$ и проверяет его соответствие $\Chart_\text{КП}$. 
%
КП вычисляет хэш-значение $H_\text{СИ}=\texttt{belt-hash}(\Req_\text{ПС})$.

СИ выбирает терминал, права доступа которого позволяют запрашивать все 
атрибуты владельца КТ в соответствии с перечнем $\Chart_\text{КП}$

СИ отправляет КП сертификат $\Cert(Id_T,Q_T)$ выбранного терминала
(вместе с маршрутом сертификации).

\subsection{Шаг 5: открытие BAUTH}

КП получает от СИ сертификат $\Cert(Id_T,Q_T)$.

КП начинает выполнение протокола BAUTH от имени Т, отправляя КТ сертификат 
$\Cert(Id_\text{Т}, Q_\text{Т})$ как часть сообщения $\text{M0}_\text{BAUTH}$.

КТ:
\begin{enumerate}
\item
начинает выполнение протокола BAUTH, получая от КП сертификат 
$\Cert(Id_\text{Т},Q_\text{Т})$ как часть сообщения $\text{M0}_\text{BAUTH}$; 
\item
проверяет $\Cert(Id_\text{Т}, Q_\text{Т})$;
\item
выделяет в $\Cert(Id_\text{Т}, Q_\text{Т})$ права доступа Т к атрибутам владельца и 
проверяет, что Т имеет права доступа ко всем атрибутам из перечня 
$\Chart_\text{КП}$, полученного на \doubt{шаге 2}. 
\end{enumerate}

КП отправляет приветственное сообщение~$\hello_\text{Т}$ 
как часть сообщения $\text{M0}_\text{BAUTH}$. 
В качестве $\hello_\text{Т}$ используется хэш-значение $H_\text{КП}$. 

КТ:
\begin{enumerate}
\item
продолжает выполнение протокола BAUTH, получая от КП приветственное 
сообщение $\hello_\text{Т}$ как часть сообщения $\text{M0}_\text{BAUTH}$; 
\item
формирует и отправляет КП сообщение $\text{M1}_\text{BAUTH}$ с пустым 
приветственным сообщением $\hello_\text{КТ}$. 
\end{enumerate}

Обмен сообщениями между КТ и КП идет по защищенному соединению,
открытому на \doubt{шаге 4}.

КП формирует и отправляет Т сообщение $\text{M1}_\text{BAUTH}$.

\subsection{Шаг 6: завершение BAUTH}

Т получает сообщение $\text{M1}_\text{BAUTH}$.

Т и КТ:
\begin{enumerate}
\item
завершают выполнение протокола BAUTH (обмениваясь сообщениями $\text{M2}_\text{BAUTH}$ и 
$\text{M3}_\text{BAUTH}$) и формируют общий ключ $K_0$. Протокол выполняется при 
посредничестве КП, которая передает на КТ команды Т и возвращает обратно 
соответствующие ответы;  
\item
создают защищенное соединение на ключе $K_0$. При этом КТ переключается
на новое соединение с предыдущего защищенного соединения с КП;
\item
Т отправляет, а КТ получает фрагменты сообщения M3. Каждый фрагмент 
содержит атрибут из перечня $\Chart_\text{КП}$. КТ проверяет, что присланный атрибут 
содержатся в перечне, полученном на шаге 3.1; 
\item
КТ отправляет, а Т получает фрагменты сообщения M4. Каждый фрагмент 
содержит значение запрошенного атрибута из перечня $\Chart_\text{КП}$; 
\end{enumerate}

\subsection{Шаг 7: выпуск билета аутентификации}

Т:
\begin{enumerate}
\item
собирает фрагменты сообщение M4;
\item
определяет по M4 атрибуты $\Attr_\text{КТ}$;
\item
определяет атрибуты $\Attr_\text{Т}$;
\item
формирует билет аутентификации.
\end{enumerate}

\begin{note}
Примечание 1~-- 
Если $\Cert(Id_\text{Т}, Q_\text{Т})$ представляет собой маршрут 
сертификации, то КП на шаге 5.1 передает последовательные 
(от \doubt{корневого} к конечному) сертификаты маршрута. Правила передачи 
маршрута сертификации описаны в~\ref{CERTS.Path}. 
\end{note}

\begin{note}
Примечание 2~-- 
Защищенное соединение между КТ и терминалом может поддерживаться 
определенное время после аутентификации, например для обмена 
дополнительными информационными сообщениями. 
\end{note}

\begin{note}
Примечание 3~-- 
При возникновении ошибки во время выполнения протокола терминал 
должен фиксировать информацию об ошибке в компоненте status билета 
аутентификации. КП должна, по возможности, сообщать терминалу об ошибках во время 
выполнения протокола. Возможны ситуации, когда сервер не получает 
информацию об ошибке, либо не может доставить КП билет с ее описанием. 
Такими случаями являются ошибки на первом шаге протокола, разрыв 
соединения и т. п.  
\end{note}

\begin{note}
Примечание 4~-- 
Отсутствие запрашиваемого атрибута владельца КТ не является 
ошибкой, если этот атрибут не является обязательным. При отсутствии 
необязательного атрибута терминал может пометить его в билете как <<отсутствующий 
на КТ>>. 
\end{note}

