\chapter{Схема аутентификации}\label{FLOW}

\section{Общее описание}\label{FLOW.Common}

В настоящем разделе определяется схема 
взаимодействия КТ, КП, терминала и ПС для  
аутентификации КТ по запросу ПС с последующей выдачей ПС 
билета аутентификации. Определяемая схема может использоваться 
для построения систем массовой аутентификации.

Непосредственно КТ аутентифицирует терминал.
Для аутентификации используется протокол BAUTH (односторонний)
и этому протоколу предшествует протокол BPACE, \doubt{активирующий} КТ.

В случае успешной аутентификации терминал выпускает билет 
аутентификации. В этом билете должны быть приведены
утверждения о владельце. Эти утверждения могут быть известны
терминалу, и тогда сервер просто приводит их в билете. Но утверждения 
могут храниться только на КТ (в виде групп данных)
и тогда терминалу требуется их получить.

Перед передачей утверждений пользователь дает согласие 
на передачу. \doubt{Затем КТ проводит встречную аутентификацию
терминала, снова используя протокол BAUTH.}
В протоколе используется сертификат терминала. 
Этот сертификат содержит перечень прав доступа к группам
данных. КТ проверяет, что запрошенные группы подкреплены правами в 
сертификате. В случае успеха КТ передает утверждения по 
защищенному каналу. 

Протоколу предшествует запрос аутентификации от ПС. 
Этот запрос содержит перечень утверждений, которые ПС просит 
предоставить. Запрос обозначается через $\Req_\text{ПС}$,
перечень~--- через $\Chart_\text{ПС}$, собственно утверждения~---
через~$\Attr_\text{КТ}$.
%
При обработке запроса терминал проверяет, что 
запрашиваемый перечень соответствует правам доступа
в сертификате ПС.

\addendum{В описании схемы терминал обозначается буквой Т.}

\section{Входные и выходные данные}\label{FLOW.InOut}

Входными данными схемы является запрос $\Req_\text{ПС}$, 
а также следующие данные, необходимые для выполнения вспомогательных 
криптографических алгоритмов и протоколов:  
\begin{itemize}
\item[--]
личные ключи $d_\text{КТ}$, $d_\text{Т}$;
\item[--]
пароль $P\in\{0,1\}^{8*}$;
\item[--]
сертификаты $\Cert(Id_\text{КТ}, Q_\text{КТ})$, 
$\Cert(Id_\text{ПС}, Q_\text{ПС})$, 
$\Cert(Id_\text{Т}, Q_\text{Т})$.
\end{itemize}

Личный ключ $d_\text{КТ}$ используется в протоколе BAUTH, 
ключ~$d_\text{Т}$~--- в протоколе BAUTH и для создания подписанных 
данных. Дополнительно, вне рамок протокола ПС использует свой личный ключ 
$d_\text{ПС}$ для создания запроса аутентификации и разбора билета. 

Пароль~$P$ разделяют КТ и его владелец. Пароль используется в протоколе 
BPACE, где владельца представляет КП. 

Сертификаты $\Cert(Id_\text{КТ}, Q_\text{КТ})$, 
$\Cert(Id_\text{ПС}, Q_\text{ПС})$, 
$\Cert(Id_\text{Т}, Q_\text{Т})$ 
содержат открытые ключи, которые соответствуют личным ключам 
$d_\text{КТ}$, $d_\text{ПС}$, $d_\text{Т}$. 

В результате выполнения протокола КП получает подписанный терминалом билет 
аутентификации (в том числе атрибуты владельца КT и атрибуты 
аутентификации), либо сообщение об ошибке. Возврат сообщения об ошибке 
означает либо сбой при передаче сообщений протокола, либо нарушение 
целостности сообщений, либо нарушение их подлинности, либо ошибку 
аутентификации некоторой стороны протокола. 

\section{Переменные}

{\bf Общие секретные ключи}. 
В результате выполнения вспомогательных протоколов пары КT и КП, КТ и Т 
формируют общие секретные ключи $K_0\in\{0,1\}^{256}$. 
Ключи используются для создания защищенного соединения между сторонами. 

{\bf Перечень согласованных атрибутов}. 
КП формирует перечень атрибутов $\Chart_\text{КП}$, на предоставление которых ПС 
владелец КТ дает согласие. Перечень $\Chart_\text{КП}$ является подмножеством 
$\Chart_\text{ПС}$ и не включает не одобренные владельцем необязательные 
атрибуты из $\Chart_\text{ПС}$.  

{\bf Хэш-значение}. 
КП и терминал используют хэш-значение $H \in\{0,1\}^{256}$ запроса 
аутентификации. 

\section{Сообщения}

Стороны обмениваются сообщениями вспомогательных протоколов BPACE и BAUTH. 
Сообщения этих протоколов маркируются индексом, в котором указывается 
название протокола. Например, $\text{M1}_\text{BAUTH}$~--- 
это сообщение M1 протокола BAUTH. 

Стороны могут обмениваться дополнительными информационными сообщениями, 
например, характеристиками КТ, параметрами дополнительных атрибутов.  

Сообщения протоколов BPACE и BAUTH и информационные сообщения высылаются 
между следующими основными сообщениями: 

M1 ($\text{Т}\to\text{КП}$): 
$\langle\langle\Chart_\text{ПС},\Cert(Id_\text{Т},Q_\text{Т})\rangle\rangle$;

M2 ($\text{КП}\to\text{Т}$): 
$\langle\langle\Chart_\text{КП}, \text{M1}_\text{BAUTH}\rangle\rangle$;

M3 ($\text{Т}\to\text{КТ}$): 
$\langle\langle\Chart_\text{КП}\rangle\rangle$;

M4 ($\text{КТ}\to\text{Т}$): 
$\langle\langle\Attr_\text{КТ}\rangle\rangle$.

Сообщения протокола BAUTH пересылаются по защищенному соединению, 
которое устанавливается после выполнения протокола BPACE. 
Сообщения M3, M4 пересылаются по защищенному соединению, которое 
устанавливается после выполнения протокола BAUTH. Хотя КП является 
посредником при передаче сообщений, он не может раскрыть их содержимое. 

В сообщении M2 КП может дополнительно передать терминалу информацию о КТ, 
например, списки установленных на КТ приложений и корневых сертификатов.
Перечень дополнительной информации о КТ и формат ее представления в M3 
определяются за рамками настоящего протокола. 
Формат сообщений детализируется в разделе~\ref{CMDS}.

\section{Шаги}

Взаимодействие состоит в выполнении следующих шагов:
\begin{enumerate}
\item Т:
\begin{enumerate}
\item
выделяет в $\Cert(Id_\text{ПС}, Q_\text{ПС})$ права доступа ПС к атрибутам 
владельца; 
\item
выделяет в $\Req_\text{ПС}$ перечень запрошенных атрибутов 
$\Chart_\text{ПС}$ и проверяет, что права доступа ПС позволяют запрашивать 
все эти атрибуты, в том числе дополнительные; 
\item
проверяет, что собственные права доступа позволяют запрашивать все 
атрибуты владельца КТ в соответствии с перечнем $\Chart_\text{ПС}$; 
\item
$H\gets \texttt{belt-hash}(\Req_\text{ПС})$;
\item
отправляет КП сообщение M1.
\end{enumerate}
\item КП:
\begin{enumerate}
\item
получает от Т сообщение M1;
\item
выделяет в M1 сертификат $\Cert(Id_\text{Т}, Q_\text{Т})$;
\item
согласует с владельцем КТ перечень предоставляемых атрибутов $\Chart_\text{КП}$; 
\item
$H\gets\texttt{belt-hash}(\Req_\text{ПС})$.
\end{enumerate}
\item
КП и КТ:
\begin{enumerate}
\item
выполняют протокол BPACE (обмениваясь сообщениями $\text{M1}_\text{BPACE}$~--- 
$\text{M4}_\text{BPACE}$) и формируют общий ключ $K_0$. Стороны используют 
$P$ в качестве пароля и $\Chart_\text{КП}$ в качестве приветственного 
сообщения $\hello_\text{КП}$. КТ должен сохранить $\Chart_\text{КП}$ 
для дальнейшего использования; 
\item
создают защищенное соединение на ключе $K_0$;
\item
\optional{обмениваются информационными сообщениями}.
\end{enumerate}
\item КП:
\begin{enumerate}
\item
начинает выполнение протокола BAUTH от имени Т, отправляя КТ сертификат 
$\Cert(Id_\text{Т}, Q_\text{Т})$ как часть сообщения $\text{M0}_\text{BAUTH}$; 
\end{enumerate}
\item
КТ:
\begin{enumerate}
\item
начинает выполнение протокола BAUTH, получая от КП сертификат 
$\Cert(Id_\text{Т},Q_\text{Т})$ как часть сообщения $\text{M0}_\text{BAUTH}$; 
\item
проверяет $\Cert(Id_\text{Т}, Q_\text{Т})$;
\item
выделяет в $\Cert(Id_\text{Т}, Q_\text{Т})$ права доступа Т к атрибутам владельца и 
проверяет, что Т имеет права доступа ко всем атрибутам из перечня 
$\Chart_\text{КП}$, полученного на шаге 3.1. 
\end{enumerate}
\item[[5]
КП и КТ:
\begin{enumerate}
\item
обмениваются информационными сообщениями].
\end{enumerate}
\item
КП:
\begin{enumerate}
\item
продолжает выполнение протокола BAUTH от имени Т, отправляя КТ 
приветственное сообщение~$\hello_\text{Т}$ 
как часть сообщения $\text{M0}_\text{BAUTH}$. 
В качестве $\hello_\text{Т}$ используется хэш-значение $H$. 
\end{enumerate}
\item КТ:
\begin{enumerate}
\item
продолжает выполнение протокола BAUTH, получая от КП приветственное 
сообщение $\hello_\text{Т}$ как часть сообщения $\text{M0}_\text{BAUTH}$; 
\item
формирует и отправляет КП сообщение $\text{M1}_\text{BAUTH}$ с пустым 
приветственным сообщением $\hello_\text{КТ}$. 
\end{enumerate}
\item КП:
\begin{enumerate}
\item
формирует и отправляет Т сообщение M2.
\end{enumerate}
\item Т:
\begin{enumerate}
\item
получает сообщение M2;
\item
выделяет в M2 перечень $\Chart_\text{КП}$ и проверяет, что он либо совпадает с 
перечнем $\Chart_\text{ПС}$, либо не содержит некоторые необязательные атрибуты из 
$\Chart_\text{ПС}$; 
\item
выделяет в M2 сообщение $\text{M1}_\text{BAUTH}$.
\end{enumerate}
\item
Т и КТ:
\begin{enumerate}
\item
завершают выполнение протокола BAUTH (обмениваясь сообщениями $\text{M2}_\text{BAUTH}$ и 
$\text{M3}_\text{BAUTH}$) и формируют общий ключ $K_0$. Протокол выполняется при 
посредничестве КП, которая передает на КТ команды Т и возвращает обратно 
соответствующие ответы;  
\item
создают защищенное соединение на ключе $K_0$. При этом КТ закрывает 
предыдущее защищенное соединение с КП; 
\item
Т отправляет, а КТ получает фрагменты сообщения M3. Каждый фрагмент 
содержит атрибут из перечня $\Chart_\text{КП}$. КТ проверяет, что присланный атрибут 
содержатся в перечне, полученном на шаге 3.1; 
\item
КТ отправляет, а Т получает фрагменты сообщения M4. Каждый фрагмент 
содержит значение запрошенного атрибута из перечня $\Chart_\text{КП}$; 
\item
\optional{обмениваются информационными сообщениями}.
\end{enumerate}
\item Т:
\begin{enumerate}
\item
собирает фрагменты сообщение M4;
\item
определяет по M4 атрибуты $\Attr_\text{КТ}$;
\item
определяет атрибуты $\Attr_\text{Т}$;
\item
формирует билет аутентификации.
\end{enumerate}
\end{enumerate}

\begin{note}
Примечание 1~-- 
Если $\Cert(Id_\text{Т}, Q_\text{Т})$ представляет собой маршрут 
сертификации, то КП на шаге 5.1 передает последовательные 
(от \doubt{корневого} к конечному) сертификаты маршрута. Правила передачи маршрута сертификации 
описаны в~\ref{CERTS.Path}. 
\end{note}

\begin{note}
Примечание 2~-- 
Защищенное соединение между КТ и терминалом может поддерживаться 
определенное время после аутентификации, например для обмена 
дополнительными информационными сообщениями. 
\end{note}

\begin{note}
Примечание 3~-- 
При возникновении ошибки во время выполнения протокола терминал 
должен фиксировать информацию об ошибке в компоненте status билета 
аутентификации. КП должна, по возможности, сообщать терминалу об ошибках во время 
выполнения протокола. Возможны ситуации, когда сервер не получает 
информацию об ошибке, либо не может доставить КП билет с ее описанием. 
Такими случаями являются ошибки на первом шаге протокола, разрыв 
соединения и т. п.  
\end{note}

\begin{note}
Примечание 4~-- 
Отсутствие запрашиваемого атрибута владельца КТ не является 
ошибкой, если этот атрибут не является обязательным. При отсутствии 
необязательного атрибута терминал может пометить его в билете как <<отсутствующий 
на КТ>>. 
\end{note}

