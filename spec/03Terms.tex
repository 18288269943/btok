\chapter{Термины и определения и сокращения}

В настоящем стандарте применяют термины, устaновленные 
в СТБ 34.101.19, СТБ 34.101.27, СТБ 34.101.31, СТБ 34.101.45, 
СТБ 34.101.66, СТБ 34.101.78, а также следующие термины с 
соответствующими определениями: 

{\bf \thedefctr~билет}:
Информационный объект, выпускаемый по итогам успешной аутентификации
или после предъявления другого билета. Подтверждает определенные 
события (аутентификация), факты (идентификационные данные) 
или состояние (сеанс). 

{\bf \thedefctr~билет аутентификации}:
Билет, который содержит утверждения аутентификации и, возможно, другие 
утверждения. 

{\bf \thedefctr~владелец}:
Пользователь, владеющий криптографическим токеном.

{\bf \thedefctr~клиентская программа}:
Программа, которая организует взаимодействие между терминалом, 
пользователем и его криптографическим токеном.

{\bf \thedefctr~команда криптографического токена}:
Сообщение, которое посылается 
криптографическому токену и которое содержит описание определенного 
сервиса, предоставляемого токеном, и входные данные сервиса. Команда 
сопровождается ответом, который содержит выходные данные сервиса. 


{\bf \thedefctr~криптографический токен}: \doubt{...}

{\bf \thedefctr~прикладная программа криптографического токена}:
Имеющий определенное 
назначение набор объектов и сервисов криптографического токена. 

{\bf \thedefctr~утверждение}:
Характеристика (атрибут) стороны или события, 
в том числе данные о пользователе (идентификационные данные) 
или данные о прохождении аутентификации (утверждения аутентификации).


\if 0
{\bf \thedefctr~центр идентификации, ЦИ}:
Информационная система, которая управляет открытыми ключами
криптографических токенов и возможно организует их выпуск.
%
В качестве центра идентификации может выступать РУЦ или специально  
выделенный ПУЦ инфраструктуры открытых ключей, определенной в СТБ 
34.101.78.
\fi

%В настоящем стандарте используются сокращения СТБ 34.101.78.
В настоящем стандарте применяются следующие сокращения:

АСН.1 -- абстрактно-синтаксическая нотация версии 1;

КП -- клиентская программа;

КТ -- криптографический токен;

ПС -- прикладная система;

СИ -- служба идентификации;

ТНПА -- технические нормативные правовые акты;

УЦ -- удостоверяющий центр.

%todo: нужно  ли вводить сокращения PIN, PUC, CAN, IS, ... ?


