\chapter{Термины и определения}\label{TERMS}

В настоящем стандарте применяют термины, устaновленные 
в СТБ 34.101.19, СТБ 34.101.27, СТБ 34.101.31, СТБ 34.101.45, 
СТБ 34.101.66, а также следующие термины с соответствующими определениями:

{\bf \thedefctr~аутентификация}:
Проверка подлинности стороны.

{\bf \thedefctr~билет}:
Информационный объект, который выпускается по итогам успешной аутентификации
или после предъявления другого билета и подтверждает определенные 
события (аутентификация), факты (идентификационные данные) 
или состояние (сеанс). 

{\bf \thedefctr~билет аутентификации}:
Билет, который содержит утверждения аутентификации и, возможно, другие 
утверждения. 

{\bf \thedefctr~владелец}:
Пользователь, владеющий криптографическим токеном.

{\bf \thedefctr~защищенное соединение}:
Соединение, которое обеспечивает конфиденциальность, 
контроль целостности и возможно подлинности сообщений. 

\begin{note}
Примечание~--- Контроль подлинности сообщений от стороны~$A$ к стороне~$B$ 
обеспечивается после того, как~$B$ провела аутентификацию~$A$.
\end{note}

{\bf \thedefctr~идентификационный атрибут}:
Компонент идентификационных данных. 

{\bf \thedefctr~идентификационные данные}:
Данные, которые однозначно характеризуют определенную 
сторону в определенном контексте. 

\begin{note}
Примечание~--- В разных контекстах могут использоваться 
различные идентификационные данные одной и той же стороны.
\end{note}

{\bf \thedefctr~клиентская программа}:
Программа, которая организует взаимодействие между терминалом, 
пользователем и его криптографическим токеном.

{\bf \thedefctr~команда (криптографического токена)}:
Сообщение, которое посылается 
криптографическому токену и которое содержит описание определенного 
сервиса, предоставляемого токеном, и входные данные сервиса. 
%
\begin{note}
Примечание~--- Команда сопровождается ответом, который содержит выходные данные 
сервиса.
\end{note}

{\bf\thedefctr~криптографический токен}: 
Средство криптографической защиты информации, имеющее конкретного 
владельца и выступающее от его лица при взаимодействии с другими 
сторонами. 
%
\begin{note}
Примечание~--- В настоящем стандарте криптографический токен хранит один 
или несколько личных ключей владельца и реализует операции с ними.  
%
На токене могут размещаться идентификационные данные владельца. 
\end{note}

{\bf \thedefctr~прикладная программа (криптографического токена)}:
Имеющий определенное 
назначение набор объектов и сервисов криптографического токена. 

{\bf\thedefctr~терминал}: 
Сторона, которая взаимодействует с криптографическим токеном по 
защищенному соединению после аутентификации перед ним и, возможно, 
встречной аутентификации токена.

{\bf \thedefctr~утверждение}:
Характеристика (атрибут) стороны или события, 
в том числе данные о пользователе (идентификационные данные) 
или данные о прохождении аутентификации (утверждения аутентификации).



