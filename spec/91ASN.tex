\begin{appendix}{А}{рекомендуемое}{Модуль АСН.1}
\label{ASN}

\mbox{}

\hiddensection{Идентификаторы}

Идентификаторы АСН.1 назначаются следующим объектам, введенным в настоящем стандарте:
\begin{center}
\begin{tabular}{p{4.5cm}p{11.5cm}}
\verb|id-DocumentValidity| &
дополнительный атрибут DocumentValidity (см.~\ref{DATA.Optional});\\
%
\verb|id-AgeVerification| &
дополнительный атрибут AgeVerification (см.~\ref{DATA.Optional});\\
%
\verb|id-PlaceVerification| &
дополнительный атрибут PlaceVerification (см.~\ref{DATA.Optional});\\
%
\verb|id-eID| &
прикладная программа eID (см.~\ref{OBJ.eID});\\
%
\verb|id-eSign| &
прикладная программа eSign (см.~\ref{OBJ.eSign});\\
%
\verb|id-eIdAccess| &
права доступа к прикладной программе eID 
(см.~\ref{DATA.Access});\\
%
\verb|id-eSignAccess| &
права доступа к прикладной программе eSign 
(см.~\ref{DATA.Access});\\
%
\verb|id-SignAuthExt| &
расширение с правами доступа к прикладной программе eSign 
(см.~\ref{CERTS.Format});\\
%
\verb|btok-bauth| &
полный протокол BAUTH 
(см.~\ref{CRYPTO.BAUTH});\\
%
\verb|btok-bauth1| &
протокол BAUTH с односторонней аутентификацией
(см.~\ref{CRYPTO.BAUTH}).\\
\end{tabular}
\end{center}

Долговременному открытому ключу, который используется в протоколе 
BAUTH, присваивается идентификатор \verb|bign-pubkey|, определенный в СТБ 
34.101.45 (приложение Д).

\hiddensection{Модуль}

\verbatiminput{btok-module-v1.asn}

\end{appendix}
