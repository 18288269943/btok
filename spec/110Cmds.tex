\chapter{Соединения и состояния}
\label{STATES}

\section{Защищенные соединения}
\label{STATES.SM}

КТ поддерживает следующие защищенные соединения:
%
\begin{itemize}
\item[1)]
защищенное соединение между КТ и КП;
\item[2)]
защищенное соединение между КТ и терминалом.
\end{itemize}
%
Защищенные соединения обеспечивают защиту передаваемых сообщений
(см.~\ref{CRYPTO.SM}). Формат сообщений защишенного
соединения определяется в \ref{CMDS.SM}.

Защищенное соединение между КТ и КП устанавливается 
после успешной аутентификация по протоколу BPACE (см.~\ref{CRYPTO.BPACE}).
В качестве ключа защиты данного соединения используется 
ключ, формируемый протоколом BPACE.

Защищенное соединение между КТ и терминалом устанавливается 
после успешной аутентификация по протоколу BAUTH (см.~\ref{CRYPTO.BAUTH.Vars}).
В качестве ключа защиты данного соединения используется 
ключ, формируемый протоколом BAUTH.
После установки защищенного соединения между КТ и терминалом
ранее установленное защищенное соединение между КТ и КП
не закрывается. 

КТ поддерживает переключение между защищенными соединениями.
Переключение с защищенного соединения, установленного 
между КТ и терминалом, на защищенное соединение,
установленное между КТ и КП, 
может потребоваться, например,
для подтверждения пароля PIN, при котором ввод 
пароля осуществляется через КП. 

Защищенные соединения автоматически закрываются 
после окончания работы с КТ 
(например, после отключения питания аппаратного КТ).
В некоторых случаях защищенные соединение закрываются
принудительно (см.~\ref{CMDS.SM}).

\section{Состояния криптографического токена}
\label{STATES.ST}

КТ может находится в следующих состояниях:
%
\begin{itemize}
\item[1)]
IS~--- начальное состояние;
\item[2)]
PS~--- состояние аутентификации по паролю;
\item[3)]
AS~--- состояние аутентификации терминала.
\end{itemize}

В состояние IS токен переходит сразу после включения.
При определенных условиях КТ может переходить в состояние
IS из состояний PS и AS.
В данном состоянии взаимодействие с КТ осуществляется без использования 
защищенного соединения. При переходе в IS все защищенные соединения, 
которые были ранее установлены, закрываются.
В состоянии IS доступ к прикладным программам eID и eSign запрещен.

В состояние PS токен может перейти из любого состояния 
после успешной аутентификации по паролю, 
т.е. после успешного выполнения протокола BPACE. 
%
%В зависимости от используемого при аутентификации пароля 
%для состояния PS выделяются подсостояния: 
%состояния аутентификации по паролю CAN, PIN и PUK. 

Взаимодействие с КТ в состоянии PS  осуществляется по 
защищенному соединению, установленному между КТ и КП. 
Если при переходе в состояние PS защищенное соединение между КТ и ПС уже было 
установлено ранее, то оно переустанавливается. 
Если же при переходе в состояние PS дополнительно было установлено 
защищенное соединение между КТ и терминалом, то оно закрывается.

В состоянии PS может использоваться только прикладная программа eSign.
В данном состоянии при выработке подписи и разборе токена ключа
могут использоваться только ключи, которые были 
сгенерированы в состоянии PS.

В состояние AS токен переходит из состояния PS 
после успешной аутентификации терминала, т.е. 
после успешного выполнения протокола BAUTH 
(с односторонней или взаимной аутентификацией). 

Взаимодействие с КТ в состоянии AS осуществляется по 
защищенному соединению, установленному между КТ и терминалом.
Дополнительно, в данном состоянии 
взаимодействие с КТ может осуществляться 
по защищенному соединению, установленному между КТ и КП.
Для первого случая будем использовать обозначение AS:TA, 
а для второго --- AS:CP. 
После аутентификации активным является защищенное 
соединение между КТ и терминалом.

%Защищенное соединение между КТ и терминалом
%устанавливается как активное сразу после успешного выполнения 
%протокола BAUTH.

В состоянии AS могут использоваться прикладные программы eSign и eID. 
Приложения, которые могут использоваться конкретным терминалом,
определяются правами доступа терминала (см. \ref{DATA.Access}),
которые задаются в его сертификате (см.~\ref{CERTS}).
 

\chapter{Командный интерфейс}
\label{CMDS}

\section{Общие сведения}\label{CMDS.Intro}

КП и терминал взаимодействуют с КТ посредством передачи команд APDU 
(аббревиатура от Application Protocol Data Unit), определенных в~\cite{APDU}. 
При получении команды КТ обрабатывает ее и возвращает ответ на команду. При 
этом до подачи на КТ новой команды должен быть получен ответ на 
предыдущую. 

Команды и ответы на них содержат обязательные компоненты и дополнительно 
могут содержать необязательные компоненты.  

Для команды обязательными компонентами являются: CLA~--- класс команды, 
INS~--- инструкция команды, P1 и P2~--- параметры команды. В свою очередь для 
ответа обязательными являются статусы обработки команды SW1 и SW2.  

Необязательным для команды является компонент CDF, который содержит данные 
команды. При этом если в команде присутствует компонент CDF, то команда 
должна также содержать необязательный компонент Lc, определяющий длину 
компонента CDF. 

Необязательным для ответа является компонент RDF, который содержит данные 
ответа. Если при выполнении команды ожидается, что в ответе будет 
содержаться компонент RDF, то команда должна содержать необязательный 
компонент Le, определяющий максимально возможную длину компонента RDF в 
ожидаемом ответе. 

В~\cite{APDU} определяются соглашения по возможным значениям компонент команд и 
ответов, рассматривается возможные способы представления компонент и 
взаимосвязь между ними. В частности, согласно~\cite{APDU} при наличии в команде 
компонентов Lc и Le они могут быть представлены в коротком или расширенном 
виде (они должны одновременно иметь либо короткий, либо расширенный вид). 
При этом они должны кодироваться по следующим правилам: 

\begin{enumerate}
\item[1)]
Lc в коротком виде состоит из одного октета, отличного от $\hex{00}$ и 
определяющего значения от 1 до 255; 

\item[2)] 
Lc в расширенном виде состоит из трех октетов, при этом первый октет 
равен $\hex{00}$, а остальные два октета отличны от 
$\hex{0000}$ и определяют значения от 1 до 65535; 

\item[3)] 
Le в коротком виде состоит из одного октета, определяющего значения 
от 1 до 256 (значению 256 соответствует $\hex{00}$); 

\item[4)] 
если компонент Lc присутствует в команде, то Le в расширенном виде 
состоит из двух октетов, которые определяют значения от 1 до 65536 
(значению 65536 соответствует $\hex{0000}$); 

\item[5)] 
если компонент Lc отсутствует в команде, то Le в расширенном виде 
состоит из трех октетов, при этом первый октет равен $\hex{00}$ и 
следует за двумя другими, которые определяют значения от 1 до 65536 
(значению 65536 соответствует $\hex{0000}$). 
\end{enumerate}

В таблице~\ref{Table.CMDS.Fmt} приводятся формат пары (команда, ответ) 
с указанием возможных длин компонентов.

\begin{table}[h]
\caption{Формат пары (команда, ответ)}\label{Table.CMDS.Fmt}
\begin{tabular}{|c|p{10.5cm}|c|}
\hline
Компонент & Описание & Длина в октетах \\
\hline
\hline
CLA & Класс команды & 1 \\
\hline
INS & Инструкция команды & 1 \\
\hline
P1 & Параметр команды & 1 \\
\hline
P2 & Параметр команды & 1 \\
\hline
Lc & Закодированная длина данных команды & 0 или 3  \\
\hline
CDF & Данные команды & 0~--- 65 535 \\
\hline
Le & Закодированная максимально возможная длина данных ответа & 0, 2 или 3 
\\
\hline
RDF & Данные ответа & 0~--- 65 536 \\
\hline
SW1 & Статус ответа & 1 \\
\hline
SW2 & Статус ответа & 1 \\
\hline
\end{tabular}
\end{table}

При формировании компонент CDF и RDF, включаемые в них данные могут быть 
представлены явно или в закодированном виде. При кодировании используются 
отличительные правила, описанные в СТБ 34.101.19 (приложение Б). В 
настоящем стандарте, если не оговорено противное, используются 
контекстно-зависимые теги, состоящие из одного или двух октетов, а данные 
кодируются как строки октетов (см. ГОСТ 34.974) и не могут превышать 65515 
октетов. При таком кодировании по данным $X\in\{0,1\}^{8*}$ 
с тегом $T\in\{0,1\}^{8*}$ строится строка 
октетов~$\der(T, X) = T \parallel L \parallel X$, 
где $L\in\{0,1\}^{8*}$~--- 
закодированная длина $X$ [см. ГОСТ 34.974 (п. 6.3)]. 

В случае если при разборе команды или ответа обнаруживается нарушение их 
формата или формата закодированных данных, то должна быть возвращена 
ошибка и соответствующее сообщение не должно обрабатываться. 

Команды используются для реализации операций, выполняемых на КТ.
Минимальный набор операций, который должен быть поддержан для КТ,
представлен в таблице~\ref{Table.Oper.List} и описывается
в~\ref{Oper.Descr}. 
В таблице для операций указываются состояния, 
в которых они могут выполняться (см.~\ref{STATES.ST}),
пароль, который должен использоваться 
в протоколе BPACE при аутентификации (см.~\ref{OBJ.PWD}), 
и уровень (мастер-файл, прикладная программа), 
на котором может выполняться операция 
(см.~\ref{FILES}).
Символ <</>> в таблице обозначает <<или>>.

% todo: AS - взаимная или односторонняя аутентификация, не всегда понятно? 

\begin{table}[p]
\caption{Операции КТ}
\label{Table.Oper.List}
\begin{tabular}{|p{7.5cm}|p{1.3cm}|p{2.6cm}|p{1.8cm}| p{1.7cm}|}
\hline
Операция & Пункт & Состояние и соединение & Пароль BPACE & Уровень \\
\hline
\hline
%
Активация личного ключа & \ref{Oper.Descr.ActivateKey} & PS/AS:AT & PIN & eSign \\
\hline
%
Активация PIN & \ref{Oper.Descr.ActivatePIN} & PS/AS:AT & PUK & eID/eSign \\
\hline
%
Выбор мастер-файла & \ref{Oper.Descr.SelectMF} & IS/PS/AS & любой & все \\
\hline
%
Выбор прикладной программы & \ref{Oper.Descr.SelectApp} & PS/AS & любой & все \\
\hline
%
Выбор элементарного файла eID & \ref{Oper.Descr.SelectEF} & AS:AT & CAN/PIN & eID \\
\hline
%
Выбор элементарного файла eSign & \ref{Oper.Descr.SelectEF} & PS/AS:AT & PIN & eSign \\
\hline
%
Выполнение основных шагов протокола BAUTH & \ref{Oper.Descr.GABAUTH} & PS & любой & MF \\
\hline
%
Выполнение шагов протокола BPACE & \ref{Oper.Descr.GABPACE} & IS/PS/AS:CP & ---/любой & MF \\
\hline
%
Выработка подписи & \ref{Oper.Descr.Signature} & PS/AS:AT & PIN & eSign \\
\hline
%
Генерация ключевой пары & \ref{Oper.Descr.GenKeys} & PS/AS:AT & PIN & eSign \\
\hline
%
Деактивация личного ключа & \ref{Oper.Descr.DeactivateKey}  & PS/AS:AT & PIN & eSign \\
\hline
%
Деактивация PIN & \ref{Oper.Descr.DeactivatePIN}  & PS/AS:AT & PIN/PUK & eID/eSign \\
\hline
%
Изменение PIN & \ref{Oper.Descr.ChangePIN} & PS/AS:CP & PIN & eID/eSign \\
\hline
%
Инициализация алгоритма выработки подписи & \ref{Oper.Descr.SetDST} & PS/AS:AT & PIN & eSign \\
\hline
%
Инициализация алгоритма разбора токена ключа & \ref{Oper.Descr.SetCT} & PS/AS:AT & PIN & eSign \\
\hline
%
Инициализация протокола BAUTH & \ref{Oper.Descr.SetBAUTH} & PS & любой & MF \\
\hline
%
Инициализация протокола BPACE & \ref{Oper.Descr.SetBPACE} & IS/PS/AS:CP & ---/любой & MF \\
\hline
%
Обновление данных eID & \ref{Oper.Descr.Update} & AS:AT & CAN/PIN & eID \\
\hline
%
Обновление данных eSign & \ref{Oper.Descr.Update} & PS/AS:AT & PIN & eSign \\
\hline
%
Переключение между соединениями & \ref{Oper.Descr.SetCS} & AS & PIN/PUK & eID/eSign \\
\hline
%
Подтверждение PIN & \ref{Oper.Descr.VerifyPIN} & PS/AS:CP & PIN & eSign \\
\hline
%
Получение значений дополнительных атрибутов & \ref{Oper.Descr.VerifyData}& AS:AT & CAN/PIN & eID \\
\hline
%
Проверка сертификата & \ref{Oper.Descr.VerifyCert} & PS & любой & MF \\
\hline
%
Проверка статуса подтверждения PIN & \ref{Oper.Descr.VerifyAuth} & PS/AS:AT & PIN & eSign \\
\hline
%
Разблокировка PIN & \ref{Oper.Descr.UnblockPIN} & PS/AS:CP  & PUK & eID/eSign \\
\hline
%
Разбор токена ключа & \ref{Oper.Descr.Decipher} & PS/AS:AT & PIN & eSign \\
\hline
%
Сброс статуса подтверждения PIN & \ref{Oper.Descr.VerifyDeauth} & PS/AS:AT  & PIN & eSign \\
\hline
%
Чтение данных eID & \ref{Oper.Descr.Read} & AS:AT & CAN/PIN & eID \\
\hline
%
Чтение данных eSign & \ref{Oper.Descr.Read} & PS/AS:AT& PIN & eSign \\
\hline
%
Уничтожение личного ключа & \ref{Oper.Descr.Terminate} & PS/AS:AT  & PIN & eSign \\
\hline
\end{tabular}
\end{table}

Типичные последовательности операций, которые могут выполняться 
с использованием КТ, описываются в~\ref{Oper.Seq}.

В~\ref{CMDS.SM} описывается преобразование команд и ответов из исходной формы в 
защищенную и обратно при использовании защищенного соединения (см.~\ref{CRYPTO.SM}). 



\section{Описание операций}\label{Oper.Descr}

\subsection{Активация личного ключа}\label{Oper.Descr.ActivateKey}

Для активации личного ключа используется команда <Activate>,
которая определяется согласно таблице~\ref{Table.Oper.ActivateCmd}.

\begin{table}[hbt]
\caption{}\label{Table.Oper.ActivateCmd}
\begin{tabular}{|c|p{14cm}|}
\hline
Компонент & Описание\\
\hline
\hline
INS & $\hex{44}$: активировать\\
\hline
$\text{P1} \parallel \text{P2}$ & $\hex{2100}$: 
активировать ключ, определяемый CDF\\
\hline
CDF &  $\der(\hex{84}, X)$,   
где $X$ определяет идентификатор личного ключа 
(см. таблицу~\ref{Table.Oper.KeyRef}) \\ 
\hline
RDF &  --- \\
\hline
$\text{SW1} \parallel \text{SW2}$ & $\hex{9000}$: ключ активирован успешно \\
%  & $ \hex{6A88}$: cсылочные данные (ключ) не найдены \\
  & другое: см. таблицу~\ref{Table.Errors.General} \\
\hline
\end{tabular}
\end{table}


При генерации личного ключа (см. \ref{Oper.Descr.GenKeys})
он автоматически активируется, 
и в вызове команды <Activate> нет явной необходимости. 
Вызов команды может потребоваться после 
принудительной деактивации ключа (см.~\ref{Oper.Descr.DeactivateKey}).  

Если ключ уже активирован, то должен быть 
возвращен статус $\text{SW1} \parallel \text{SW2} = \hex{9000}$.

Команда может быть вызвана для приложения eSign в состояниях PS, AS:AT.
В состоянии AS:AT команда может быть вызвана только
авторизованным терминалом, в сертификате которого задано право
активировать ключи (см. \ref{DATA.Access}).

Команда требует предварительной аутентификации 
по протоколу BPACE c паролем PIN.

% todo: Может стоит указывать ссылку на одну из таблиц 
% \ref{Table.DATA.Access} или \ref{Table.DATA.SMTAccess}, а не на раздел 
% целиком? 
%
% todo: Не сказано четко, когда взаимная аутентификация по BAUTH.


%%%%%%%%%%%%%%%%%%%%%%%%%%%%%%%%%%%%%%%%%%%%%%%%%%%%%%%%%
\subsection{Активация PIN}\label{Oper.Descr.ActivatePIN}

Для активации PIN (см.~\ref{OBJ.PIN})  используется
команда <Activate>, которая определяется согласно 
таблице~\ref{Table.Oper.ActivatePINCmd}.

\begin{table}[hbt]
\caption{}\label{Table.Oper.ActivatePINCmd}
\begin{tabular}{|c|p{14cm}|}
\hline
Компонент & Описание\\
\hline
\hline
INS & $\hex{44}$: активировать\\
\hline
P1 & $\hex{10}$: активировать пароль, определяемый P2\\
\hline
P2 & $\hex{03}$: PIN \\
\hline
CDF &  ---  \\
\hline 
RDF &  --- \\
\hline
$\text{SW1} \parallel \text{SW2}$ & 
  $\hex{9000}$:  PIN активирован успешно\\
  & другое: см. таблицу~\ref{Table.Errors.General}\\
\hline
\end{tabular}
\end{table}

Первоначально PIN является активированным, и в вызове команды <Activate> 
нет явной необходимости. Вызов команды может потребоваться после 
принудительной деактивации PIN (см.~\ref{Oper.Descr.DeactivatePIN}). 

Если PIN уже активирован, то должен быть возвращен 
статус $\text{SW1} \parallel \text{SW2} = \hex{9000}$.

Команда может быть вызвана для приложения eSign 
в состояниях PS, AS:AT и для приложения eID в состоянии AS:AT. 
В состоянии AS:AT команда может быть вызвана 
только авторизованным терминалом, в сертификате которого задано право
управлять паролем PIN (см. \ref{DATA.Access}).

Команда требует предварительной аутентификации по 
протоколу BPACE c паролем PUK.


%%%%%%%%%%%%%%%%%%%%%%%%%%%%%%%%%%%%%%%%%%%%%%%%%%%%
\subsection{Выбор мастер-файла}
\label{Oper.Descr.SelectMF}

Для выбора мастер-файла (см. приложение~\ref{FILES}) 
используется команда <Select File>, 
которая определяется согласно 
таблице~\ref{Table.Oper.SelectMFCmd}.

\begin{table}[hbt]
\caption{}\label{Table.Oper.SelectMFCmd}
\begin{tabular}{|c|p{14cm}|}
\hline
Компонент & Описание \\
\hline
\hline
INS & $\hex{A4}$: выбрать файл\\ 
\hline
$\text{P1} \parallel \text{P2}$ & $\hex{0000}$: выбор мастер-файла\\
\hline
CDF & --- \\
\hline 
RDF &  --- \\
\hline
$\text{SW1}\parallel\text{SW2}$ & 
$\hex{9000}$: мастер-файл выбран успешно \\
  & другое: см. таблицу~\ref{Table.Errors.General} \\
\hline
\end{tabular}
\end{table}

Первоначально выделенный файл (см. приложение~\ref{FILES}), 
соответствующий мастер-файлу, является текущим.
и в вызове команды <Select File> нет явной необходимости. 
Вызов команды может потребоваться после 
выбора приложения (см.~\ref{Oper.Descr.SelectApp})
или выбора элементарного файла (см.~\ref{Oper.Descr.SelectEF}). 

Если мастер-файл уже выбран, то должен быть возвращен 
статус $\text{SW1} \parallel \text{SW2} = \hex{9000}$.

Команда может быть вызвана в любом из состояний на 
любом уровне.

После успешного выполнения команды выделенный файл, 
соответствующий мастер-файлу, становится текущим.


%%%%%%%%%%%%%%%%%%%%%%%%%%%%%%%%%%%%%%%%%%%%%%%%%%%%%%%%%
\subsection{Выбор прикладной программы}
\label{Oper.Descr.SelectApp}

Для выбора прикладной программы используется 
команда <Select File>, 
которая определяется согласно 
таблице~\ref{Table.Oper.SelectAppCmd}.

\begin{table}[hbt]
\caption{}\label{Table.Oper.SelectAppCmd}
\begin{tabular}{|c|p{14cm}|}
\hline
Компонент & Описание \\
\hline
\hline
INS & $\hex{A4}$: выбрать файл\\ 
\hline
P1 & $\hex{04}$: выбор прикладной программы\\
\hline
P2 & $\hex{0C}$: возврат информации о файле не требуется \\
\hline
CDF & AID прикладной программы (см. приложение~\ref{FILES})\\
\hline 
RDF &  --- \\
\hline
$\text{SW1}\parallel\text{SW2}$ & 
$\hex{9000}$: прикладная программа выбрана успешно \\
  & $\hex{6A82}$: файл не найден \\
  & другое: см. таблицу~\ref{Table.Errors.General}\\
\hline
\end{tabular}
\end{table}

Команда может быть вызвана в состояниях PS и AS
на любом уровне.

Для выбора приложения eSign команда требует 
предварительной аутентификации по протоколу 
BPACE c паролем PIN или PUK.

Для выбора приложения eID команда требует 
предварительной аутентификации по протоколу BPACE c 
паролем CAN, PIN или PUK.

Если прикладная программа уже выбрана, то должен быть возвращен
статус $\text{SW1} \parallel \text{SW2} = \hex{9000}$.

После успешного выполнения команды выделенный файл 
(см. приложение~\ref{FILES}), 
соответствующий выбранной прикладной программе, становится текущим.


%%%%%%%%%%%%%%%%%%%%%%%%%%%%%%%%%%%%%%%%%%%%%%%%%%%%%%%%%%%%%%%%%%%%%%%
\subsection{Выбор элементарного файла}
\label{Oper.Descr.SelectEF}

Для выбора элементарного файла используется 
команда <Select File>, 
которая определяется согласно 
таблице~\ref{Table.Oper.SelectEFCmd}.

\begin{table}[hbt]
\caption{}\label{Table.Oper.SelectEFCmd}
\begin{tabular}{|c|p{14cm}|}
\hline
Компонент & Описание \\
\hline
\hline
INS & $\hex{A4}$: выбрать файл\\ 
\hline
P1 & $\hex{02}$: выбор элементарного файла для текущей прикладной программы\\
\hline
P2 & $\hex{0C}$: возврат информации о файле не требуется \\
\hline
CDF & FID файла для текущей прикладной программы (см. приложение~\ref{FILES})\\
\hline 
RDF &  --- \\
\hline
$\text{SW1}\parallel\text{SW2}$ & 
$\hex{9000}$: элементарный файл выбран успешно \\
  & $\hex{6A82}$: файл не найден \\
  & другое: см. таблицу~\ref{Table.Errors.General}\\
\hline
\end{tabular}
\end{table}

Команда может вызываться в состояниях PS, AS:AT 
для прикладной программы eSign и в состоянии 
AS:AT для прикладной программы eID.
В состоянии AS:AT команда может вызываться только 
авторизованным терминалом, в сертификате которого
задано право доступа к соответствующему элементарному файлу (см. \ref{DATA.Access}).

Для выбора элементарных файлов приложения eSign команда требует 
предварительной аутентификации по протоколу BPACE c 
паролем PIN.

Для выбора элементарных файлов приложения eID команда требует 
предварительной аутентификации по протоколу BPACE c 
паролем CAN или PIN.

Для выбора элементарных файлов в состоянии AS:AT
команда требует взаимной аутентификации КТ и терминала
по протоколу BAUTH (см.~\ref{Oper.Descr.SetBAUTH}).

Элементарные файлы, которые могут быть выбраны 
для приложений eSign и eID, приводятся в приложении~\ref{FILES}. 

После успешного выполнения команды выбранный элементарный файл
становится текущим.
Выбор элементарного файла может потребоваться для
чтения (см.~\ref{Oper.Descr.Read}) или 
обновления (см.~\ref{Oper.Descr.Update}) данных.

%%%%%%%%%%%%%%%%%%%%%%%%%%%%%%%%%%%%%%%%%%%%%%%%%%%%%%%%%%%%%
\subsection{Выполнение основных шагов протокола BAUTH}
\label{Oper.Descr.GABAUTH} 

Для выполнения основных шагов протокола BAUTH 
используется команда <General Authenticate>, 
которая определяется согласно 
таблице~\ref{Table.Oper.GABAUTHCmd}.

\begin{table}[hbt]
\caption{}\label{Table.Oper.GABAUTHCmd}
\begin{tabular}{|c|p{14cm}|}
\hline
Компонент & 	Описание \\
\hline
\hline
INS & $\hex{86}$: общая аутентификация \\
\hline
$\text{P1} \parallel \text{P2}$ & $\hex{0000}$: протокол уже задан\\ 
\hline
CDF & отсутствует или $\der(\hex{7C}, X)$, 
где~$X$~--- сообщение протокола, сформированное терминалом
(см. табл.~\ref{Table.Oper.BAUTH})\\
\hline 
RDF & $\der(\hex{7C}, X)$, где~$X$~--- 
сообщение протокола, сформированное КТ 
(см. таблицу~\ref{Table.Oper.BAUTH})\\
\hline
$\text{SW1} \parallel \text{SW2}$ & $\hex{9000}$: протокол (шаг) выполнен успешно \\
& $\hex{6300}$: протокол (шаг) выполнен с ошибкой\\
  & другое: см. таблицу~\ref{Table.Errors.General} \\
\hline
\end{tabular}
\end{table}

Для выполнения основных шагов протокола BAUTH
должна вызываться \doubt{цепочка команд} <General 
Authenticate> с входными и выходными данными из таблицы~\ref{Table.Oper.BAUTH}, 
которые определяются в соответствии с~\ref{CRYPTO.BAUTH}. 
При односторонней аутентификации компонент RDF в ответе на вторую команду
в цепочке не возвращается.

\begin{table}[hbt]
\caption{Данные цепочки команд, реализующих протокол BAUTH}
\label{Table.Oper.BAUTH}
\begin{tabular}{|c|c|c|}
\hline
№ вызова & Данные в команде & Данные в ответе\\
\hline
\hline
1 & --- & $\der(\hex{80}, \text{M1})$\\
\hline
2 & $\der(\hex{81}, \text{M2})$ & 
$\der(\hex{82}, \text{M3})$  \\
\hline
\end{tabular}
\end{table}

Команда может вызываться на уровне мастер-файла в состоянии PS 
непосредственно после проверки сертификата терминала
(см. \ref{Oper.Descr.VerifyCert}).

После успешного выполнения протокола BAUTH между КТ и терминалом 
устанавливается защищенное соединение (см.~\ref{CMDS.SM}).

%%%%%%%%%%%%%%%%%%%%%%%%%%%%%%%%%%%%%%%%%%%%%%%%%%%%%%%%%%%%%%%%
\subsection{Выполнение шагов протокола BPACE}
\label{Oper.Descr.GABPACE} 

Для выполнения шагов протокола BPACE 
используется команда <General Authenticate>,
которая определяется согласно 
таблице~\ref{Table.Oper.GABPACECmd}.

\begin{table}[hbt]
\caption{}\label{Table.Oper.GABPACECmd}
\begin{tabular}{|c|p{14cm}|}
\hline
Компонент & 	Описание \\
\hline
\hline
INS & $\hex{86}$: общая аутентификация \\
\hline
$\text{P1} \parallel \text{P2}$ & $\hex{0000}$: протокол уже задан\\ 
\hline
CDF & $\der(\hex{7C}, X)$, 
где~$X$~--- сообщение протокола, сформированное КП (см. таблицу~\ref{Table.Oper.BPACE})\\
\hline 
RDF & $\der(\hex{7C}, X)$, где~$X$~--- 
сообщение протокола, сформированное КТ (см. таблицу~\ref{Table.Oper.BPACE})\\
\hline
$\text{SW1} \parallel \text{SW2}$ & 
  $\hex{9000}$: протокол (шаг) выполнен успешно \\
  & $\hex{63CX}$: протокол (шаг) выполнен с ошибкой, 
значение \texttt{X} содержит количество оставшихся попыток аутентификации 
(при $\texttt{X} = 1$ пароль приостановлен, а при $\texttt{X} = 0$~--- 
заблокирован) \\
  & $\hex{6984}$: пароль деактивирован \\
  & другое: см. таблицу~\ref{Table.Errors.General} \\
\hline
\end{tabular}
\end{table}

Для выполнения шагов протокола BPACE должна вызываться цепочка 
команд <General Authenticate> с входными и выходными данными 
из таблицы~\ref{Table.Oper.BPACE}, 
которые определяются в соответствии с~\ref{CRYPTO.BPACE}. 

\begin{table}[hbt]
\caption{Данные цепочки команд, реализующих протокол BPACE}
\label{Table.Oper.BPACE}
\begin{tabular}{|c|c|c|}
\hline
№ вызова & Данные в команде & Данные в ответе\\
\hline
\hline
 & <General Authenticate> &  \\
\hline
\hline
1 & $\der(\hex{80}, \text{M1})$ & 
$\der(\hex{81}, \text{M2})$\\
\hline
2 & $\der(\hex{82}, \text{M3})$ & 
$\der(\hex{83}, \text{M4})$\\
\hline
\end{tabular}
\end{table}

Команда может вызываться на уровне мастер-файла в любом состоянии
непосредственно после инициализации протокола 
BPACE (см. \ref{Oper.Descr.SetBPACE}).

Успешное выполнение протокола BPACE с использованием пароля PIN 
устанавливает статус подтверждения пароля PIN.
В свою очередь, неуспешное выполнение протокола BPACE с использованием 
пароля PIN сбрасывает статус подтверждения пароля PIN, 
если он был установлен ранее.

После успешного выполнения протокола BPACE между КТ и КП 
устанавливается защищенное соединение (см.~\ref{CMDS.SM}).


%%%%%%%%%%%%%%%%%%%%%%%%%%%%%%%%%%%%%%%%%%%%%%%%%%%%%%%%%%%%%%%%%%%%%
\subsection{Выработка подписи}
\label{Oper.Descr.Signature}

Для выработки подписи используется 
команда <PSO: Compute Digital Signature>
(аббр. от <<Perform Security Operation: Compute Digital Signature>>),
которая определяется согласно 
таблице~\ref{Table.Oper.SignatureCmd}.

\begin{table}[hbt]
\caption{}\label{Table.Oper.SignatureCmd}
\begin{tabular}{|c|p{14cm}|}
\hline
Компонент & Описание\\ 
\hline
\hline
INS & $\hex{2A}$: управление средой безопасности \\
\hline
$\text{P1} \parallel \text{P2}$ & $\hex{9E9A}$: выработать
электронную подпись \\ 
\hline
CDF & Хэш-значение, вычисленное от данных\\
\hline 
RDF &  --- \\
\hline
$\text{SW1} \parallel \text{SW2}$ & 
  $\hex{9000}$: подпись выработана успешно \\
  & другое: см. таблицу~\ref{Table.Errors.General} \\
\hline
\end{tabular}
\end{table}

В компоненте CDF команды должно передаваться хэш-значение, которое вычислено
от подписываемых данных алгоритмом хэширования, указанным 
при инициализации алгоритма подписи, и длина которого соответствует
уровню стойкости личного ключа, выбранного при инициализации алгоритма подписи.

При выработке подписи используется алгоритм \texttt{bign-sign}
(см.~\ref{CRYPTO.StdAlg}) со стандартными параметрами, 
которые определяются неявно по личному ключу.  

%Команда требует предварительной аутентификации по 
%протоколу BPACE c паролем PIN. 

Команда может вызываться для приложения eSign в состояниях 
PS и AS:AT непосредственно после инициализации алгоритма подписи
(см. \ref{Oper.Descr.SetDST}).
%В состоянии AS:AT команда может вызываться только авторизованным 
%терминалом с правом выработки подписи (см. \ref{DATA.Access}).

Выполнение команды приводит к сбрасыванию статуса подтверждения пароля PIN
после успешного или неуспешного выполнения $N$ команд, где $N$ --- значение, 
задаваемое при инициализации протокола BPACE (см. \ref{Oper.Descr.SetBPACE}).
Для установки статуса подтверждения пароля PIN 
необходимо либо подтвердить пароль PIN (см.~\ref{Oper.Descr.VerifyPIN}), 
либо повторно выполнить аутентификацию по паролю PIN (см.~\ref{Oper.Seq.BPACE}).


%%%%%%%%%%%%%%%%%%%%%%%%%%%%%%%%%%%%%%%%%%%%%%%%%%%%%%%%%%%%%%%%%%%%%%%
\subsection{Генерация ключевой пары}\label{Oper.Descr.GenKeys}

Для генерации ключевой пары (личного и открытого ключа) прикладной программы eSign
используется команда <Generate Asymmetric Key Pair>. При выполнении команды 
сгенерированный личный ключ сохраняется в КТ,
а открытый ключ возвращается как данные ответа.
После генерации личный ключ может использоваться 
для выработки подписи (см. \ref{Oper.Descr.Signature}) и
разбора токена ключа (см. \ref{Oper.Descr.Decipher}).
Команда определяется согласно 
таблице~\ref{Table.Oper.GenKeysCmd}.

\begin{table}[hbt]
\caption{}\label{Table.Oper.GenKeysCmd}
\begin{tabular}{|c|p{14cm}|}
\hline
Компонент & Описание\\
\hline
\hline
INS & $\hex{47}$: сгенерировать ключевую пару \\
\hline
$\text{P1} \parallel\text{P2}$ & $\hex{8200}$:
вернуть открытый ключ \\
\hline
CDF & $\der(\hex{B6},\der(\hex{84}, X))$,
где $X$ определяет идентификатор генерируемого личного ключа
(см. таблицу~\ref{Table.Oper.KeyRef}) \\
\hline 
RDF & $\der(\hex{7F49}, Y)$, где $Y$ является открытым ключом длины $4l$ бит\\
\hline
$\text{SW1} \parallel \text{SW2}$ & 
$\hex{9000}$: ключевая пара сгенерирована успешно \\
  & $\hex{6984}$: ключ уже существует \\
%  & $\hex{6A88}$: cсылочные данные (ключ) не найдены \\
  & другое: см. таблицу~\ref{Table.Errors.General} \\
\hline
\end{tabular}
\end{table}

Допустимые значения идентификаторов личного ключа определяются
в соответствии с таблицей~\ref{Table.Oper.KeyRef} и зависят от
уровня стойкости личного ключа и состояния КТ:
младшя тетрада идентификатора характеризует уровень стойкости ключа,
а старшая~--- состояние, в котором ключ может использоваться.
КТ должен поддерживать генерацию ключей уровня стойкости 
$l=128$ и может поддерживать генерацию ключей 
уровней стойкости $l=192, 256$.

\begin{table}[hbt]
\caption{Допустимые значения идентификаторов личного ключа}
\label{Table.Oper.KeyRef}
\begin{tabular}{|c|c|c|c|}
\hline
Уровень стойкости & \multicolumn{2}{|c|}{Значение } & Поддержка\\
\cline{2-3}
ключа & Состояние PS & Состояние AS & \\
\hline
\hline
128 & $\hex{01}$ & $\hex{11}$ & Обязательна \\
192 & $\hex{02}$ & $\hex{12}$ & Необязательна\\
256 & $\hex{03}$ & $\hex{13}$ & Необязательна\\
\hline
\end{tabular}
\end{table}

При генерации ключей используется алгоритм \texttt{bign-sign}
со стандартными параметрами, которые определяются неявно
по идентификатору личного ключа. 

Команда может вызываться для приложения eSign в состояниях 
PS и AS:AT. В состоянии AS:AT команда может вызываться 
только авторизованным терминалом, в сертификате которого задано право
генерации ключей в терминальном режиме (см. \ref{DATA.Access}).

Команда требует предварительной аутентификации по 
протоколу BPACE c паролем PIN. При вызове
команды статус подтверждения пароля PIN должен быть 
установлен.

Выполнение команды приводит к сбрасыванию статуса подтверждения пароля PIN.
Для установки статуса подтверждения пароля PIN 
необходимо либо подтвердить пароль PIN (см.~\ref{Oper.Descr.VerifyPIN}), 
либо повторно выполнить аутентификацию по паролю PIN (см.~\ref{Oper.Seq.BPACE}).

Если при выполнении команды личный ключ с указанным в команде идентификатором
уже существует, то должен быть возвращен статус 
$\text{SW1} \parallel \text{SW2} = \hex{6984}$. 
При этом для генерации нового ключа старый ключ должен быть предварительно 
уничтожен командой <Terminate> (см.~\ref{Oper.Descr.Terminate}).


%%%%%%%%%%%%%%%%%%%%%%%%%%%%%%%%%%%%%%%%%%%%%%%%%%%%%%%%%%%%%%%%%%%%%%%
\subsection{Деактивация личного ключа}
\label{Oper.Descr.DeactivateKey} 

Для деактивации личного ключа 
используется команда <Deactivate>,
которая определяется согласно 
таблице~\ref{Table.Oper.DeactivateKeyCmd}.

\begin{table}[hbt]
\caption{}\label{Table.Oper.DeactivateKeyCmd}
\begin{tabular}{|c|p{14cm}|}
\hline
Компонент & Описание\\
\hline
\hline
INS & $\hex{04}$: деактивировать\\
\hline
$\text{P1} \parallel \text{P2}$ & $\hex{2100}$: 
деактивировать ключ, определяемый полем данных\\
\hline
CDF &  $\der(\hex{84}, X)$,   
где $X$ определяет идентификатор личного ключа 
(см. таблицу~\ref{Table.Oper.KeyRef})\\
\hline 
RDF & --- \\
\hline
$\text{SW1} \parallel \text{SW2}$ & 
$\hex{9000}$: ключ деактивирован успено \\
%  & $\hex{6A88}$: cсылочные данные (ключ) не найдены \\
  & другое: см. таблицу~\ref{Table.Errors.General} \\
\hline
\end{tabular}
\end{table}

Команда может вызываться для приложения eSign в состояниях 
PS и AS:AT. В состоянии AS:AT команда может вызываться 
только авторизованным терминалом, в сертификате которого задано право
деактивировать ключи (см. \ref{DATA.Access}).

Команда требует предварительной аутентификации по 
протоколу BPACE c паролем PIN. 

Если ключ уже деактивирован, то должен быть возвращен статус
$\text{SW1} \parallel \text{SW2} = \hex{9000}$.

После деактивации личного ключа выполнение любых операций, 
требующих его использования, становится невозможным.

%%%%%%%%%%%%%%%%%%%%%%%%%%%%%%%%%%%%%%%%%%%%%%%%%%%%%%%%%%%%%%%%%
\subsection{Деактивация PIN}
\label{Oper.Descr.DeactivatePIN} 

Для деактивации PIN (см.~\ref{OBJ.PIN}) 
используется команда <Deactivate>, 
которая определяется согласно 
таблице~\ref{Table.Oper.DeactivatePINCmd}.

\begin{table}[hbt]
\caption{}\label{Table.Oper.DeactivatePINCmd}
\begin{tabular}{|c|p{14cm}|}
\hline
Компонент & Описание \\
\hline
\hline
INS & $\hex{04}$: деактивировать \\
\hline
P1 & $\hex{10}$: деактивировать пароль, определяемый P2\\
\hline
P2 & $\hex{03}$: PIN \\
\hline
CDF &  --- \\
\hline 
RDF & --- \\
\hline
$\text{SW1} \parallel \text{SW2}$ & 
$\hex{9000}$: PIN деактивирован успено \\
  & другое: см. таблицу~\ref{Table.Errors.General}\\
\hline
\end{tabular}
\end{table}

Команда может вызываться для приложения eSign в состояниях 
PS, AS:AT и для приложения eID в состоянии AS:AT. 
В состоянии AS:AT команда может вызываться 
только авторизованным терминалом,
в сертификате которого задано право 
управлять паролем PIN или право деактивировать 
пароль PIN (см. \ref{DATA.Access}).

Команда требует предварительной аутентификации по 
протоколу BPACE c паролем PIN или PUK. 

Если PIN уже деактивирован, то должен быть возвращен 
статус $\text{SW1} \parallel \text{SW2} = \hex{9000}$.

После деактивации PIN выполнение любых операций, 
требующих аутентификации по паролю PIN (см. таблицу~\ref{Table.Oper.List}), 
становится невозможным. 
При этом КТ переходит в состояние IS и мастер-файл 
становится текущим выделенным файлом. 


%%%%%%%%%%%%%%%%%%%%%%%%%%%%%%%%%%%%%%%%%%%%%%%%%%%%%%%%%%%%%%%%
\subsection{Изменение PIN}
\label{Oper.Descr.ChangePIN}

Для изменения PIN используется команда <Change reference data>,
которая определяется согласно 
таблице~\ref{Table.Oper.ChangePINCmd}.

\begin{table}[hbt]
\caption{}\label{Table.Oper.ChangePINCmd}
\begin{tabular}{|c|p{14cm}|}
\hline
Компонент & 	Описание \\
\hline
\hline
INS & $\hex{24}$: изменить данные\\
\hline
P1 & $\hex{00}$: задавать старое и новое значение пароля \\
   & $\hex{01}$: задавать только новое значение пароля\\
\hline
P2 & $\hex{03}$: значение PIN \\
\hline
CDF & при $\hex{00}$ старое и новое значение PIN в формате UTF8\\
    & при $\hex{01}$ новое значение PIN в формате UTF8\\
\hline 
RDF & 	 --- \\
\hline
$\text{SW1}\parallel\text{SW2}$ & 
 $\hex{9000}$: PIN изменен успешно \\
  & другое: произошла ошибка (см. таблицу~\ref{Table.Errors.General})\\
\hline
\end{tabular}
\end{table}

Команда может вызываться для приложения eSign в состояниях 
PS, AS:AT и для приложения eID в состоянии AS:AT. 
В состоянии AS:AT команда может вызываться 
только авторизованным терминалом,
в сертификате которого задано право 
управлять паролем PIN или право изменять пароль 
PIN (см. \ref{DATA.Access}).

Команда требует предварительной аутентификации по 
протоколу BPACE c паролем PIN. 

Выполнение команды приводит к сбрасыванию 
статуса подтверждения пароля PIN.
Для установки статуса подтверждения пароля PIN 
необходимо либо подтвердить пароль PIN (см.~\ref{Oper.Descr.VerifyPIN}), 
либо повторно выполнить аутентификацию по паролю PIN (см.~\ref{Oper.Seq.BPACE}).


%%%%%%%%%%%%%%%%%%%%%%%%%%%%%%%%%%%%%%%%%%%%%%%%%%%%%%%%%%
\subsection{Инициализация алгоритма выработки подписи}
\label{Oper.Descr.SetDST}

Для инициализации алгоритма подписи используется команда     
<MSE: Set DST> (аббревиатура от <<Manage Security Environment: Set 
Digital Signature Template>).
Команда задает личный ключ и алгоритм
хэширования, которые будут использоваться при выработке подписи.
Команда определяется согласно таблице~\ref{Table.Oper.SetDSTCmd}.

\begin{table}[hbt]
\caption{}\label{Table.Oper.SetDSTCmd}
\begin{tabular}{|c|p{14cm}|}
\hline
Компонент & Описание \\
\hline
\hline
INS & $\hex{22}$: управление средой безопасности\\ 
\hline
$\text{P1} \parallel\text{P2}$ & $\hex{41B6}$: 
выбрать для алгоритма выработки подписи \\
\hline
CDF & 
$\der(\hex{84}, X) \parallel \der(\hex{80}, Y)$, 
где $X$ определяет идентификатор личного ключа (см. таблицу \ref{Table.Oper.KeyRef}), 
а $Y$ определяет идентификатор используемого при выработке
подписи алгоритма хэширования 
и принимает значение $\hex{90}$ для алгоритма хэширования \texttt{belt-hash},  
$\hex{B0}$ для алгоритма хэширования \texttt{bash384} и 
$\hex{C0}$ для алгоритма хэширования \texttt{bash512} (см.~\ref{CRYPTO.StdAlg})\\
\hline 
RDF &  --- \\
\hline
$\text{SW1} \parallel \text{SW2}$ & 
$\hex{9000}$: алгоритм инициализирован успешно \\
  & $\hex{6283}$: личный ключ деактивирован \\
  & $\hex{6984}$: личный ключ уничтожен \\
%  & $\hex{6A88}$: cсылочные данные (ключ) не найдены \\
  & другое: см. таблицу~\ref{Table.Errors.General} \\
\hline
\end{tabular}
\end{table}

\if 0

\begin{table}[hbt]
\caption{Допустимые значения идентификаторов алгоритмов хэширования}
\label{Table.Oper.AlgRef}
\begin{tabular}{|c|c|c|}
\hline
Уровень стойкости ключа & Алгоритм хэширования  & Идентификатор  \\
\hline
\hline
128 & \texttt{belt-hash} & $\hex{90}$ \\
192 & \texttt{bash384} & $\hex{B0}$ \\
256 & \texttt{bash512} & $\hex{C0}$ \\
\hline
\end{tabular}
\end{table}

\fi


В компоненте CDF алгоритм хэширования \texttt{belt-hash}
может быть задан только совместно с личным ключом 
уровня стойкости $l=128$,
а алгоритмы хэширования \texttt{bash384} и \texttt{bash512}~--- 
с личными ключами уровней стойкости $l=192$ или $l=256$
соответственно.
Уровень стойкости алгоритма хэширования \texttt{bash} определяется 
неявно по уровню стойкости выбранного личного ключа.

Команда требует предварительной аутентификации по 
протоколу BPACE c паролем PIN. 
При вызове команды статус подтверждения 
пароля PIN должен быть установлен.
Для установки статуса подтверждения пароля PIN 
необходимо либо подтвердить пароль PIN (см.~\ref{Oper.Descr.VerifyPIN}), 
либо повторно выполнить аутентификацию по паролю PIN (см.~\ref{Oper.Seq.BPACE}).

Команда может вызываться для приложения eSign в 
состояниях PS и AS:AT. В состоянии AS:AT команда 
может вызываться только авторизованным терминалом,
в сертификате которого задано право 
выработки подписи в терминальном режиме (см. \ref{DATA.Access}).


%%%%%%%%%%%%%%%%%%%%%%%%%%%%%%%%%%%%%%%%%%%%%%%%%%%%%%%%%%%%%%%%%%%%
\subsection{Инициализация алгоритма разбора токена ключа}
\label{Oper.Descr.SetCT}

Для инициализации алгоритма разбора токена ключа
используется команда <MSE: Set CT> 
(аббревиатура от <<Manage Security Environment: Set Confidentialy template>).
Команда задает личный ключ, который будет использоваться при разборе токена ключа.
Команда определяется согласно таблице~\ref{Table.Oper.SetCTCmd}.

\begin{table}[hbt]
\caption{}\label{Table.Oper.SetCTCmd}
\begin{tabular}{|c|p{14cm}|}
\hline
Компонент & Описание \\
\hline
\hline
INS & $\hex{22}$: управление средой безопасности\\ 
\hline
$\text{P1} \parallel\text{P2}$ & $\hex{41B8}$: 
выбрать для алгоритма разбора токена ключа
(расшифрования ключа) \\
\hline
CDF & 
$\der(\hex{84}, X)$, 
где $X$ определяет идентификатор личного ключа
(см. таблицу \ref{Table.Oper.KeyRef})\\
\hline
RDF &  --- \\
\hline
$\text{SW1} \parallel \text{SW2}$ & 
$\hex{9000}$: алгоритм инициализирован успешно \\
  & $\hex{6283}$: личный ключ деактивирован \\
  & $\hex{6984}$: личный ключ уничтожен \\
%  & $\hex{6A88}$: cсылочные данные (ключ) не найдены \\
  & другое: см. таблицу~\ref{Table.Errors.General} \\
\hline
\end{tabular}
\end{table}

Команда может вызываться для приложения eSign в 
состояниях PS и AS:AT. В состоянии AS:AT команда 
может вызываться только авторизованным терминалом,
в сертификате которого задано право 
разбора токена ключа в терминальном режиме (см. \ref{DATA.Access}).

Команда требует предварительной аутентификации по 
протоколу BPACE c паролем PIN. 


%%%%%%%%%%%%%%%%%%%%%%%%%%%%%%%%%%%%%%%%%%%%%%%%%%%%%%%%%%%%%
\subsection{Инициализация протокола BAUTH}
\label{Oper.Descr.SetBAUTH}

Для инициализации протокола BAUTH используется
команда <MSE: Set AT> 
(аббревиатура от <<Manage Security Environment: Set 
Authentication Template>>), 
которая определяется согласно 
таблице~\ref{Table.Oper.SetBAUTHCmd}.

\begin{table}[hbt]
\caption{}\label{Table.Oper.SetBAUTHCmd}
\begin{tabular}{|c|p{14cm}|}
\hline
Компонент & Описание \\
\hline
\hline
INS & $\hex{22}$: управление средой безопасности\\ 
\hline
$\text{P1} \parallel\text{P2}$ & $\hex{C1A4}$: выбрать и 
инициализировать протокол BAUTH с взаимной 
аутентификацией\\ 
 & $\hex{81A4}$: выбрать и инициализировать протокол BAUTH с 
односторонней аутентификацией\\
\hline
CDF & Объект данных 
$\der(\hex{80}, X)$, где~$X$~--- 
закодированный объектный идентификатор (без поля тега и поля 
длины) протокола (см. приложение~\ref{ASN})\\
 & Объект данных $\der(\hex{85}, X)$, 
где~$X$ определяет хэш-значение запроса аутентификации (см.~\ref{FLOW})\\
 & Объект данных~$X$, который является 
закодированным значением типа \verb|AuthAuxData| (см.~\ref{DATA.Optional}). 
Используется в команде <Verify> (см.~\ref{Oper.Descr.VerifyData}) 
для получения значений дополнительных атрибутов DocumentValidity, 
AgeVerification, RegionVerification. В~$X$ 
обязательно должны быть включены параметры атрибута DocumentValidity\\
\hline 
RDF &  --- \\
\hline
$\text{SW1} \parallel \text{SW2}$ & 
$\hex{9000}$: протокол инициализирован успешно \\
  & другое: см. таблицу~\ref{Table.Errors.General}\\
\hline
\end{tabular}
\end{table}

В компоненте CDF все объекты данных являются обязательными, 
они должны передаваться с помощью одной команды, 
т.е. использование цепочки команд <MSE: Set AT> не допускается. 
При этом порядок следования объектов данных в компоненте CDF не важен. 

\if 0
В случае если для команды <MSE: Set AT> в компоненте CDF необходимо 
передать несколько объектов данных, то они должны передаваться с помощью 
одной команды, т.е. использование цепочки команд <MSE: Set AT> не 
допускается. При этом порядок следования объектов данных в компоненте CDF 
не важен. 
\fi

Команда может вызываться на уровне мастер-файла в состоянии PS 
непосредственно после выполнения протокола BPACE (см. \ref{Oper.Descr.GABPACE}).

%\doubt{Для использования прикладной программы eID  должен инициализироваться
%протокол BAUTH с взаимной  аутентификацией.}

При успешном выполнении протокола BAUTH дата, которая передается в команде 
для проверки срока действия КТ, может использоваться для обновления даты, 
которая хранится на КТ (см.~\ref{OBJ.Date}). 


\subsection{Инициализация протокола BPACE}
\label{Oper.Descr.SetBPACE}

Для инициализации протокола BPACE используется команда
<MSE: Set AT> (аббревиатура от <<Manage Security Environment: Set 
Authentication Template>>), 
которая определяется согласно 
таблице~\ref{Table.Oper.SetBPACECmd}.

\begin{table}[h]
\caption{}\label{Table.Oper.SetBPACECmd}
\begin{tabular}{|c|p{14cm}|}
\hline
Компонент & Описание \\
\hline
\hline
INS & $\hex{22}$: управление средой безопасности\\ 
\hline
$\text{P1} \parallel\text{P2}$ & $\hex{C1A4}$: выбрать и 
инициализировать протокол BPACE\\ 
\hline
CDF & Обязательный объект данных 
$\der(\hex{80}, X)$, где~$X$~--- 
закодированный объектный идентификатор (без поля тега и поля 
длины) протокола (см. приложение~\ref{ASN})\\
& Обязательный объект данных $\der(\hex{83}, X)$, 
где $X$ определяет, какой пароль будет использоваться в протоколе: 
$\hex{02}$~--- CAN,  $\hex{03}$~--- PIN, 
$\hex{04}$~--- PUK\\
 & Необязательный объект данных~$X$, который является 
закодированным значением типа \verb|CertHAT| (см.~\ref{DATA.Access}). 
Используется для установки владельцем 
КТ ограничений на доступ к данным и сервисам определенной 
прикладной программы КТ.\\
 & Необязательный объект данных $\der(\hex{53}, N)$, 
где $N$ является октетом и 
определяет количество последовательных подписей, 
которые могут быть выработаны в текущем сеансе 
без повторного подтверждения владельца КТ.
При этом октеты $\hex{01}, \hex{02}, \ldots, \hex{FF}$
определяют для $N$ значения $1, 2, \ldots, 255$ соответственно, 
а октет $\hex{00}$ определяет, что может быть выработано 
неограниченное количество последовательных подписей. 
Может задаваться, если предполагается использование 
прикладной программы eSign.  
Если данный объект не задан, то прикладной программой eSign 
испольуется значение по умолчанию, равное 1. \\
\hline 
RDF &  --- \\
\hline
$\text{SW1} \parallel \text{SW2}$ & 
  $\hex{9000}$: протокол инициализирован успешно,
количество возможных попыток аутентификации равно начальному значению \\
 & $\hex{63CX}$: протокол инициализирован успешно,
значение \texttt{X} содержит количество 
оставшихся попыток аутентификации, которое не равно начальному значению
(при $\texttt{X} = 1$ пароль приостановлен, а при $\texttt{X} = 0$~--- заблокирован)\\
& $\hex{6984}$: пароль деактивирован \\
 & другое: см. таблицу~\ref{Table.Errors.General} \\
\hline
\end{tabular}
\end{table}


В случае если для команды <MSE: Set AT> в компоненте CDF необходимо 
передать несколько объектов данных, то они должны передаваться с помощью 
одной команды, т.е. использование цепочки команд <MSE: Set AT> не 
допускается. При этом порядок следования объектов данных в компоненте CDF 
не важен. Например, для протокола BPACE, использующего PIN, компонент CDF, 
содержащий только обязательные объекты данных, может иметь вид: 
$\text{CDF} = 
\der(\hex{83}, \hex{03}) \parallel 
\der(\hex{80}, X)$, где~$X$~--- закодированный (без поля тега и 
поля длины) объектный идентификатор протокола BPACE (см. СТБ 34.101.66). 

%Для разблокировки или активации PIN (см.~\ref{OBJ.PIN})
%при инициализации BPACE в команде <MSE: Set AT> требуется использовать PUK 
%(см.~\ref{OBJ.PUK}).
%До выполнения протокола BPACE должна быть выбрана прикладная 
%программа КТ, пароль которой будет передаваться в команде <MSE: Set AT>. 
%Для этого должна быть вызвана команда <Select File> с нужным 
%идентификатором прикладной программы (AID, см. приложение~\ref{FILES}). 

Для возобновления PIN (см.~\ref{OBJ.PIN}) требуется 
использовать при инициализации BPACE пароль CAN. 
Тип пароля, который должен использоваться при 
инициализации BPACE для возможности выполнения различных 
операций, определяется в таблице~\ref{Table.Oper.List}.

При инициализации BPACE задаются разрешения 
владельца на доступ к сервисам и данным  
КТ, а также количество подписей, которые могут быть выработаны 
в текущем сеансе без повторного 
подтверждения владельцем КТ пароля PIN.

Команда может вызываться в любом из состояний на уровне мастер-файла.


%%%%%%%%%%%%%%%%%%%%%%%%%%%%%%%%%%%%%%%%%%%%%%%%%%%%%%%%%
\subsection{Обновление данных}
\label{Oper.Descr.Update}

Для обновления данных элементарных файлов используется
команда <Update Binary>, 
которая определяется согласно 
таблице~\ref{Table.Oper.UpdateCmd}.

\begin{table}[hbt]
\caption{}\label{Table.Oper.UpdateCmd}
\begin{tabular}{|c|p{14cm}|}
\hline
Компонент & Описание\\
\hline
\hline
INS & $\hex{D6}$: обновление бинарных данных\\
\hline
P1 & Старший байт смещения, с которого бдут перезаписываться данные 
(старший бит должен быть равен нулю) \\
\hline
P2 & Младший байт смещения, с которого будут перезаписываться данные \\
\hline
CDF & Записываемые данные \\
\hline 
RDF &  --- \\
\hline
$\text{SW1} \parallel \text{SW2}$ & 
$\hex{9000}$: данные перезаписаны успешно \\
 & другое: см. таблицу~\ref{Table.Errors.General} \\
\hline
\end{tabular}
\end{table}

Размер данных, которые нужно записать, определяется компонентом Lс команды 
(см.~\cite{APDU}).

Команда может вызываться в состояниях PS, AS:AT 
для прикладной программы eSign и в состоянии 
AS:AT для прикладной программы eID.
В состоянии AS:AT команда может вызываться только 
авторизованным терминалом, в сертификате которого
задано право обновления соответствующего 
элементарного файла (см. \ref{DATA.Access}).

Для обновления элементарных файлов приложения eSign команда требует 
предварительной аутентификации по протоколу BPACE c 
паролем PIN.

Для обновления элементарных файлов приложения eID команда требует 
предварительной аутентификации по протоколу BPACE c 
паролем CAN или PIN.

Файл, в который записываются данные, должен быть предварительно
выбран (см. \ref{Oper.Descr.SelectEF}).
Для записи могут быть выбраны только те файлы, для которых 
нет ограничений по записи при текущем состоянии КТ (см.~\ref{DATA.Access}). 
При попытке записи в файлы, доступ к которым ограничен при текущем состоянии КТ, 
должен возвращаться статус $\text{SW1} \parallel\text{SW2} = \hex{6982}$. 


\subsection{Переключение между соединениями}
\label{Oper.Descr.SetCS}

Для переключения между защищенным соединением,
устанавливаемым между КТ и КП после выполнения 
протокола BPACE, и соединением, 
устанавливаемым между КТ и терминалом после выполнения 
протокола BAUTH, используется 
команда <MSE: Set CS> (аббревиатура от <<Manage Security Environment: Set 
Context Switch>),
которая определяется согласно 
таблице~\ref{Table.Oper.SetCSCmd}.

\begin{table}[hbt]
\caption{}\label{Table.Oper.SetCSCmd}
\begin{tabular}{|c|p{14cm}|}
\hline
Компонент & Описание \\
\hline
\hline
INS & $\hex{22}$: управление средой безопасности\\ 
\hline
$\text{P1} \parallel\text{P2}$ & $\hex{01A4}$: 
переключиться между соединениями \\
\hline
CDF & 
$\der(\hex{E1}, \der(\hex{81}, X))$, 
где $X$ определяет идентификатор
соединения и принимает значение $\hex{00}$ для
переключения на защищенное соединение,
установленное между КТ и КП,
и значение $\hex{01}$ для переключения на защищенное
соединение, установленное между КТ и терминалом\\ 
\hline 
RDF &  --- \\
\hline
$\text{SW1} \parallel \text{SW2}$ & 
$\hex{9000}$: переключение между соединениями выполнено успешно \\
 & другое: см. таблицу~\ref{Table.Errors.General} \\
\hline
\end{tabular}
\end{table}

Команда может вызываться для приложений eID и eSign
в состоянии AS.

Команда требует предварительной аутентификации по протоколу BPACE
с паролем PIN или PUK.

Переключение между соединениями может понадобиться
при подтверждении (см. \ref{Oper.Descr.VerifyPIN}), 
изменении (см. \ref{Oper.Descr.ChangePIN})
или разблокировке (см. \ref{Oper.Descr.UnblockPIN}) PIN.

После успешного выполнения команды выделенный файл (см.
приложение~\ref{FILES}), 
соответствующий мастер-файлу, становится текущим.

%%%%%%%%%%%%%%%%%%%%%%%%%%%%%%%%%%%%%%%%%%%%%%%%%
\subsection{Подтверждение PIN}
\label{Oper.Descr.VerifyPIN}

Для подтверждения PIN используется команда
<Verify>, которая определяется согласно 
таблице~\ref{Table.Oper.VerifyPINCmd}.

\begin{table}[hbt]
\caption{}\label{Table.Oper.VerifyPINCmd}
\begin{tabular}{|c|p{14cm}|}
\hline
Компонент & Описание \\
\hline
\hline
INS & $\hex{20}$: проверить данные\\
\hline
$\text{P1} \parallel \text{P2}$ & $\hex{0003}$: 
проверка пароля PIN\\
\hline
CDF & пароль PIN в формате UTF8 \\
\hline 
RDF &  --- \\
\hline
$\text{SW1} \parallel \text{SW2}$ & $\hex{9000}$: аутентификация успешна\\
 & $\hex{63CX}$: аутентификация неуспешна, значение \texttt{X} содержит количество 
оставшихся попыток аутентификации (при $\texttt{X} = 1$ пароль 
приостановлен, а при $\texttt{X} = 0$~--- заблокирован)\\
& $\hex{6984}$: пароль деактивирован \\
 & другое: см. таблицу~\ref{Table.Errors.General} \\
\hline
\end{tabular}
\end{table}

Команда может вызываться для приложения eSign
в состояниях PS и AS:CP.

Команда требует предварительной аутентификации по 
протоколу BPACE с паролем PIN.

Успешное выполнение команды устанавливает 
статус подтверждения пароля PIN.
В свою очередь, неуспешное выполнение команды 
сбрасывает данный статус.

Команду может потребоваться вызвать после сброса статуса
подтверждения пароля PIN, который происходит 
после выработки определенного количества подписей (см.~\ref{Oper.Descr.Signature}),
генерации ключевой пары (см.~\ref{Oper.Descr.GenKeys}),
изменения пароля PIN (см.~\ref{Oper.Descr.ChangePIN}),
уничтожение личного ключа (см.~\ref{Oper.Descr.Terminate}).

%%%%%%%%%%%%%%%%%%%%%%%%%%%%%%%%%%%%%%%%%%%%%%%%%%%%%%%
\subsection{Проверка дополнительного атрибута}
\label{Oper.Descr.VerifyData}

Для проверки дополнительных 
атрибутов DocumentValidity, AgeVerification, RegionVerification 
(см.~\ref{DATA.Optional}) 
используется команда  <Verify>, 
которая определяется согласно 
таблице~\ref{Table.Oper.VerifyDataCmd}.

\begin{table}[hbt]
\caption{}\label{Table.Oper.VerifyDataCmd}
\begin{tabular}{|c|p{14cm}|}
\hline
Компонент & Описание \\
\hline
\hline
INS & $\hex{20}$: проверить данные\\
\hline
$\text{P1} \parallel \text{P2}$ & $\hex{8000}$: 
проверить дополнительный атрибут\\
\hline
CDF & Закодированный идентификатор дополнительного атрибута 
(см.~\ref{DATA.Optional}). 
Может принимать одно из значений 
\verb|id-DocumentValidity|, \verb|id-AgeVerification|, \verb|id-PlaceVerification| 
(см. приложение~\ref{ASN})\\
\hline 
RDF &  --- \\
\hline
$\text{SW1} \parallel \text{SW2}$ & $\hex{9000}$: атрибут проверен успешно\\
 & $\hex{6300}$: атрибут проверен неуспешно\\
 & другое: см. таблицу~\ref{Table.Errors.General} \\
\hline
\end{tabular}
\end{table}

Команда может вызываться для приложения eID в состоянии AS:AT
авторизованным терминалом, в сертификате которого задано право
проверки соответствующего атрибута (см. \ref{DATA.Access}).  

Команда требует предварительной аутентификации по протоколу BPACE 
c паролем CAN или PIN.

Установка проверяемых командой данных производится 
при инициализации протокола BAUTH (см.~\ref{Oper.Descr.SetBAUTH}).  

Для команды должен использоваться $\text{CLA}=\hex{84}$ 
(прикладной класс для защищенного соединения без использования цепочки 
команд). 




%%%%%%%%%%%%%%%%%%%%%%%%%%%%%%%%%%%%%%%%%%%%%%%%%%%%%%%%%%%%%%
\subsection{Проверка сертификата}
\label{Oper.Descr.VerifyCert}

Для проверки сертификата используется команда 
<PSO: Verify Certificate> (аббревиатура от <<Perform Security 
Operation: Verify Certificate>>).
Команда применяется для импорта и проверки 
сертификата при аутентификации терминала по протоколу BAUTH. 
которая определяется согласно 
таблице~\ref{Table.Oper.VerifyCertCmd}.

\begin{table}[hbt]
\caption{}\label{Table.Oper.VerifyCertCmd}
\begin{tabular}{|c|p{14cm}|}
\hline
Компонент & Описание\\ 
\hline
\hline
INS & $\hex{2A}$: управление средой безопасности \\
\hline
$\text{P1} \parallel \text{P2}$ & $\hex{00BE}$: проверить 
сертификат откыртого ключа \\ 
\hline
CDF & Закодированный стандартный сертификат (см.~\ref{CERTS.Std})\\
 & Закодированное значение компонента certificateBody, определяющего тело 
облегченного сертификата (см.~\ref{CERTS.Light})\\
 & Закодированное значение компонента signature, определяющего подпись 
облегченного сертификата (см.~\ref{CERTS.Light})\\
\hline 
RDF &  --- \\
\hline
$\text{SW1} \parallel \text{SW2}$ & $\hex{9000}$: сертификат проверен успешно \\
 & другое: см. таблицу~\ref{Table.Errors.General} \\
\hline
\end{tabular}
\end{table}

В компоненте CDF команды должен передаваться либо стандартный сертификат, 
либо тело и подпись облегченного сертификата. Сертификаты могут 
передаваться с использованием цепочки команд <PSO: Verify Certificate>. 

Для передачи стандартного сертификата с использованием
цепочки команд должна использоваться цепочка команд,
содержащих последовательные части сертификата. 

Для передачи облегченного сертификата с использованием
цепочки команд должна использоваться цепочка из двух команд,
одна из которых содержит тело облегченного сертификата,
а вторая~--- подпись облегченного сертификата.

Для передачи нескольких сертификатов, составляющих маршрут 
сертификации, должен использоваться 
последовательный вызов команд <PSO: Verify Certificate>. 
При этом, 
%должны использоваться сертификаты одного типа и 
если какой-либо сертификат передается цепочкой команд, 
то для него должна использоваться своя цепочка.

Открытый ключ, используемый при проверке сертификата после его передачи в
КТ, извлекается из сертификата, который хранится на КТ или который был 
импортирован в КТ предшествующим успешным вызовом команды <PSO: Verify Certificate>.

% todo: пояснить процедуру проверки сертификата:
% - на КТ хранится сертификат/ОК доверенного УЦ; ОК помещается в буфер;
% - при проверке следующего сертификата из цепочки используется ОК из буфера,
% - при успехе проверки ОК из сертификата попадает в буфер, 
%   иначе буфер \doubt{очищается};
% - при последующей работе (выполнении шагов протокола BAUTH) используется 
%   ОК из буфера.

Команда может вызываться на уровне мастер-файла 
в состоянии PS непосредственно после 
инициализации протокола BAUTH (см. \ref{Oper.Descr.SetBAUTH}).

Команда требует предварительной аутентификации по протоколу BPACE 
c паролем CAN, PIN или PUK. Если аутентификация 
по протоколу BPACE выполнялась с паролем CAN,
то в сертификате терминала должно быть установлено 
право доступа по паролю CAN  (см. \ref{DATA.Access}).

Если сертификат является недействтельным, то должен 
возвращаться статус $\text{SW1} \parallel \text{SW2} = \hex{6A80}$.

%%%%%%%%%%%%%%%%%%%%%%%%%%%%%%%%%%%%%%%%%%%%%%%%%%
\subsection{Проверка статуса подтверждения PIN}
\label{Oper.Descr.VerifyAuth}

Для проверки статуса подтверждения PIN
используется команда <Verify>, 
которая определяется согласно 
таблице~\ref{Table.Oper.VerifyAuthCmd}.

\begin{table}[hbt]
\caption{}\label{Table.Oper.VerifyAuthCmd}
\begin{tabular}{|c|p{14cm}|}
\hline
Компонент & Описание \\
\hline
\hline
INS & $\hex{20}$: проверить данные\\
\hline
$\text{P1} \parallel \text{P2}$ & $\hex{0003}$: проверить статус подтверждения пароля PIN \\
\hline
CDF & --- \\
\hline 
RDF &  --- \\
\hline
$\text{SW1} \parallel \text{SW2}$ & $\hex{9000}$: 
 статус подтверждения пароля PIN установлен\\
 & другое: см. таблицу~\ref{Table.Errors.General} \\
\hline
\end{tabular}
\end{table}

Команда может вызываться для приложения eSign в состояниях PS и AS:AT.

Команда требует предварительной аутентификации по протоколу BPACE
c паролем PIN.

Если статус подтверждения пароля PIN не установлен, то должен 
возвращаться статус $\text{SW1} \parallel \text{SW2} = \hex{6982}$.

%%%%%%%%%%%%%%%%%%%%%%%%%%%%%%%%%%%%%%%%%%%%%%%%%%%%%%%%%
\subsection{Разблокировка PIN}
\label{Oper.Descr.UnblockPIN}

Для разблокировки PIN используется команда
<Reset Retry Counter>,
которая определяется согласно 
таблице~\ref{Table.Oper.UnblockPINCmd}.

\begin{table}[hbt]
\caption{}\label{Table.Oper.UnblockPINCmd}
\begin{tabular}{|c|p{14cm}|}
\hline
Компонент & 	Описание \\
\hline
\hline
INS & $\hex{2С}$: разблокировать PIN\\
\hline
%P1 & $\hex{02}$: разблокировать PIN с его изменением\\
P1 & $\hex{03}$: разблокировать PIN без его изменения\\
\hline
P2 & $\hex{00}$: PIN выбран неявно\\
\hline
%CDF & при $\text{P1} = \hex{02}$ содержит новое значение PIN в формате UTF8\\
% & при $\text{P1} = \hex{03}$ не задается \\
  & --- \\
\hline 
RDF & 	 --- \\
\hline
$\text{SW1}\parallel\text{SW2}$ & $\hex{9000}$: 
разблокировка PIN была выполнена успешно\\
& другое: произошла ошибка (см. таблицу~\ref{Table.Errors.General}) \\
\hline
\end{tabular}
\end{table}

Команда может вызываться для приложения eSign в состояниях 
PS, AS:AT и для приложения eID в состоянии AS:AT. 
В состоянии AS:AT команда может вызываться 
только авторизованным терминалом,
в сертификате которого задано право 
управлять паролем PIN или право разблокировать пароль 
PIN (см. \ref{DATA.Access}).

Команда требует предварительной аутентификации по 
протоколу BPACE c паролем PUK. 

%%%%%%%%%%%%%%%%%%%%%%%%%%%%%%%%%%%%%%%%%%%%%%%%%%%%%%%%%%%%%%%
\subsection{Разбор токена ключа}
\label{Oper.Descr.Decipher}

Для разбора токена ключа используется 
команда <PSO: Decipher> (аббр. от 
<<Perform Security Operation: Decipher>>), 
которая определяется согласно 
таблице~\ref{Table.Oper.DecipherCmd}.

\begin{table}[hbt]
\caption{}\label{Table.Oper.DecipherCmd}
\begin{tabular}{|c|p{14cm}|}
\hline
Компонент & Описание\\ 
\hline
\hline
INS & $\hex{2A}$: управление средой безопасности \\
\hline
$\text{P1} \parallel \text{P2}$ & $\hex{8086}$: расшифровать
данные \\ 
\hline
CDF & токен ключа, длина которого не
меньше 96 октетов и не больше 112 октетов\\
\hline 
RDF &  расшифрованный ключ \\
\hline
$\text{SW1} \parallel \text{SW2}$ & $\hex{9000}$: 
токен разобран успешно\\
& другое: см. таблицу~\ref{Table.Errors.General} \\
\hline
\end{tabular}
\end{table}

При разборе токена используется алгоритм $\texttt{bign-keytransport}^{-1}$
(см.~\ref{CRYPTO.StdAlg})
со стандартными параметрами из СТБ 34.101.45 (приложение Б), 
уровень которых определяется неявно по личному ключу,
выбираемому при инициализации алгоритма разбора токена 
(см.~\ref{Oper.Descr.SetCT}). 

В компоненте CDF команды должен передаваться токен ключа, 
сформированный с нулевым заголовком ключа. 

Команда может вызываться для приложения eSign в состояниях 
PS и AS:AT непосредственно после успешной инициализации 
алгоритма разбора токена (см.~\ref{Oper.Descr.SetCT}).

%Порядок, в котором должна вызываться команда, 
%определяется в~\ref{Oper.Seq.Decipher}.





%%%%%%%%%%%%%%%%%%%%%%%%%%%%%%%%%%%%%%%%%%%%
\subsection{Сброс статуса подтверждения PIN}
\label{Oper.Descr.VerifyDeauth}

Для сброса статуса подтверждения пароля PIN
используется команда <Verify>,
которая определяется согласно 
таблице~\ref{Table.Oper.VerifyDeauthCmd}.

\begin{table}[hbt]
\caption{}\label{Table.Oper.VerifyDeauthCmd}
\begin{tabular}{|c|p{14cm}|}
\hline
Компонент & Описание \\
\hline
\hline
INS & $\hex{20}$: проверить данные\\
\hline
$\text{P1} \parallel \text{P2}$ & $\hex{FF03}$: сбросить статуса подтверждения
пароля PIN  \\
\hline
CDF & ---  \\
\hline 
RDF &  --- \\
\hline
$\text{SW1} \parallel \text{SW2}$ & $\hex{9000}$ 
статус подтверждения пароля PIN сброшен успешно\\
& другое: см. таблицу~\ref{Table.Errors.General} \\
\hline
\end{tabular}
\end{table}

Команда может вызываться для приложения eSign  
в состояниях PS и AS:AT.

Команда требует предварительной аутентификации 
по протоколу BPACE c паролем PIN.

Успешное выполнение команды сбрасывает статус подтверждения пароля PIN,
а неуспешное~--- не изменяет его статус.

Если статус подтверждения пароля PIN уже сброшен, то должен быть возвращен статус
$\text{SW1} \parallel \text{SW2} = \hex{9000}$.

%%%%%%%%%%%%%%%%%%%%%%%%%%%%%%%%%%%%%%%%%%%%%%%%%%%%%%%%%%%%%%%%
\subsection{Чтение данных}
\label{Oper.Descr.Read}

Для чтения данных из элементарных файлов 
используется команда <Read Binary>, 
которая определяется согласно 
таблице~\ref{Table.Oper.ReadCmd}.

\begin{table}[hbt]
\caption{}\label{Table.Oper.ReadCmd}
\begin{tabular}{|c|p{14cm}|}
\hline
Компонент & Описание \\
\hline
\hline
INS & $\hex{B0}$: чтение бинарных данных \\
\hline
P1 & Старший байт смещения, с которого будут читаться данные (старший бит 
должен быть равен нулю) \\
\hline
P2 & Младший байт смещения, с которого будут читаться данные\\
\hline
CDF &  --- \\
\hline 
RDF & 	Прочитанные данные \\
\hline
$\text{SW1} \parallel\text{SW2}$ & 
$\hex{9000}$: данные прочитаны успешно \\
& другое: см. таблицу~\ref{Table.Errors.General} \\
\hline
\end{tabular}
\end{table}

Размер данных, которые нужно прочитать, определяется компонентом 
Le (см.~\cite{APDU}).

Команда может вызываться в состояниях PS, AS:AT 
для прикладной программы eSign и в состоянии 
AS:AT для прикладной программы eID.
В состоянии AS:AT команда может вызываться только 
авторизованным терминалом, в сертификате которого
задано право чтения соответствующего файла или группы данных
(см. \ref{DATA.Access}).

Для чтения элементарных файлов приложения eSign команда требует 
предварительной аутентификации по протоколу BPACE c 
паролем PIN.

Для чтения элементарных файлов приложения eID команда требует 
предварительной аутентификации по протоколу BPACE c 
паролем CAN или PIN.

Элементарный файл, из которого читаются данные, должен быть предварительно 
выбран (см. \ref{Oper.Descr.SelectEF}). Прочитаны могут быть только те данные, 
для которых нет ограничений по чтению при текущем состоянии КТ (см.~\ref{DATA.Access}). 
При попытке чтения данных, доступ к которым ограничен, должен возвращаться 
статус $\text{SW1} \parallel \text{SW2} = \hex{6982}$.

%%%%%%%%%%%%%%%%%%%%%%%%%%%%%%%%%%%%%%%%%%%%%%%%%%%%%%%%%%%%%%%%
\subsection{Уничтожение личного ключа}
\label{Oper.Descr.Terminate}

Для уничтожения личного ключа используется команда <Terminate>,
которая определяется согласно 
таблице~\ref{Table.Oper.TerminateCmd}.

\begin{table}[ht]
\caption{}\label{Table.Oper.TerminateCmd}
\begin{tabular}{|c|p{14cm}|}
\hline
Компонент & Описание\\
\hline
\hline
INS & $\hex{E6}$: уничтожить личный ключ \\
\hline
$\text{P1} \parallel\text{P2}$ & $\hex{2100}$:
идентификатор личного ключа задается в поле данных\\
\hline
CDF &  $\der(\hex{B6},\der(\hex{84}, X))$,
%$\der(\hex{84}, X)$,  
где $X$ определяет идентификатор личного ключа
(см. таблицу \ref{Table.Oper.KeyRef})\\ 
\hline 
RDF & ---  \\
\hline
$\text{SW1} \parallel \text{SW2}$ & 
$\hex{9000}$: ключ уничтожен успешно\\
% & $\hex{6A88}$: ссылочные данные (ключ) не найдены\\
 & другое: см. таблицу~\ref{Table.Errors.General} \\
\hline
\end{tabular}
\end{table}

Команда может вызываться для приложения eSign в состояниях PS и AS:AT.
В состоянии AS:AT команда может вызываться только
авторизованным терминалом, в сертификате которого задано право
управлять ключами (см. \ref{DATA.Access}).
В каждом из состояний могут быть уничтожены только те ключи, которые
были сгенерированы в данном состоянии (см. \ref{Oper.Descr.GenKeys}).

Команда требует предварительной аутентификации по протоколу BPACE
c паролем PIN. При вызове команды статус подтверждения пароля PIN 
должен быть установлен.

Выполнение команды приводит к сбрасыванию статуса подтверждения пароля PIN.
Для установки статуса подтверждения пароля PIN 
необходимо либо подтвердить пароль PIN (см.~\ref{Oper.Descr.VerifyPIN}), 
либо повторно выполнить аутентификацию по паролю PIN (см.~\ref{Oper.Seq.BPACE}).

При попытке уничтожить ключ, который не был сгенерирован
или который был уничтожен ранее, должен возвращаться 
статус $\text{SW1} \parallel \text{SW2} = \hex{9000}$.
\section{Последовательности операций}
\label{Oper.Seq}

\subsection{Аутентификация по паролю}
\label{Oper.Seq.BPACE}

Для аутентификации по паролю используется протокол BPACE (см.~\ref{CRYPTO.BPACE}). 
Для выполнения аутентифкации по паролю на стороне КТ должна 
использоваться следующая последовательность операций:

\begin{enumerate}
\item Выбрать мастер-файл (см.~\ref{Oper.Descr.SelectMF}), 
если он не является текущим выделенным файлом.

\item Инициализировать протокол BPACE (см.~\ref{Oper.Descr.SetBPACE}).

\item Выполнить шаги протокола BPACE (см.~\ref{Oper.Descr.GABPACE}).

\end{enumerate}

При аутентификации по паролю владелец КТ может
установить ограничения на доступ к данным и операциям КТ.
Данные ограничения задаются при инициализации 
протокола BPACE (см.~\ref{Oper.Descr.SetBPACE}).

При первом выполнении протокола BPACE обмен данными КП с КТ производится 
по незащищенному соединению. При успешном выполнении протокола между КТ и 
КП устанавливается защищенное соединение и КТ переходит в состояние PS
(см.~\ref{STATES.ST}). Новый сеанс протокола (если он 
предусмотрен) будет выполняться уже по защищенному соединению. 

Успешная аутентификация по паролю CAN возобновляет пароль PIN,  
если он был приостановлен (см.~\ref{OBJ.PIN}).

\subsection{Аутентификация терминала и КТ}
\label{Oper.Seq.BAUTH}

Аутентификация терминала и КТ производится по протоколу BAUTH (см.~\ref{CRYPTO.BAUTH}) 
после успешной аутентификации по паролю (см. \ref{Oper.Seq.BPACE}).
Аутентификация может быть односторонней (аутентификации терминала перед КТ)
или взаимной (аутентификации терминала перед КТ и КТ перед терминалом).
Для выполнения аутентифкации должна 
использоваться следующая последовательность операций:
%
\begin{enumerate}
\item Инициализировать протокол BAUTH (см.~\ref{Oper.Descr.SetBAUTH}).
\item Проверить сертификат терминала (см.~\ref{Oper.Descr.VerifyCert}).
\item Выполнить основные шаги протокола BAUTH (см.~\ref{Oper.Descr.GABAUTH}).
\end{enumerate}

Дополнительно, после успешного выполнения 
протокола BAUTH с взаимной аутентификацией сторон
срок действия КТ может быть получен с использованием
следующей последовательности операций:
%
\begin{enumerate}
\item Выбрать прикладную программу eID (см.~\ref{Oper.Descr.SelectApp}).
\item Получить срок действия КТ как дополнительный атрибут (см.~\ref{Oper.Descr.VerifyData}).
\end{enumerate}

%Аутентификация терминала и КТ может быть выполнена в \doubt{состоянии PS}.
При выполнении протокола BAUTH обмен данными терминала с КТ производится по 
защищенному соединению, установленному после выполнения протокола BPACE. 
При успешном выполнении протокола BAUTH между терминалом и КТ создается новое 
защищенное соединение, которое устанавливается в качестве текущего
соединения, при этом КТ переходит в состояние AS (см.~\ref{STATES.ST}).

\subsection{Активация пароля PIN}
\label{Oper.Seq.ActivatePIN}

Активация пароля PIN может потребоваться после его
принудительной деактивации (см.~\ref{Oper.Descr.DeactivatePIN}).

Для активации пароля PIN без использования терминала
должна использоваться следующая последовательность операций:
%
\begin{enumerate}
\item Выполнить аутентификацию по паролю PUK (см.~\ref{Oper.Seq.BPACE}).

\item Выбрать прикладную программу eSign (см.~\ref{Oper.Descr.SelectApp}).

\item Активировать пароль PIN (см.~\ref{Oper.Descr.ActivatePIN}).

\end{enumerate}

Для активации пароля PIN с использованием терминала
должна использоваться следующая последовательность операций:

\begin{enumerate}
\item Выполнить аутентификацию по паролю PUK (см.~\ref{Oper.Seq.BPACE}).

\item Выполнить аутентификацию терминала (см.~\ref{Oper.Seq.BAUTH}), 
в сертификате которого задано право управлять паролем PIN (см. \ref{DATA.Access}).

\item Выбрать прикладную программу  (см.~\ref{Oper.Descr.SelectApp}), 
для которой в сертификате авторизованного терминала установлено 
право управлять паролем PIN (см. \ref{DATA.Access}).

\item Активировать пароль PIN (см.~\ref{Oper.Descr.ActivatePIN}).

\end{enumerate}

Активация пароля PIN не приводит к изменению состояния КТ.
После активации пароля PIN для получения доступа к
операциям, которые требуют предварительной
аутентфикации по данному паролю, необходимо выполнить аутентификацию 
по паролю PIN (см.~\ref{Oper.Seq.BPACE}) и, при необходимости, 
аутентификацию терминала и КТ (см.~\ref{Oper.Seq.BAUTH}).

\subsection{Управление паролем PIN}
\label{Oper.Seq.ControlPIN}

К операциям по управлению паролем PIN относятся:
\begin{itemize} 
\item[--] подтверждение (см.~\ref{Oper.Descr.VerifyPIN});
\item[--] изменение (см.~\ref{Oper.Descr.ChangePIN});
\item[--] разблокировка (см.~\ref{Oper.Descr.UnblockPIN}).
\end{itemize}

Для подтверждения или изменения пароля PIN предварительно 
должна быть выполнена аутентификация по 
паролю PIN, а для разблокировки пароля PIN~---
аутентификация по паролю PUK (см.~\ref{Oper.Seq.BPACE}).
Разблокировка может выполняться с изменением или без 
изменения пароля PIN (см.~\ref{Oper.Descr.UnblockPIN}).

После успешной аутентифкации по паролю,
при необходимости, может быть выполнена аутентификация 
терминала и КТ (см.~\ref{Oper.Seq.BAUTH}),
при этом в сертификате терминала должно быть задано право
управления паролем PIN (см. \ref{DATA.Access}).

Изменение или разблокировка PIN может
производится прикладной программой eID или eSign,
а подтверждение пароля PIN~--- прикладной программой eSign.
Используемая для управления паролем PIN прикладная
программа дожна быть предварительно 
выбрана (см.~\ref{Oper.Descr.SelectApp}). 

В состоянии PS или AS:CP для управления паролем PIN
достаточно выполнить лишь соответствующую операцию.

В состоянии AS:AT для управления паролем PIN может
использоваться следующая последовательность операций:
%
\begin{enumerate} 
\item Переключиться на соединение,
      установленное между КТ и КП (см.~\ref{Oper.Descr.SetCS}).
\item Выбрать прикладную программу (см.~\ref{Oper.Descr.SelectApp}),
      для которой в сертификате авторизованного терминала задано 
      соответствующее право по управлению паролем PIN (см. \ref{DATA.Access}).
\item Выполнить требуемую операцию по управлению паролем PIN.
\item Переключиться на соединение, установленное между КТ 
      и терминалом (см.~\ref{Oper.Descr.SetCS}).
\item При необходимости, выбрать нужную прикладную программу (см.~\ref{Oper.Descr.SelectApp}).
\end{enumerate}


\subsection{Генерация ключевой пары и установка сертификата}
\label{Oper.Seq.GeKeySetCert}

Для генерации ключевой пары и 
установки сертификата предварительно 
должна быть выполнена аутентификация по 
паролю PIN (см.~\ref{Oper.Seq.BPACE}).

После успешной аутентифкации по паролю,
при необходимости, может быть выполнена взаимная 
аутентификация терминала и КТ (см.~\ref{Oper.Seq.BAUTH}),
при этом в сертификате терминала должно быть задано
право генерации ключей (см. \ref{DATA.Access}).

Генерация ключевой пары и установка сертификата производится 
прикладной программой eSign, которая дожна быть предварительно 
выбрана (см.~\ref{Oper.Descr.SelectApp}). 

В состояних PS и AS может быть сгенерировано
до трех ключевых пар (см.~\ref{Oper.Descr.GenKeys}), 
соответствующих различным уровням стойкости. 
Сгенерированные личные ключи могут использоваться
для подписи данных (см.~\ref{Oper.Seq.Sig}) и разбора токена 
ключа (см.~\ref{Oper.Seq.Decipher}).
При этом личный ключ, сгенерированный в одном из состояний, 
не может использоваться в другом состоянии. 

Открытый ключ, который возвращается при генерации ключевой пары
(см.~\ref{Oper.Descr.GenKeys}), должен использоваться при формировании 
запроса на выпуск сертификата.
При формировании запроса дополнительно могут использоваться данные eSign, 
специально для этого предназначенные и предварительно
установленные в КТ.
Указанные данные могут быть установлены (см.~\ref{Oper.Descr.Update}) 
авторизованным терминалом, в сертификате которого указано право устанавливать данные 
для запроса на выпуск сертификата (см.~\ref{DATA.Access}), 
при выполнении операции обновления данных  
и получены авторизованным терминалом, в сертификате которого задано право на установку 
сертификата (см. \ref{DATA.Access}),
при выполнении операции чтения данных (см.~\ref{Oper.Descr.Read}).

Запрос на выпуск сертификата должен быть подписан (см. \ref{Oper.Seq.Sig})
личным ключом, соответствующим открытому ключу из запроса.

После формирования сертификат может быть установлен в КТ.
Для установки сертификата должна использоваться операция 
обновления данных (см.~\ref{Oper.Descr.Update}). 
В состоянии AS установка сертификата должна выполняться 
авторизованным терминалом, в сертификате которого задано право на 
запись сертификата (см.~\ref{DATA.Access}). 

\doubt{В случае, если формирование запроса и запись сертификата
выполняются в разных сеансах, то 
после формирование запроса и до записи сертификата 
рекомендуется записывать вместо сертификата 
хэш-значение, вычисленное
алгоритмом \texttt{belt-hash} от закодированного значения 
подписанного запроса на выпуск сертификата.
Данное значение может быть использовано для получения от УЦ
сертификата после его выпуска (см. СТБ~34.101.78).
}

Сертификаты и данные, предназначенные для формирования запроса
на выпуск сертификатов, хранятся на КТ в определенных элементарных файлах, 
описанных в таблице~\ref{Table.FILES.EFSIGN}.
Данные для запроса на выпуск сертификата 
хранятся в элементарном файле в виде закодированного
значения типа Name, определенного в СТБ 34.101.19.

\subsection{Выработка подписи}
\label{Oper.Seq.Sig}

Для выработки подписи предварительно 
должна быть выполнена аутентификация по 
паролю PIN  (см.~\ref{Oper.Seq.BPACE}).

После успешной аутентифкации по паролю,
при необходимости, может быть выполнена 
аутентификация терминала и КТ (см.~\ref{Oper.Seq.BAUTH});
при этом, в сертификате терминала должно быть задано право
выработки подписи (см. \ref{DATA.Access}).

Выработка подписи производится прикладной программой eSign, которая
должна быть предварительно выбрана (см.~\ref{Oper.Descr.SelectApp}).

Для выработке подписи  должна использоваться 
следующая последовательность операций:
%
\begin{enumerate}
\item Если статус подтверждения пароля PIN не установлен,
      подтвердить пароль PIN (см.~\ref{Oper.Seq.ControlPIN}).
\item Инициализировать алгоритм выработки подписи (см.~\ref{Oper.Descr.SetDST}).
\item Выработать подпись (см.~\ref{Oper.Descr.Signature}).
\end{enumerate}
%
Указанные последовательности операций должны выполняться
при каждой выработке подписи.

Для проверки статуса подтверждения пароля PIN может использоваться 
соответствующая операция (см.~\ref{Oper.Descr.VerifyAuth}).

\subsection{Разбор токена ключа}
\label{Oper.Seq.Decipher}

Для разбора токена ключа предварительно 
должна быть выполнена аутентификация по 
паролю PIN  (см.~\ref{Oper.Seq.BPACE}).

После успешной аутентифкации по паролю,
при необходимости, может быть выполнена аутентификация 
терминала и КТ (см.~\ref{Oper.Seq.BAUTH}),
при этом в сертификате терминала должно быть задано право
разбора токена ключа (см. \ref{DATA.Access}).

Разбор токена ключа производится прикладной программой eSign, которая
должна быть предварительно выбрана (см.~\ref{Oper.Descr.SelectApp}).

Для разбора токена ключа должна использоваться 
следующая последовательность операций:
%
\begin{enumerate}
%\item Если состояние аутентификации по PIN сброшено,
%      подтвердить пароль PIN (см.~\ref{Oper.Seq.ControlPIN}).
\item Инициализировать алгоритм разбора токена ключа (см.~\ref{Oper.Descr.SetCT}).
\item Разобрать токен ключа (см.~\ref{Oper.Descr.Decipher}).
\end{enumerate}
%
Указанная последовательность операций должна выполняться
при каждом разборе токена ключа.

\section{Формат сообщений защищенного соединения}
\label{CMDS.SM}

Команда и ответ на команду передаются в виде сообщений 
$\text{cmd} = \text{CLA} \parallel \text{INS} \parallel \text{P1} \parallel 
\text{P2} \parallel \text{Lc} \parallel \text{CDF} \parallel \text{Le}$ и 
$\text{res} = \text{RDF} \parallel \text{SW1} \parallel \text{SW2}$ 
соответственно, где 
Lc, CDF, Le и RDF в зависимости от инструкции и параметров команды могут 
быть пустыми словами, т.е. отсутствовать (см.~\ref{CMDS.Intro}). 

При передаче по защищенному соединению команда cmd и ответ res 
преобразуются в защищенную команду 
$\text{cmd*} = \text{CLA*} \parallel \text{INS} \parallel \text{P1} 
\parallel \text{P2} \parallel \text{Lс*} \parallel \text{CDF*} 
\parallel \text{Le*}$ и защищенный ответ 
$\text{res*} = \text{RDF*} \parallel \text{SW1} \parallel \text{SW2}$ 
соответственно. Для защищенных команд и ответов все компоненты имеют 
ненулевую длину. 

Для защищенной команды в CLA* устанавливается один из битов, который 
является признаком защиты, Lc* кодирует длину компонента CDF* (см.~\ref{CMDS.Intro}), 
а Le* всегда устанавливается в $\hex{00}$.  

В CDF* могут включаться зашифрованный компонент CDF (совместно с 
индикатором, который является признаком наличия зашифрованных данных) и 
компонент Le, а также обязательно включается имитовставка, используемая 
для контроля целостности и подлинности команды.  Аналогично защищенной 
команде для защищенного ответа в RDF* может включаться зашифрованный 
компонент RDF (совместно с индикатором наличия зашифрованных данных) и 
обязательно включается имитовставка, используемая для контроля 
целостности и подлинности ответа. 

При формировании компонент CDF* и RDF*, включаемые в них объекты данных, 
кодируются с использованием отличительных правил (см.~\ref{CMDS.Intro}). 
В таблице~\ref{Table.CMDS.CDFRDF} 
приводятся объекты данных, которые могут включаться в CDF* и RDF*, и 
указываются их допустимые длины и теги, используемые при кодировании 
(некоторые из объектов данных могут включаться только в CDF* или RDF*). 

\begin{table}[h]
\caption{Объекты данных компонент CDF* и RDF*}
\label{Table.CMDS.CDFRDF}
\begin{tabular}{|c|c|c|}
\hline
Описание объекта данных & Длина & Тег \\
\hline
\hline
Индикатор совместно с зашифрованными данными & не менее 2 & $\hex{87}$ \\
\hline
Защищенное значение Le & 2 или 3 & $\hex{97}$\\
\hline
Защищенные статусы SW1 и SW2 & 2 & $\hex{99}$ \\
\hline      
Имитовставка & 8 & $\hex{8E}$ \\
\hline
\end{tabular}
\end{table}

Ниже описываются правила кодирования, которые используются при защите 
команд и ответов. При описании данных правил через~$\len(X)$ обозначается 
длина непустого слова~$X$, закодированная минимально возможным количеством 
октетов согласно~\ref{CMDS.Intro} для случая кодирования компонента Lc. 

{\bf Защита команд}. 
Команда защищается с помощью алгоритма~\ref{CRYPTO.SM.Algs.Encr}. 
При этом в 
качестве заголовка $I$ выступает слово 
$\text{CLA} \parallel \text{INS} \parallel \text{P1} \parallel \text{P2}$ 
(4 октета), а 
в качестве критического сообщения $X$ --- слово CDF (кодовое представление 
длины $X$ задается в компоненте Lc). 

На шаге 2 алгоритма установки защиты заголовок $I$ и зашифрованное сообщение 
$Y$ кодируются с помощью следующего алгоритма: 

\begin{enumerate}
\item
Установить $Z \gets I \in \hex{04000000}$.

\item
Если $|Y| > 0$, то $Z \gets Z \parallel \der(\hex{87}, \hex{02}\parallel Y)$.
\item
Если $\text{Le} > 0$, то $Z\gets Z \parallel \der(\hex{97}, \text{Le})$.

\item
Возвратить $Z$ в качестве $\langle\langle I, Y \rangle\rangle$.
\end{enumerate}

На шаге 3 алгоритма установки защиты заголовок $I$, защищенное сообщение $Y$ и 
имитовставка $T$ кодируются с помощью следующего алгоритма: 

\begin{enumerate}
\item
Установить $Z\gets I \in \hex{04000000}$.
\item
Положить переменную W равной пустому слову.
\item
Если $|Y| > 0$, то $W\gets \der(\hex{87}, \hex{02} \parallel Y)$.
\item
Если $Le > 0$, то $W\gets W \parallel \der(\hex{97}, \text{Le})$.
\item
Установить $W\gets W \parallel \der(\hex{8E}, T)$.
\item
Установить $Z\gets Z \parallel \len(W) \parallel W \parallel \hex{00}$.
\item
Возвратить $Z$ в качестве $\langle\langle I, Y, T \rangle\rangle$.
\end{enumerate}

Кодовое представление $\langle\langle I, Y, T \rangle\rangle$, полученное 
в результате работы алгоритма, представляет собой защищенную команду cmd*. 

{\bf Защита ответов}. 
Ответ защищается с помощью алгоритма~\ref{CRYPTO.SM.Algs.Encr}. При этом в 
качестве заголовка~$I$ выступает слово $\text{SW1} \parallel\text{SW2}$ 
(2 октета), а в качестве критического сообщения $X$~--- слово RDF. 
На шаге 2 алгоритма установки защиты заголовок $I$ и защищенное сообщение $Y$ 
кодируются с помощью следующего алгоритма: 

\begin{enumerate}
\item
$Z\gets \der(\hex{99}, I)$.
\item
Если $|Y| > 0$,  то $Z\gets \der(\hex{87}, \hex{02} \parallel Y) \parallel Z$. 
\item
Возвратить $Z$ в качестве $\langle\langle I, Y \rangle\rangle$.
\end{enumerate}

На шаге 3 алгоритма установки защиты заголовок $I$, защищенное сообщение $Y$ и 
имитовставка $T$ кодируются с помощью следующего алгоритма: 

\begin{enumerate}
\item 
Установить $W\gets\der(\hex{8E}, T)$.
\item 
Если $|Y| > 0$, 
то $W \gets \der(\hex{87}, \hex{02}\parallel Y) \parallel W$.
\item
Установить $Z \gets W \parallel \der(\hex{99}, I)$.
\item
Возвратить $Z$ в качестве $\langle \langle I, Y, T\rangle\rangle$.
\end{enumerate}

Кодовое представление  $\langle\langle I, Y, T \rangle\rangle$, 
полученное в результате работы алгоритма, 
представляет собой защищенный ответ res*. 
В защищенном ответе в поле RDF* передается слово $W$, 
которое является кодовым представлением $Y$ и $T$. 

{\bf Снятие защиты с команд и ответов}. 
%
Снятие защиты с команды и ответа производится с помощью 
алгоритма~\ref{CRYPTO.SM.Algs.Decr}.  
При снятии защиты выполняются обратные к установке защиты действия: 
защищенные команда cmd* и ответ res* преобразуются в исходные команду cmd  
и ответ res.  

КТ должен принудительно закрыть защищенное соединение только в том случае, 
когда при снятии защиты с команды обнаружено, что: 
\begin{enumerate}
\item[1)] команда представлена в незащищенном виде;
\item[2)] отсутствует необходимый объект данных;
\item[3)] объект данных является некорректным.
\end{enumerate}

В первом и втором случаях КТ должен вернуть 
статус~$\text{SW1} \parallel \text{SW2} = \hex{6987}$, 
а в третьем случае~--- статус~$\text{SW1} \parallel \text{SW2} = 
\hex{6988}$.  

При принудительном закрытии защищенного соединения КТ должен уничтожить 
ключи, используемые для защиты, и сбросить права доступа в первоначальные. 
