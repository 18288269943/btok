\chapter{Обозначения и сокращения}\label{DEFS}

\section{Обозначения}

В настоящем стандарте применяются обозначения, заданные в ГОСТ 34.973,
а также следующие обозначения:

{\tabcolsep 0pt
\begin{longtable}{lrp{13.5cm}}
$\perp$  & \hspace{2mm} &
специальный объект или ситуация: пустое слово, ошибка;
\\[4pt]
$\{0,1\}^n$  & \hspace{2mm} &
множество всех слов длины $n$ в алфавите~$\{0,1\}$;
\\[4pt]
$\{0,1\}^*$  &&
множество всех слов конечной длины в алфавите~$\{0,1\}$
(включая пустое слово~$\perp$ длины $0$);
\\[4pt]
$|u|$      &&
длина слова $u\in\{0,1\}^*$;
\\[4pt]
%
$\{0,1\}^{n*}$  &&
множество всех слов из~$\{0,1\}^*$,
длина которых кратна~$n$;
\\[4pt]
%
$\alpha^n$  &&
для~$\alpha\in\{0,1\}$ слово длины $n$ из одинаковых символов $\alpha$;
\\[4pt]
%
%
$\left\langle u\right\rangle_n$  &&
для $u\in\{0,1\}^*$
слово из первых~$n$ символов~$u$, $n\leq|u|$;
\\[4pt]
%
$u\parallel v$  &&
конкатенация
$u_1 u_2\ldots u_n v_1 v_2\ldots v_m$
слов
$u=u_1 u_2\ldots u_n$ и
$v=v_1 v_2\ldots v_m$;
\\[4pt]
%
$u\wedge v$             &&
для~$u=u_1 u_2\ldots u_n\in\{0,1\}^n$ 
и~$v=v_1 v_2\ldots v_n\in\{0,1\}^n$
слово~$w=w_1 w_2\ldots w_n\in\{0,1\}^n$
из символов~$w_i= u_i * v_i$
(посимвольное И);
\\[4pt]
%
$\text{(символы~\texttt{0}--\texttt{F})}_{16}$ && 
представление $u\in\{0,1\}^{4*}$ шестнадцатеричным словом,
при котором последовательным четырем символам~$u$ соответствует
один шестнадцатеричный символ
(например, $10100010=\texttt{A2}_{16}$);
\\[4pt]
%
$x\bmod m$             &&
для целого числа~$x$ и натурального числа~$m$ 
остаток от деления~$x$ на~$m$,
т.~е. число $r\in\{0,1,\ldots,m-1\}$ такое, что~$m$ делит $x-r$;
\\[4pt]
%
$\btoi{u}$                &&
а)~для~$u=u_1 u_2\ldots u_8\in\{0,1\}^8$
число $2^7 u_1+2^6 u_2+\ldots+u_8$ и\\[2pt]
%
                        &&
б)~для~$u=u_1\parallel u_2\parallel\ldots\parallel u_n$, $u_i\in\{0,1\}^8$,
число~$\btoi{u_1}+2^8\btoi{u_2}+\ldots+2^{8(n-1)}\btoi{u_n}$;
\\[4pt]
%
$\langle U\rangle_{8n}$ &&
для целого числа~$U$ 
слово $u\in\{0,1\}^{8n}$ такое, что $\btoi{u}=U\bmod 2^{8n}$;
\\[4pt]
%
$c\leftarrow u$         &&
присвоение переменной $c$ значения $u$;
\\[4pt]
%
$c\stackrel{R}{\leftarrow} U$    &&
случайный равновероятный (или псевдослучайный)
выбор~$c$ из множества~$U$;
\\[4pt]
%
$A\to B$    &&
для сторон~$A$ и~$B$ передача сообщения от $A$ к~$B$;
\\[4pt]
%
$\mathoptional{\text{текст}}$ &&
необязательное сообщение (действие) протокола;
\\[4pt]
%
$\mathimplicit{\text{текст}}$ &&
обязательное сообщение (действие) протокола, 
которое может быть передано (выполнено) предварительно,
до сеанса протокола, или неявно;
\\[4pt]
%
$\Cert(Id,Q)$ &&
сертификат, связывающий идентификационные данные~$Id$ определенной 
стороны с ее открытым ключом~$Q$;\\
%
$\hello$ &&
приветственное сообщение;
\\[4pt]
%
$\langle\langle d_1,d_2,\ldots,d_n\rangle\rangle$ &&
кодовое представление структурированных данных,
содержащих, по крайней мере, компоненты $d_1, d_2,\ldots,d_n$.
\\[4pt]
\end{longtable}
} % tabcolsep
\setcounter{table}{0}

\section{Пояснения к обозначениям}

\subsection{Слова}

Двоичные слова представляют собой последовательности символов из 
алфавита~$\{0,1\}$. Символы нумеруются слева направо от единицы.
%
В настоящем подразделе в качестве примера рассматривается слово
$$
w=1011 0001 1001 0100 1011 1010 1100 1000.
$$
В этом слове первый символ~--- $1$, 
второй~--- $0$, \ldots, последний~--- $0$.

Слова разбиваются на тетрады из четверок последовательных двоичных символов.
%
Тетрады кодируются шестнадцатеричными символами по следующим правилам
(см. таблицу~\ref{Table.Hex}):

\begin{table}[H]
\caption{}\label{Table.Hex}
\begin{tabular}{|c|c||c|c||c|c||c|c|}
\hline
тетрада & символ & тетрада & символ & тетрада & символ & тетрада & символ\\
\hline
\hline
0000 & $\texttt{0}_{16}$ & 0001 & $\texttt{1}_{16}$ & 
0010 & $\texttt{2}_{16}$ & 0011 & $\texttt{3}_{16}$\\
0100 & $\texttt{4}_{16}$ & 0101 & $\texttt{5}_{16}$ & 
0110 & $\texttt{6}_{16}$ & 0111 & $\texttt{7}_{16}$\\ 
1000 & $\texttt{8}_{16}$ & 1001 & $\texttt{9}_{16}$ & 
1010 & $\texttt{A}_{16}$ & 1011 & $\texttt{B}_{16}$\\ 
1100 & $\texttt{C}_{16}$ & 1101 & $\texttt{D}_{16}$ & 
1110 & $\texttt{E}_{16}$ & 1111 & $\texttt{F}_{16}$\\ 
\hline
\end{tabular}
\end{table}

Пары последовательных тетрад образуют октеты.
Последовательные октеты слова~$w$ имеют вид:
$$
1011 0001=\texttt{B1}_{16},\ 
1001 0100=\texttt{94}_{16},\ 
1011 1010=\texttt{BA}_{16},\  
1100 1000=\texttt{C8}_{16}.
$$

\subsection{Слова как числа}

Октету $u=u_1 u_2\ldots u_8$ ставится в соответствие байт~--- 
число $\btoi{u}=2^7u_1+2^6 u_2+\ldots + u_8$. 
Например, октетам $w$ соответствуют байты
$$
177=2^7+2^5+2^4+1,\ 
148=2^7+2^4+2^2,\ 
186=2^7+2^5+2^4+2^3+2^1,\ 
200=2^7+2^6+2^3.
$$

Число ставится в соответствие не только октетам, но и любому другому
двоичному слову, длина которого кратна~$8$. 
%
При этом используется распространенное для многих современных 
процессоров соглашение <<от младших к старшим>> (little-endian):
считается, что первый байт~--- младший, последний~--- старший.
Например, слову $w$ соответствует число
$$
\btoi{w}=177+2^{8}\cdot 148+2^{16}\cdot 186+2^{24}\cdot 200 = 3367670961.
$$

\subsection{Обозначения протоколов}

Протоколы настоящего стандарта выполняют две стороны.
Объекты и атрибуты (ключи, сообщения, идентификаторы, переменные)
определенной стороны снабжаются нижним индексом,
указывающим на эту сторону.

Квадратными скобками окаймляются 
необязательные сообщения и действия сторон протокола.

Фигурными скобками окаймляются сертификаты~(см.~\ref{CERTS})
и приветственные сообщения,
которые могут передаваться предварительно или неявно. 
Действия по передаче и обработке таких сообщений
также окаймляются фигурными скобками.

\subsection{Обозначения данных}

В настоящем стандарте в некоторых случаях применяется двухступенчатая 
схема описания данных. На первой ступени данные описываются рамочно, без 
исчерпывающих деталей. Запись $D = \langle\langle d_1, d_2,\ldots, d_n\rangle\rangle$ 
означает, что слово $D \in\{0,1\}^*$ является кодовым представлением 
структурированных данных, содержащих, по крайней мере, 
компоненты~$d_1, d_2,\ldots, d_n$. 
%
Например, $\Cert(Id,Q)=\langle\langle Id,Q\rangle\rangle$.
%
На второй ступени  уточняются список компонентов, правила кодирования
компонентов и всей структуры в целом.

В настоящем стандарте уточнения даются с помощью АСН.1.
В других ТНПА уточнения могут быть даны другими способами.
Уточнения с помощью АСН.1 собраны в модуль, представленный 
в приложении~\ref{ASN}. 

\section{Сокращения}

В настоящем стандарте применяются следующие сокращения:

АСН.1 -- абстрактно-синтаксическая нотация версии 1 (ГОСТ 34.973);

КП -- клиентская программа;

КТ -- криптографический токен;

ПС -- прикладная система (СТБ 34.101.78);

СИ -- служба идентификации (СТБ 34.101.78);

УЦ -- удостоверяющий центр (СТБ 34.101.19);

ЭЦП -- электронная цифровая подпись.


