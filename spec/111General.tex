\section{Общие сведения}\label{CMDS.Intro}

КП и терминал взаимодействуют с КТ посредством передачи команд APDU 
(аббревиатура от Application Protocol Data Unit), определенных в~\cite{APDU}. 
При получении команды КТ обрабатывает ее и возвращает ответ на команду. При 
этом до подачи на КТ новой команды должен быть получен ответ на 
предыдущую. 

Команды и ответы на них содержат обязательные компоненты и дополнительно 
могут содержать необязательные компоненты.  

Для команды обязательными компонентами являются: CLA~--- класс команды, 
INS~--- инструкция команды, P1 и P2~--- параметры команды. В свою очередь для 
ответа обязательными являются статусы обработки команды SW1 и SW2.  

Необязательным для команды является компонент CDF, который содержит данные 
команды. При этом если в команде присутствует компонент CDF, то команда 
должна также содержать необязательный компонент Lc, определяющий длину 
компонента CDF. 

Необязательным для ответа является компонент RDF, который содержит данные 
ответа. Если при выполнении команды ожидается, что в ответе будет 
содержаться компонент RDF, то команда должна содержать необязательный 
компонент Le, определяющий максимально возможную длину компонента RDF в 
ожидаемом ответе. 

В~\cite{APDU} определяются соглашения по возможным значениям компонент команд и 
ответов, рассматривается возможные способы представления компонент и 
взаимосвязь между ними. В частности, согласно~\cite{APDU} при наличии в команде 
компонентов Lc и Le они могут быть представлены в коротком или расширенном 
виде (они должны одновременно иметь либо короткий, либо расширенный вид). 
При этом они должны кодироваться по следующим правилам: 

\begin{enumerate}
\item[1)]
Lc в коротком виде состоит из одного октета, отличного от $\hex{00}$ и 
определяющего значения от 1 до 255; 

\item[2)] 
Lc в расширенном виде состоит из трех октетов, при этом первый октет 
равен $\hex{00}$, а остальные два октета отличны от 
$\hex{0000}$ и определяют значения от 1 до 65535; 

\item[3)] 
Le в коротком виде состоит из одного октета, определяющего значения 
от 1 до 256 (значению 256 соответствует $\hex{00}$); 

\item[4)] 
если компонент Lc присутствует в команде, то Le в расширенном виде 
состоит из двух октетов, которые определяют значения от 1 до 65536 
(значению 65536 соответствует $\hex{0000}$); 

\item[5)] 
если компонент Lc отсутствует в команде, то Le в расширенном виде 
состоит из трех октетов, при этом первый октет равен $\hex{00}$ и 
следует за двумя другими, которые определяют значения от 1 до 65536 
(значению 65536 соответствует $\hex{0000}$). 
\end{enumerate}

В таблице~\ref{Table.CMDS.Fmt} приводятся формат пары (команда, ответ) 
с указанием возможных длин компонентов.

\begin{table}[h]
\caption{Формат пары (команда, ответ)}\label{Table.CMDS.Fmt}
\begin{tabular}{|c|p{10.5cm}|c|}
\hline
Компонент & Описание & Длина в октетах \\
\hline
\hline
CLA & Класс команды & 1 \\
\hline
INS & Инструкция команды & 1 \\
\hline
P1 & Параметр команды & 1 \\
\hline
P2 & Параметр команды & 1 \\
\hline
Lc & Закодированная длина данных команды & 0 или 3  \\
\hline
CDF & Данные команды & 0~--- 65 535 \\
\hline
Le & Закодированная максимально возможная длина данных ответа & 0, 2 или 3 
\\
\hline
RDF & Данные ответа & 0~--- 65 536 \\
\hline
SW1 & Статус ответа & 1 \\
\hline
SW2 & Статус ответа & 1 \\
\hline
\end{tabular}
\end{table}

При формировании компонент CDF и RDF, включаемые в них данные могут быть 
представлены явно или в закодированном виде. При кодировании используются 
отличительные правила, описанные в СТБ 34.101.19 (приложение Б). В 
настоящем стандарте, если не оговорено противное, используются 
контекстно-зависимые теги, состоящие из одного или двух октетов, а данные 
кодируются как строки октетов (см. ГОСТ 34.974) и не могут превышать 65515 
октетов. При таком кодировании по данным $X\in\{0,1\}^{8*}$ 
с тегом $T\in\{0,1\}^{8*}$ строится строка 
октетов~$\der(T, X) = T \parallel L \parallel X$, 
где $L\in\{0,1\}^{8*}$~--- 
закодированная длина $X$ [см. ГОСТ 34.974 (п. 6.3)]. 

В случае если при разборе команды или ответа обнаруживается нарушение их 
формата или формата закодированных данных, то должна быть возвращена 
ошибка и соответствующее сообщение не должно обрабатываться. 

Команды используются для реализации операций, выполняемых на КТ.
Минимальный набор операций, который должен быть поддержан для КТ,
представлен в таблице~\ref{Table.Oper.List} и описывается
в~\ref{Oper.Descr}. 
В таблице для операций указываются состояния, 
в которых они могут выполняться (см.~\ref{STATES.ST}),
пароль, который должен использоваться 
в протоколе BPACE при аутентификации (см.~\ref{OBJ.PWD}), 
и уровень (мастер-файл, прикладная программа), 
на котором может выполняться операция 
(см.~\ref{FILES}).
Символ <</>> в таблице обозначает <<или>>.

% todo: AS - взаимная или односторонняя аутентификация, не всегда понятно? 

\begin{table}[p]
\caption{Операции КТ}
\label{Table.Oper.List}
\begin{tabular}{|p{7.5cm}|p{1.3cm}|p{2.6cm}|p{1.8cm}| p{1.7cm}|}
\hline
Операция & Пункт & Состояние и соединение & Пароль BPACE & Уровень \\
\hline
\hline
%
Активация личного ключа & \ref{Oper.Descr.ActivateKey} & PS/AS:AT & PIN & eSign \\
\hline
%
Активация PIN & \ref{Oper.Descr.ActivatePIN} & PS/AS:AT & PUK & eID/eSign \\
\hline
%
Выбор мастер-файла & \ref{Oper.Descr.SelectMF} & IS/PS/AS & любой & все \\
\hline
%
Выбор прикладной программы & \ref{Oper.Descr.SelectApp} & PS/AS & любой & все \\
\hline
%
Выбор элементарного файла eID & \ref{Oper.Descr.SelectEF} & AS:AT & CAN/PIN & eID \\
\hline
%
Выбор элементарного файла eSign & \ref{Oper.Descr.SelectEF} & PS/AS:AT & PIN & eSign \\
\hline
%
Выполнение основных шагов протокола BAUTH & \ref{Oper.Descr.GABAUTH} & PS & любой & MF \\
\hline
%
Выполнение шагов протокола BPACE & \ref{Oper.Descr.GABPACE} & IS/PS/AS:CP & ---/любой & MF \\
\hline
%
Выработка подписи & \ref{Oper.Descr.Signature} & PS/AS:AT & PIN & eSign \\
\hline
%
Генерация ключевой пары & \ref{Oper.Descr.GenKeys} & PS/AS:AT & PIN & eSign \\
\hline
%
Деактивация личного ключа & \ref{Oper.Descr.DeactivateKey}  & PS/AS:AT & PIN & eSign \\
\hline
%
Деактивация PIN & \ref{Oper.Descr.DeactivatePIN}  & PS/AS:AT & PIN/PUK & eID/eSign \\
\hline
%
Изменение PIN & \ref{Oper.Descr.ChangePIN} & PS/AS:CP & PIN & eID/eSign \\
\hline
%
Инициализация алгоритма выработки подписи & \ref{Oper.Descr.SetDST} & PS/AS:AT & PIN & eSign \\
\hline
%
Инициализация алгоритма разбора токена ключа & \ref{Oper.Descr.SetCT} & PS/AS:AT & PIN & eSign \\
\hline
%
Инициализация протокола BAUTH & \ref{Oper.Descr.SetBAUTH} & PS & любой & MF \\
\hline
%
Инициализация протокола BPACE & \ref{Oper.Descr.SetBPACE} & IS/PS/AS:CP & ---/любой & MF \\
\hline
%
Обновление данных eID & \ref{Oper.Descr.Update} & AS:AT & CAN/PIN & eID \\
\hline
%
Обновление данных eSign & \ref{Oper.Descr.Update} & PS/AS:AT & PIN & eSign \\
\hline
%
Переключение между соединениями & \ref{Oper.Descr.SetCS} & AS & PIN/PUK & eID/eSign \\
\hline
%
Подтверждение PIN & \ref{Oper.Descr.VerifyPIN} & PS/AS:CP & PIN & eSign \\
\hline
%
Получение значений дополнительных атрибутов & \ref{Oper.Descr.VerifyData}& AS:AT & CAN/PIN & eID \\
\hline
%
Проверка сертификата & \ref{Oper.Descr.VerifyCert} & PS & любой & MF \\
\hline
%
Проверка статуса подтверждения PIN & \ref{Oper.Descr.VerifyAuth} & PS/AS:AT & PIN & eSign \\
\hline
%
Разблокировка PIN & \ref{Oper.Descr.UnblockPIN} & PS/AS:CP  & PUK & eID/eSign \\
\hline
%
Разбор токена ключа & \ref{Oper.Descr.Decipher} & PS/AS:AT & PIN & eSign \\
\hline
%
Сброс статуса подтверждения PIN & \ref{Oper.Descr.VerifyDeauth} & PS/AS:AT  & PIN & eSign \\
\hline
%
Чтение данных eID & \ref{Oper.Descr.Read} & AS:AT & CAN/PIN & eID \\
\hline
%
Чтение данных eSign & \ref{Oper.Descr.Read} & PS/AS:AT& PIN & eSign \\
\hline
%
Уничтожение личного ключа & \ref{Oper.Descr.Terminate} & PS/AS:AT  & PIN & eSign \\
\hline
\end{tabular}
\end{table}

Типичные последовательности операций, которые могут выполняться 
с использованием КТ, описываются в~\ref{Oper.Seq}.

В~\ref{CMDS.SM} описывается преобразование команд и ответов из исходной формы в 
защищенную и обратно при использовании защищенного соединения (см.~\ref{CRYPTO.SM}). 


